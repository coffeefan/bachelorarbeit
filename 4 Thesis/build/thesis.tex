%%%%%%%%%%%%%%%%%%%%%%%%%%%%%%%%%%%%%%%%%%%%%%%%%%%%%%%%%%%%%%%%%
%  _____   ____  _____                                          %
% |_   _| /  __||  __ \    Institute of Computitional Physics   %
%   | |  |  /   | |__) |   Zuercher Hochschule Winterthur       %
%   | |  | (    |  ___/    (University of Applied Sciences)     %
%  _| |_ |  \__ | |        8401 Winterthur, Switzerland         %
% |_____| \____||_|                                             %
%%%%%%%%%%%%%%%%%%%%%%%%%%%%%%%%%%%%%%%%%%%%%%%%%%%%%%%%%%%%%%%%%
%
% Project     : LaTeX doc Vorlage für Windows ProTeXt mit TexMakerX
% Title       : 
% File        : header.tex Rev. 00
% Date        : 23.4.12
% Author      : Remo Ritzmann
% Feedback bitte an Email: remo.ritzmann@pfunzle.ch
%
%%%%%%%%%%%%%%%%%%%%%%%%%%%%%%%%%%%%%%%%%%%%%%%%%%%%%%%%%%%%%%%%%

\documentclass[ oneside,openright,titlepage,numbers=noenddot,headinclude,%1headlines,% letterpaper a4paper
                BCOR=5mm,paper=a4,fontsize=11pt,%11pt,a4paper,%
                ngerman,american,%
                ]{scrreprt}

%***********************************************************************
% include some libs
%***********************************************************************
\usepackage[utf8]{inputenc}
\usepackage{listings}
\usepackage{color}
\usepackage{fancyhdr}
\usepackage{rotating}
\usepackage{titlesec}
\usepackage{mathptmx}
% \usepackage{helvet}
\usepackage[scaled]{uarial}
\renewcommand*\familydefault{\sfdefault} %% Only if the base font of the document is to be sans serif
\usepackage[T1]{fontenc}
\usepackage{ngerman}
\usepackage{textgreek}
\usepackage{textcomp}
\usepackage[squaren]{SIunits}
\usepackage{graphicx}
\usepackage{url}
\usepackage{geometry}
\usepackage[absolute]{textpos}
\usepackage{makeidx}
\usepackage{colortbl}
\usepackage{pdflscape}
\usepackage{pdfpages}
\usepackage{tabularx}
\usepackage{lmodern}
\usepackage{longtable}
\usepackage{array}
\usepackage{float}
\usepackage{scrhack}

\definecolor{iagreen}{RGB}{43,108,28}
%
\usepackage[colorlinks=true,linkcolor=blue,citecolor=iagreen,filecolor=black,urlcolor=red,linktoc=page,plainpages=false, pdfborder={0 0 0}]{hyperref}
\usepackage{wallpaper} %\ThisTileWallPaper{}
%usepackage[super,square]{natbib} für BibTeX Literaturverzeichnis
\usepackage{enumitem}
\usepackage{subfig}
\usepackage[export]{adjustbox}
\usepackage{setspace}
\usepackage{amsmath} 
\usepackage{fancyvrb}
\usepackage{graphicx}
\usepackage{booktabs}


\usepackage{etoolbox}
\apptocmd{\thebibliography}{\setlength{\itemsep}{18pt}}{}{}





% Bibliography
\usepackage[
backend=bibtex,
style=authoryear,
]{biblatex}
\bibliography{Citer.bib}

\setlength\bibitemsep{1.5\itemsep}

\DeclareCiteCommand{\cite}
  {\usebibmacro{prenote}}
  {\usebibmacro{citeindex}%
   \printtext[bibhyperref]{\usebibmacro{cite}}}
  {\multicitedelim}
  {\usebibmacro{postnote}}

\DeclareCiteCommand*{\cite}
  {\usebibmacro{prenote}}
  {\usebibmacro{citeindex}%
   \printtext[bibhyperref]{\usebibmacro{citeyear}}}
  {\multicitedelim}
  {\usebibmacro{postnote}}

\DeclareCiteCommand{\parencite}[\mkbibparens]
  {\usebibmacro{prenote}}
  {\usebibmacro{citeindex}%
    \printtext[bibhyperref]{\usebibmacro{cite}}}
  {\multicitedelim}
  {\usebibmacro{postnote}}

\DeclareCiteCommand*{\parencite}[\mkbibparens]
  {\usebibmacro{prenote}}
  {\usebibmacro{citeindex}%
    \printtext[bibhyperref]{\usebibmacro{citeyear}}}
  {\multicitedelim}
  {\usebibmacro{postnote}}

\DeclareCiteCommand{\footcite}[\mkbibfootnote]
  {\usebibmacro{prenote}}
  {\usebibmacro{citeindex}%
  \printtext[bibhyperref]{ \usebibmacro{cite}}}
  {\multicitedelim}
  {\usebibmacro{postnote}}

\DeclareCiteCommand{\footcitetext}[\mkbibfootnotetext]
  {\usebibmacro{prenote}}
  {\usebibmacro{citeindex}%
   \printtext[bibhyperref]{\usebibmacro{cite}}}
  {\multicitedelim}
  {\usebibmacro{postnote}}

\DeclareCiteCommand{\textcite}
  {\boolfalse{cbx:parens}}
  {\usebibmacro{citeindex}%
   \printtext[bibhyperref]{\usebibmacro{textcite}}}
  {\ifbool{cbx:parens}
     {\bibcloseparen\global\boolfalse{cbx:parens}}
     {}%
   \multicitedelim}
  {\usebibmacro{textcite:postnote}}





\usepackage{calc}




\lstdefinestyle{CSstyle} {
  language=[Sharp]C
}

%***********************************************************************
% various styles
%***********************************************************************	

%create index
\makeindex

%define pagestyle
\pagestyle{fancy}

%use sans-serif font 
%\renewcommand{\familydefault}{\sfdefault}

%define page margin
\geometry{a4paper, top=30mm, left=30mm, right=30mm, bottom=30mm,headsep=10mm,footskip=10mm}

%textpos parameter
\setlength{\TPHorizModule}{30mm}
\setlength{\TPVertModule}{\TPHorizModule}
\textblockorigin{10mm}{10mm} % start everything near the top-left corner
\setlength{\parindent}{0pt}

%horizontal lines for titlepage 
\newcommand{\HRule}{\rule{\linewidth}{0.5mm}}

%reference to source items inlc source number
\newcommand{\srcref}[1]{\nameref{src:#1} \cite{#1}}

%header / footer 
\renewcommand{\headrulewidth}{0.3pt}
\renewcommand{\footrulewidth}{0.3pt}

\fancyhead[LO,RE]{} %clear headings for contents 

\fancyhead[RO,LE]{\nouppercase{\rightmark}} %right odd pages and left even pages
\fancyhead[LO,RE]{\MakeUppercase{\leftmark}} %left odd pages and right even pages
%\fancyhead[LO,RE]{\fontsize{9}{9}}
\fancyfoot[LE,RO]{\thepage} %page numbering
\fancyfoot[C]{} %clear centered page numbering 

%define some colors
\definecolor{gray}{rgb}{0.95,0.95,0.95}
\definecolor{darkgray}{rgb}{0.4,0.4,0.4}
%listing colors
\definecolor{lgray}{RGB}{250,250,250}
\definecolor{lgreen}{RGB}{63,127,95}
\definecolor{lred}{RGB}{127,0,85}
\definecolor{lblue}{RGB}{42,0,255}

%***********************************************************************
% listing
%***********************************************************************

\lstset{		
		basicstyle=\small\ttfamily,
		frame=single,
		numbers=left,	
		numberstyle=\tiny,
		%firstnumber=auto,
		numberblanklines=true,
		captionpos=b,
		extendedchars=true,
		float=ht,
		showtabs=false,
		tabsize=2,
		showspaces=false,
		showstringspaces=false,
		breaklines=true,
		%prebreak=\Righttorque,
		backgroundcolor=\color{lgray},
		keywordstyle=\color{lred}\bfseries, 
		commentstyle=\color{lgreen}\ttfamily,
%		morekeywords={printstr, printhexln},
		stringstyle=\color{lblue},
		xleftmargin=0.5cm,
		xrightmargin=0.5cm
}

\lstloadlanguages{R}

%\lstdefinelanguage{xc}{
%     keywords={printstr, printhexln, attributes, class, classend, do, empty, endif, endwhile, fail, function, functionend, if, implements, in, inherit, inout, not, of, operations, out, return, set, then, types, while, use},
%     keywordstyle=\color{lred}\bfseries,
%     ndkeywords={},
%     ndkeywordstyle=\color{yellow}\bfseries,
%     identifierstyle=\color{black},
%     sensitive=false,
%     comment=[l]{//},
%     commentstyle=\color{lgreen}\ttfamily,
%     string=[l]{"},
%     stringstyle=\color{lblue}\ttfamily
%  }






\begin{document}
\include{titlepage}


% We will generate all images so they have a width \maxwidth. This means
% that they will get their normal width if they fit onto the page, but
% are scaled down if they would overflow the margins.
\makeatletter
\def\maxwidth{\ifdim\Gin@nat@width>\linewidth\linewidth
\else\Gin@nat@width\fi}
\makeatother
\let\Oldincludegraphics\includegraphics
\renewcommand{\includegraphics}[1]{\Oldincludegraphics[width=\maxwidth]{#1}}

\VerbatimFootnotes

\setlength{\parindent}{0pt}
\setlength{\parskip}{6pt plus 2pt minus 1pt}
\setlength{\emergencystretch}{3em}  % prevent overfull lines
\providecommand{\tightlist}{%
  \setlength{\itemsep}{0pt}\setlength{\parskip}{0pt}}
\VerbatimFootnotes % allows verbatim text in footnotes




\setcounter{page}{1}


%Inhaltsverzeichnis
\tableofcontents
\newpage

\chapter{Einführung}\label{einfuxfchrung}

\section{Motivation}\label{motivation}

Die Digitalisierung fordert die Schweizer Wirtschaft heraus. Ob Banken,
Pharmaindustrie, Detailhandel oder Medienhäuser -- es gibt keine
Branche, die nicht vor fundamentalen Veränderungen steht.\footnote{\autocite{digitalerevolutionhz}}
Da verwundert es nicht, dass Wettbewerbe oder Kreuzworträtsel nicht nur
auf den letzten Seiten der Klatschheftchen oder Zeitungen abgedruckt
werden, sondern vermehrt online publiziert und durchgeführt werden. Dass
bei meinungsbildenden Umfragen oder Abstimmungen weniger auf
Telefonumfragen zurückgegriffen wird, sondern diese immer mehr im
Internet durchgeführt werden, ist ebenso wenig erstaunlich.

In der Schweiz konnten die grossen Medienhäuser ihre Zugriffszahlen auch
2015 steigern und ihre Toprangierungen beibehalten.\footnote{\autocite{netmatrixaudit}}
Um ihren Werbegewinn und Resonanz zu bewahren oder sogar auszubauen,
sind Medien darauf angewiesen, dass ihre Stories Inhalte auf den Social
Media Kanälen verlinkt und so viral verbreitet werden. Neben
altbekannten plakativen Titeln und interessanten Bildern beleben die
Medienhäuser immer mehr ihren Inhalt mit so genannten ``Playfull
Contents'' oder auf Deutsch: Mit Interaktivitäten. Dabei handelt es sich
um Abstimmungen, Wettbewerbe und Umfragen oder andere Interaktivitäten
im Zusammenhang mit dem verfassten Inhalt. Diese Social-Module werden
gerne verlinkt und fördern so die Verbreitung des Contents und dadurch
einen Anstieg der Besucherzahlen.

Bei den meisten angebotenen Umfragen, Abstimmungen und Wettbewerben ist
es relativ simpel (gewisses Know-How vorausgesetzt) mehrfach
teilzunehmen oder gefälschte Daten zu übermitteln. Dies ist auf zu
einfach realisierte Programmierungen zurückzuführen, was der
Glaubwürdigkeit solcher Angebote schadet. Interaktivitäten bedürfen
somit einer Authentifizierung, um Betrug oder falschen Stimmabgaben
vorzubeugen. Die Eigenentwicklung der gewünschten Authentifizierung für
eine Interaktivität übersteigt meist die kleinen Budgets für diese
Angebote.

Die Glaubwürdigkeit der Umfragen, Abstimmungen und Wettbewerbe ist durch
die aktuelle Situation gefährdet und soll wiederhergestellt werden.
Deshalb soll diese Bachelorarbeit die Möglichkeit eines
Authentifizierungsservices erörtern. Mit dieser sollen Programmierer
über eine visuelle Oberfläche die Bedürfnisse eines Angebots
konfigurieren und in ihren jeweiligen Modulen einbinden können.

\newpage

\section{Aufgabenstellung}\label{aufgabenstellung}

\subsection{Ausgangslage}\label{ausgangslage}

Bei populären Medienhäusern und grösseren Unternehmen werden häufig
Umfragen, Abstimmungen oder Gewinnspiele im Internet durchgeführt. Bei
den meisten angebotenen Programmen ist es relativ simpel (gewisses
Know-How vorausgesetzt) mehrfach teilzunehmen oder gefälschte Daten zu
übermitteln. Dies ist auf zu einfach realisierte Programmierungen
zurückzuführen, was der Glaubwürdigkeit solcher Angebote schadet.
Social-Media Module wie Umfragen, Abstimmungen oder Wettbewerbe bedürfen
somit einer Authentifizierung, um Betrug oder falschen Stimmabgaben
vorzubeugen. Die Eigenentwicklung der gewünschten Authentifizierung für
ein Modul übersteigt meist die kleinen Budgets für diese Angebote. Die
Firma inaffect AG erstellt Individuallösungen und Webapplikationen im
Bereich neuer Medien. Sie ist auf der Suche nach einem
Authentifizierungsservice, welche ihre Programmierer mit einer visuellen
Oberfläche den Bedürfnissen eines Angebots konfigurieren und in ihr
jeweiliges Modul einbinden können.

\subsection{Ziel der Arbeit}\label{ziel-der-arbeit}

Es soll ein Konzept für eine Authentifizierungsschnittstelle erstellt
werden. Dieser Service wird über mehrere Sicherheitsstufen verfügen, die
sich in der Menge und Art der zu übermittelnden User-Informationen
unterscheiden. Diese Stufen sollen für den Programmierer eines Angebots
über eine grafische Oberfläche individuell konfigurierbar sein. Das
Konzept soll in Form eines Prototypen umgesetzt werden. Weiter soll mit
mehreren Usern eine Studie zur Akzeptanz und Geschwindigkeit der
verschiedenen Sicherheitsstufen durchgeführt werden. Die Ergebnisse der
Studie werden im Prototyp integriert sein und sollen den Programmierer
bei der Auswahl der Sicherheitsstufe unterstützen.

\subsection{Aufgabenstellung}\label{aufgabenstellung-1}

Im Rahmen der Bachelorarbeit werden vom Studenten folgende Aufgaben
durchgeführt:

Recherche

\begin{itemize}
\tightlist
\item
  Research und Marktanalyse bestehender Produkte
\item
  Arten und Methoden der Sicherheits- und Identitätsüberprüfung
\item
  Durchführung einer Anforderungsanalyse für eine
  Authentifizierungsschnittstelle
\end{itemize}

Konzept

\begin{itemize}
\tightlist
\item
  Evaluation von geeigneten Authentifizierungsmethoden für verschiedene
  Stufen
\item
  Spezifikation einer Prototypenapplikation für die
  Authentifizierungsschnittstelle
\item
  Spezifikation einer Prototypenapplikation für das Verwalten der
  Authentifizierungsschnittstelle
\item
  Erstellen einer Software-Architektur für die
  Authentifizierungsschnittstelle und dessen Verwaltung
\item
  Ausarbeiten einer Studie über Akzeptanz und Geschwindigkeit von
  Authentifizierungsmethoden
\end{itemize}

\newpage

Studie

\begin{itemize}
\tightlist
\item
  Durchführen der ausgearbeiteten Studie
\item
  Auswertung der Studie
\end{itemize}

Proof of Concept

\begin{itemize}
\tightlist
\item
  Entwicklung eines Prototypen der Authtenifizierungsschnittstelle und
  der Verwaltung, basierend auf den erarbeiteten Spezifikationen und
  Architektur
\item
  Integration der Studienresultate im Prototypen
\end{itemize}

Fazit

\subsection{Erwartete Resultate}\label{erwartete-resultate}

Im Rahmen dieser Bachelorarbeit werden vom Studenten folgende Resultate
erwartet:

Recherche

\begin{itemize}
\tightlist
\item
  Dokumentation des Research und Marktanalyse bestehender Produkte
\item
  Dokumentation der Arten und Methoden der Sicherheits- und
  Identitätsüberprüfung
\end{itemize}

Analyse

\begin{itemize}
\tightlist
\item
  Dokumentierte Anforderungsanalyse für eine
  Authentifizierungsschnittstelle
\end{itemize}

Konzept

\begin{itemize}
\tightlist
\item
  Dokumentation der Evaluation von geeigneten Authentifizierungsmethoden
  für verschiedene Stufen
\item
  Dokumentierte Spezifikation einer Prototypenapplikation für die
  Authentifizierungsschnittstelle
\item
  Dokumentierte Spezifikation einer Prototypenapplikation für das
  Verwalten der Authentifizierungsschnittstelle
\item
  Dokumentation der Software-Architektur für die
  Authentifizierungsschnittstelle und dessen Verwaltung
\item
  Dokumentation des Ausarbeitens einer Studie über Akzeptanz und
  Geschwindigkeit von Authentifizierungsmethoden
\end{itemize}

Studie

\begin{itemize}
\tightlist
\item
  Dokumentation der Studien-Durchführung
\item
  Dokumentation der Auswertung der Studie
\end{itemize}

Proof of Concept

\begin{itemize}
\tightlist
\item
  Dokumentierte Entwicklung eines Prototypen der
  Authentifizierungsschnittstelle und der Verwaltung, basierend auf den
  erarbeiteten Spezifikationen und Architektur
\item
  Dokumentierte Integration der Studienresultate im Prototypen
\item
  Dokumentiertes Fazit
\end{itemize}

\hypertarget{rahmenbedingungen-bachelorarbeit}{\section{Rahmenbedingungen
Bachelorarbeit}\label{rahmenbedingungen-bachelorarbeit}}

Die vorliegende Bachelorarbeit umfasst gemäss Regelment unter anderem
folgende Punkte:

\begin{itemize}
\tightlist
\item
  Eine Bachelorarbeit besteht aus einer konzeptionellen Arbeit und deren
  Umsetzung. Der Schwerpunkt liegt auf dem konzeptionellen Teil, in dem
  die theoretischen und methodischen Grundlagen einer Entwicklung oder
  eines Konzeptes ausgearbeitet und dargelegt werden. Im Umsetzungsteil
  erfolgt anschliessend die Beschreibung der Implementierung bzw. der
  Anwendung. Die Umsetzung besteht zumindest aus einem „Proof of
  Concept``, um die prinzipielle Realisierbarkeit darzulegen. Die
  Bachelorarbeit ist als praxisnahes Projekt durchzuführen.
\item
  Der Aufwand für die Fertigstellung einer Bachelorarbeit beträgt
  insgesamt mindestens 360 Stunden.
\item
  Die Bachelorarbeit hat die Form eines technischen Berichtes.
  \footnote{\autocite{bachelorreglement}}
\end{itemize}

\newpage

\chapter{Projektmanagement}\label{projektmanagement}

In diesem Kapitel wird die Planung der Bachelorarbeit ausgeführt. Weiter
wird die verwendete Infrastruktur erläutert.

\section{Grobe Projektplanung}\label{grobe-projektplanung}

Der grobe Projektplan illustriert die Strukturierung der Bachelorarbeit
über die knapp sechs Monate lange Projektzeit. Der Projektplan liefert
einen generellen Überblick über den zeitlichen Ablauf der Bachelorarbeit
und legt die Milestones fest. Als Soll-Aufwand der Bachelorarbeit wurden
376 Stunden veranschlagt. Der effektive Aufwand betrug xx Stunden.

\includegraphics{images/projektplan.jpg}

\newpage

\section{Aufwand}\label{aufwand}

Im Unterkapitel
\protect\hyperlink{rahmenbedingungen-bachelorarbeit}{Rahmenbedingungen
Bachelorarbeit} wurde bereits aufgeführt, dass eine Bachelorarbeit laut
Regelement mindestens 360 Stunden betragen soll. Diese Rahmenbedingung
wurde bei der Aufgabenstellung und Aufwandschätzung der Bachelorarbeit
berücksichtigt.

\begin{longtable}[c]{@{}lcl@{}}
\caption{Soll/Ist Analyse}\tabularnewline
\toprule
\begin{minipage}[b]{0.32\columnwidth}\raggedright\strut
\textbf{Arbeitsschritt}
\strut\end{minipage} &
\begin{minipage}[b]{0.24\columnwidth}\centering\strut
\textbf{Soll}
\strut\end{minipage} &
\begin{minipage}[b]{0.35\columnwidth}\raggedright\strut
\textbf{Ist}
\strut\end{minipage}\tabularnewline
\midrule
\endfirsthead
\toprule
\begin{minipage}[b]{0.32\columnwidth}\raggedright\strut
\textbf{Arbeitsschritt}
\strut\end{minipage} &
\begin{minipage}[b]{0.24\columnwidth}\centering\strut
\textbf{Soll}
\strut\end{minipage} &
\begin{minipage}[b]{0.35\columnwidth}\raggedright\strut
\textbf{Ist}
\strut\end{minipage}\tabularnewline
\midrule
\endhead
\begin{minipage}[t]{0.32\columnwidth}\raggedright\strut
Initialisierung
\strut\end{minipage} &
\begin{minipage}[t]{0.24\columnwidth}\centering\strut
10
\strut\end{minipage} &
\begin{minipage}[t]{0.35\columnwidth}\raggedright\strut
\strut\end{minipage}\tabularnewline
\begin{minipage}[t]{0.32\columnwidth}\raggedright\strut
Recherche
\strut\end{minipage} &
\begin{minipage}[t]{0.24\columnwidth}\centering\strut
45
\strut\end{minipage} &
\begin{minipage}[t]{0.35\columnwidth}\raggedright\strut
\strut\end{minipage}\tabularnewline
\begin{minipage}[t]{0.32\columnwidth}\raggedright\strut
Analyse
\strut\end{minipage} &
\begin{minipage}[t]{0.24\columnwidth}\centering\strut
20
\strut\end{minipage} &
\begin{minipage}[t]{0.35\columnwidth}\raggedright\strut
\strut\end{minipage}\tabularnewline
\begin{minipage}[t]{0.32\columnwidth}\raggedright\strut
Konzeption
\strut\end{minipage} &
\begin{minipage}[t]{0.24\columnwidth}\centering\strut
80
\strut\end{minipage} &
\begin{minipage}[t]{0.35\columnwidth}\raggedright\strut
\strut\end{minipage}\tabularnewline
\begin{minipage}[t]{0.32\columnwidth}\raggedright\strut
Prototyp
\strut\end{minipage} &
\begin{minipage}[t]{0.24\columnwidth}\centering\strut
60
\strut\end{minipage} &
\begin{minipage}[t]{0.35\columnwidth}\raggedright\strut
\strut\end{minipage}\tabularnewline
\begin{minipage}[t]{0.32\columnwidth}\raggedright\strut
Dokumentation
\strut\end{minipage} &
\begin{minipage}[t]{0.24\columnwidth}\centering\strut
140
\strut\end{minipage} &
\begin{minipage}[t]{0.35\columnwidth}\raggedright\strut
\strut\end{minipage}\tabularnewline
\begin{minipage}[t]{0.32\columnwidth}\raggedright\strut
Abgabe
\strut\end{minipage} &
\begin{minipage}[t]{0.24\columnwidth}\centering\strut
20
\strut\end{minipage} &
\begin{minipage}[t]{0.35\columnwidth}\raggedright\strut
\strut\end{minipage}\tabularnewline
\begin{minipage}[t]{0.32\columnwidth}\raggedright\strut
\textbf{Total}
\strut\end{minipage} &
\begin{minipage}[t]{0.24\columnwidth}\centering\strut
\textbf{375}
\strut\end{minipage} &
\begin{minipage}[t]{0.35\columnwidth}\raggedright\strut
\textbf{xx}
\strut\end{minipage}\tabularnewline
\bottomrule
\end{longtable}

\newpage

\section{Meilensteine}\label{meilensteine}

Meilensteine sind zum einen sehr wichtig für das Projektmanagement, da
sie den gesamten Ablauf der Bachelorarbeit in mehrere kleine und
überschaubarere Etappen und Zwischenziele einteilen. Dadurch kann auf
dem Weg zur erfolgreichen Umsetzung der Bachelorarbeit immer wieder Inne
gehalten und kontrolliert werden, wie der Stand der Dinge ist und ob die
Richtung geändert werden muss. So bleibt stets der Überblick gewahrt und
das Projekt ``Bachelorarbeit'' gerät nicht ausser Kontrolle. \footnote{\autocite{meilensteine}}

\begin{longtable}[c]{@{}ll@{}}
\caption{Meilensteine}\tabularnewline
\toprule
\begin{minipage}[b]{0.37\columnwidth}\raggedright\strut
\textbf{Ende Meilstein}
\strut\end{minipage} &
\begin{minipage}[b]{0.26\columnwidth}\raggedright\strut
\textbf{Meilenstein}
\strut\end{minipage}\tabularnewline
\midrule
\endfirsthead
\toprule
\begin{minipage}[b]{0.37\columnwidth}\raggedright\strut
\textbf{Ende Meilstein}
\strut\end{minipage} &
\begin{minipage}[b]{0.26\columnwidth}\raggedright\strut
\textbf{Meilenstein}
\strut\end{minipage}\tabularnewline
\midrule
\endhead
\begin{minipage}[t]{0.37\columnwidth}\raggedright\strut
bis 10. Januar 2016
\strut\end{minipage} &
\begin{minipage}[t]{0.26\columnwidth}\raggedright\strut
Recherche beendet
\strut\end{minipage}\tabularnewline
\begin{minipage}[t]{0.37\columnwidth}\raggedright\strut
bis 28. Februar 2016
\strut\end{minipage} &
\begin{minipage}[t]{0.26\columnwidth}\raggedright\strut
Anforderungsanalyse beendet
\strut\end{minipage}\tabularnewline
\begin{minipage}[t]{0.37\columnwidth}\raggedright\strut
bis 20. März 2016
\strut\end{minipage} &
\begin{minipage}[t]{0.26\columnwidth}\raggedright\strut
Design Review
\strut\end{minipage}\tabularnewline
\begin{minipage}[t]{0.37\columnwidth}\raggedright\strut
bis 24. April 2016
\strut\end{minipage} &
\begin{minipage}[t]{0.26\columnwidth}\raggedright\strut
Applikation steht zur Verfügung
\strut\end{minipage}\tabularnewline
\begin{minipage}[t]{0.37\columnwidth}\raggedright\strut
bis 8. Mai 2016
\strut\end{minipage} &
\begin{minipage}[t]{0.26\columnwidth}\raggedright\strut
schriftliche Arbeit abgeschlossen
\strut\end{minipage}\tabularnewline
\begin{minipage}[t]{0.37\columnwidth}\raggedright\strut
bis 22. Mai 2016
\strut\end{minipage} &
\begin{minipage}[t]{0.26\columnwidth}\raggedright\strut
Abgabe schriftliche Arbeit
\strut\end{minipage}\tabularnewline
\begin{minipage}[t]{0.37\columnwidth}\raggedright\strut
bis 29. Mai 2016
\strut\end{minipage} &
\begin{minipage}[t]{0.26\columnwidth}\raggedright\strut
Präsentation
\strut\end{minipage}\tabularnewline
\bottomrule
\end{longtable}

\newpage

\section{Termine}\label{termine}

\begin{longtable}[c]{@{}ll@{}}
\caption{Termine der Bachelorarbeit}\tabularnewline
\toprule
\begin{minipage}[b]{0.16\columnwidth}\raggedright\strut
\textbf{Datum}
\strut\end{minipage} &
\begin{minipage}[b]{0.47\columnwidth}\raggedright\strut
\textbf{Termin}
\strut\end{minipage}\tabularnewline
\midrule
\endfirsthead
\toprule
\begin{minipage}[b]{0.16\columnwidth}\raggedright\strut
\textbf{Datum}
\strut\end{minipage} &
\begin{minipage}[b]{0.47\columnwidth}\raggedright\strut
\textbf{Termin}
\strut\end{minipage}\tabularnewline
\midrule
\endhead
\begin{minipage}[t]{0.16\columnwidth}\raggedright\strut
28.10.2015
\strut\end{minipage} &
\begin{minipage}[t]{0.47\columnwidth}\raggedright\strut
Besprechung Aufgabenstellung mit Betreuer
\strut\end{minipage}\tabularnewline
\begin{minipage}[t]{0.16\columnwidth}\raggedright\strut
18.11.2015
\strut\end{minipage} &
\begin{minipage}[t]{0.47\columnwidth}\raggedright\strut
Freigabe der Aufgabenstellung
\strut\end{minipage}\tabularnewline
\begin{minipage}[t]{0.16\columnwidth}\raggedright\strut
9.12.2015
\strut\end{minipage} &
\begin{minipage}[t]{0.47\columnwidth}\raggedright\strut
Kichkoff
\strut\end{minipage}\tabularnewline
\begin{minipage}[t]{0.16\columnwidth}\raggedright\strut
6.01.2016
\strut\end{minipage} &
\begin{minipage}[t]{0.47\columnwidth}\raggedright\strut
Statusmeeting mit Betreuer
\strut\end{minipage}\tabularnewline
\begin{minipage}[t]{0.16\columnwidth}\raggedright\strut
\strut\end{minipage} &
\begin{minipage}[t]{0.47\columnwidth}\raggedright\strut
Statusmeeting mit Betreuer
\strut\end{minipage}\tabularnewline
\begin{minipage}[t]{0.16\columnwidth}\raggedright\strut
\strut\end{minipage} &
\begin{minipage}[t]{0.47\columnwidth}\raggedright\strut
Statusmeeting mit Betreuer
\strut\end{minipage}\tabularnewline
\begin{minipage}[t]{0.16\columnwidth}\raggedright\strut
\strut\end{minipage} &
\begin{minipage}[t]{0.47\columnwidth}\raggedright\strut
Statusmeeting mit Betreuer
\strut\end{minipage}\tabularnewline
\begin{minipage}[t]{0.16\columnwidth}\raggedright\strut
\strut\end{minipage} &
\begin{minipage}[t]{0.47\columnwidth}\raggedright\strut
Designreview
\strut\end{minipage}\tabularnewline
\begin{minipage}[t]{0.16\columnwidth}\raggedright\strut
\strut\end{minipage} &
\begin{minipage}[t]{0.47\columnwidth}\raggedright\strut
Statusmeeting mit Betreuer
\strut\end{minipage}\tabularnewline
\begin{minipage}[t]{0.16\columnwidth}\raggedright\strut
\strut\end{minipage} &
\begin{minipage}[t]{0.47\columnwidth}\raggedright\strut
Abgabe schriftliche Arbeit
\strut\end{minipage}\tabularnewline
\begin{minipage}[t]{0.16\columnwidth}\raggedright\strut
\strut\end{minipage} &
\begin{minipage}[t]{0.47\columnwidth}\raggedright\strut
Präsentation
\strut\end{minipage}\tabularnewline
\bottomrule
\end{longtable}

\newpage

\section{Infrastruktur}\label{infrastruktur}

Im Unterkapitel ``Infrastruktur'' sollen die verwendeten Tools erläutert
werden.

\subsection{Quellcode-Verwaltung mit
GitHub}\label{quellcode-verwaltung-mit-github}

Um einerseits eine Datensicherung zu gewährleisten und anderseits die
Änderungen nachvollziehbar abzulegen, wird die Bachelorarbeit mittels
Git und GitHub versioniert. Das Repository \footnote{https://github.com/coffeefan/bachelorarbeit}
ist für den Betreuer, Experten und Auftraggeber jederzeit einsehbar.

\subsection{Zeitmanagement mit toggl}\label{zeitmanagement-mit-toggl}

Beim Arbeiten an der Bachelorarbeit kann man sich schnell in Details
verlieren. Das Zeitmanagement-Tool toggl\footnote{https://toggl.com}
gibt schnell ein Feedback zur aktuell gebrauchten Zeit und einen
Überblick um die geplante mit der real verwendeten Zeit zu
vergleichen.Die Software ist besonders unter Kreativagenturen und
Freelancern beliebt. Sie präsentiert sich als eine besonders simple
Lösung, die die flexible Zeiterfassung in den Fokus stellt. Der User
kann neue Aufgaben mit nur einem Klick anlegen und die Stoppuhr starten,
um Arbeitszeiten automatisch zu erfassen.

\subsection{Dokumentieren mit Pandoc und
LaTex}\label{dokumentieren-mit-pandoc-und-latex}

Die Thesis dieser Bachelorarbeit soll basierend auf anerkannten
wissenschaftlichen Formaten erzeugt werden. Im Intranet der ZHAW wird
die Erstellung von wissenschaftlichen Arbeiten mit LaTex empfohlen.
LaTex Templates der einzelnen Abteilungen können erworben werden. Die
Effizienz bei der Erstellung von LaTex arbeiten ist umstritten. Diese
Arbeit wird zuerst im Markdown Syntax geschrieben und mittels Pandoc in
LaTex umgewandelt. Basierend auf den Templates und Einstellungen in
reinem LaTex wird dann das endgültige PDF-Dokument generiert.

\subsection{Design Mockup Balsamiq}\label{design-mockup-balsamiq}

Der Auftraggeber wünscht, dass eine strukturelle Vorlage des Designs vor
der Umsetzung illustriert wird. Dafür stellt der Auftraggeber eine
Lizenz des Tools Balsamiq zur Verfügung. Balsamiq ist ein wireframing
Tool. Dank den vielen konfigurierbaren Elementen kann rasch ein
Design-Mockup von Webseiten erstellt werden.

\newpage

\subsection{yUML}\label{yuml}

Um Ablaufe-Diagramm, Use Case-Diagramme und andere Uml-Diagramme zu
visualisieren, bedarf es ein Tool, dass die Diagramme sowohl optisch
ansprechend wie aber auch einfach und schnell anpassbar umsetzt. yUML
ist ein gratis Online-Service, über welchen mittels Code ein
UML-Diagramm kreiert werden kann. Diese Art von UML designen ist daher
sehr strukturiert und nachvollziehbar. Der Code, welcher zum Diagramm
führt, kann so einfach als Textdatei abgespeichert werden und wird in
dieser Bachelorarbeit im Github-Repository hinterlegt.

\begin{figure}[htbp]
\centering
\includegraphics{images/yuml.JPG}
\caption{Screenshot yUML Beispiel Klassendiagramm}
\end{figure}

\subsection{Draw.io}\label{draw.io}

Alle Diagramme, welche nicht via yUML designt werden können, werden mit
dem Online Tool Draw.io erstellt. Draw.io wird in Entwicklerkreisen als
webbasiertes Visio gehandelt. Seit dem letzten Release ist Draw.io ohne
Einschränkung gratis verwendbar. Die gezeichneten Diagramme können
direkt im Daten-Cloud Dienst Google Drive gespeichert werden.

\subsection{Infrastruktur Entwicklung}\label{infrastruktur-entwicklung}

Die Infrastruktur welche zur Entwicklung verwendet wird, kann je nach
Anforderung und Konzeption varieren. Die verwendete Infrastruktur wird
deshalb im Kapitel
\protect\hyperlink{entwicklungswerkezeuge}{Entwicklungswerkezeuge}
aufgelistet.

\newpage

\chapter{Recherche}\label{recherche}

\section{Fachbegriffe}\label{fachbegriffe}

Eine ausführliche Erklärung der Fachbegriffe befindet sich im Anhang
unter dem Kapitel ``\protect\hyperlink{glossar}{Glossar}''.

\section{Erläuterung der
Grundlagen}\label{erluxe4uterung-der-grundlagen}

In diesem Kapitel werden Funktionsweisen und Grundlage ausgeführt,
welche für die Bearbeitung dieser Bachelorthesis herangezogen wurden.

\hypertarget{authentifizierung}{\subsection{Authentifizierung}\label{authentifizierung}}

Duden: Authentifizierung - beglaubigen, die Echtheit von etwas bezeugen
\footnote{\autocite{duden}}

Eine Person oder Objekt eindeutig zu \textbf{authentifizieren} bedeutet
zu ermitteln, ob die- oder derjenige auch die Person ist, als welche sie
oder er sich ausgibt. \footnote{\autocite{authentifizierungsdef}} Dies
unterstreicht auch die Ableitung des Wortes vom Englischen Verb
\emph{authenticate}, was auf Deutsch ``sich als \emph{echt erweisen,
sich verbürgen, glaubwürdig sein}'' bedeutet. Das bekannteste Verfahren
der Authenfizierung ist die Eingabe von Benutzernamen und Passwort.
Weiter ist die PIN-Eingabe bei Bankautomaten oder Mobiltelefonen häufig
verbreitet. Die Möglichkeiten von verschiedenen Authentifizierungen ist
nahe zu grenzenlos. \footnote{\autocite{authentifizierungsdeforg}}

\subsection{Autorisierung}\label{autorisierung}

Autorisierung - Befugnis, Berechtigung, Erlaubnis, Genehmigung
\footnote{\autocite{duden}}

Wenn die \protect\hyperlink{authentifizierung}{Authentifizierung}
erfolgreich war, erteilt das System die Autorisierung. Dabei wird der
Person oder dem Objekt erlaubt, bestimmte Aktionen/Zugriffe
durchzuführen. Meist verfügen unterschiedliche Benutzer eines Systems
über verschiedene Zugriffsrechte. Die korrekte Zuweisung der
individuellen Rechte ist ebenfalls Bestandteil der Autorisierung.

Der Begriff Authentifizierung wird vielfach mit dem Begriff
Autorisierung verwechselt. Die Authentifizierung wird vom Benutzer
initiiert. Sie dient dem Nachweis, zur Ausübung bestimmter Rechte befugt
zu sein. Die anschliessende Autorisierung erfolgt automatisch durch das
System selbst. Im Zuge der Autorisierung werden dem Benutzer seine
Zugriffsrechte zugewiesen. \autocite{authentifizierungsdeforg}

\newpage

\newpage

\section{Grundlegende
Sicherheitsprinzipien}\label{grundlegende-sicherheitsprinzipien}

In diesem Unterkapitel werden die Grundlagen der Sicherheitsprinzipien
vermittelt, auf denen eine Authentifizierungssoftware aufgebaut werden
kann.

\subsection{KISS}\label{kiss}

\textbf{K}eep \textbf{I}t \textbf{S}tupid \emph{and} \textbf{S}imple

Ein verbreitetes Problem unter Softwareentwickern und Programmiern heute
ist, dass dazu tendiert wird, Probleme zu kompliziert und verschachtelt
zu lösen. Acht bis neun von zehn Entwickeln machen den Fehler, Probleme
zu wenig auseinanderzubrechen und alles in einem grossen Programm zu
lösen, anstatt es in kleinen Paketen verständlich zu
programmieren.{[}\^{}apachekiss{]}

Die folgenden Punkte listen die Vorteile für Softwareentwickler beim
Verwenden von Kiss auf:

\begin{itemize}
\tightlist
\item
  Mehr Probleme sollen schneller gelöst werden
\item
  Der Entwickler kann komplexe Probleme mit wenigen Zeilencodes lösen
\item
  Die Codequalität steigt
\item
  Der Entwickler kann grössere Systeme erstellen und unterhalten
\item
  Der Code wird flexibler werden, einfach wieder zu verwenden und zu
  modifizieren
\item
  Die Zusammenarbeit in grösseren Entwicklerteams und Projekten wird
  vereinfacht, da der Code bei allen ``stupid and simple'' ist
\end{itemize}

\subsubsection{KISS fördert die
Sicherheit}\label{kiss-fuxf6rdert-die-sicherheit}

Die Begründung, warum KISS die Sicherheit fördert, liefert Saltzer und
Schroeder: Ungewollte Zugriffspfade können nur durch zeilenweise
Codeinspektion entdeckt werden und dies wiederum setzt voraus, dass
Designs einfach und klein sind. Designs müssen so beschaffen sein, dass
sie abgeschlossene Bereiche enthalten, über die konkrete und sichere
Aussagen über Zugriffsmöglichkeiten und Effekte getroffen werden können.
{[}\^{}sicheresysteme\_93{]}

\subsection{Default-is-deny}\label{default-is-deny}

Ob eine Person oder ein Programm Zugriff auf Daten und Funktionen hat,
sollte nicht durch Verbote, sondern durch eine explizite Erlaubnis
geregelt werden. Dies bedeutet, dass solange keine explizite Erlaubnis
gesetzt ist, kann das Programm oder die Person nicht auf die Daten oder
Funktionen zugreifen. You \emph{deny} it. So simpel und logisch diese
Idee klingt, umso verwunderlicher ist es, dass viele Organisationen und
Entwicklungsfirma nicht dieses Vorgehen verwenden. Zum Beispiel
Filesysteme setzen auf Verbote anstatt auf explizite
Erlaubnisse.{[}\^{}sicheresysteme\_94{]} {[}\^{}defaultdeny{]}

\subsection{Open Design}\label{open-design}

Abgeleitet von der Theorie der Kryptografie gilt Folgendes: Nicht das
Design der Software sollte die Sicherheit sein, sondern der verwendete
Schlüssel. Dieses Konzept gilt es in der Softwareentwicklung und
Systemtechnik nur bedingt einzuhalten. Die Software soll eher nach dem
Ansatz entworfen werden: Mindestens intern soll das Software-Design
durch einen Design-Review Prozess analysiert werden. In manchen Fällen
macht es jedoch Sinn, das Softwaredesign geheimzuhalten, um einem
Angreifer nicht zu viele Informationen zur Verfügung zu stellen.
{[}\^{}sicheresysteme\_95{]}

\subsection{Zusammenfassung der
Sicherheitsprinzipien}\label{zusammenfassung-der-sicherheitsprinzipien}

Die wichtigsten Sicherheitsprinzipien lauten zusammengefasst wie folgt:

\begin{itemize}
\tightlist
\item
  Die Software muss aus kleinen, isolierten Einheiten aufgebaut werden,
  deren externe Beziehungen am Interface deutlich werden. Damit werden
  sowohl praktische Schadensreduzierung durch Isolation als auch eine
  schnelle und einfache Sicherheitsanalyse möglich.
\item
  Zugriffsentscheidungen dürfen nur auf der Basis expliziter, minimaler
  und keinesfalls durch immer und global verfügbare Permissions fallen.
\item
  Das Softwaredesign von Applikationen sollte wenn möglich öffentlich
  sein. Zumindest sollte ein interner Review-Prozess stattfinden, in
  dessen Verlauf eine Sicherheitsanalyse durch nicht an der Entwicklung
  Beteiligte erstellt wird.
\end{itemize}

{[}\^{}sicheresysteme\_93{]} : \autocite[ pp.93]{sicheresysteme}
{[}\^{}sicheresysteme\_94{]} :\autocite[ pp.94]{sicheresysteme}
{[}\^{}sicheresysteme\_95{]} :\autocite[ pp.95]{sicheresysteme}
{[}\^{}apachekiss{]}: \autocite{apachekiss} {[}\^{}defaultdeny{]}:
\autocite{defaultdeny}

\section{Ähnliche Produkte auf dem
Markt}\label{uxe4hnliche-produkte-auf-dem-markt}

Dieses Unterkapitel erläutert existierenden Produkte auf dem Markt.

\hypertarget{oauth-provider}{\subsection{OAuth-Provider}\label{oauth-provider}}

\subsubsection{OAuth}\label{oauth}

OAuth ist ein Protokoll. Es erlaubt sichere API-Autorisierungen.

\subsubsection{Das Bedürfnis nach
OAuth}\label{das-beduxfcrfnis-nach-oauth}

2006 implementierte Blaine Cook OpenID für Twitter. Ma.gnolia erhielt
dabei ein Dashboard, welches sich durch OpenID autorisieren liess.
Deshalb suchten die Entwickler von Ma.gnolia und Blaine Cook eine
Möglichkeit, OpenID auch für die Verwendung von APIs zu gebrauchen. Sie
diskutierten verschiedene Implementierungen und stellten fest, dass es
keinen offenen Standart für API-Access Delegation gab. So fingen sie an,
einen Standard zu entwickeln. 2007 entstand daraus eine Google Group. Am
3. October 2007 war dann der OAuth Core 1.0 bereits veröffentlicht
worden.

\subsubsection{Funktionalität von
OAuth}\label{funktionalituxe4t-von-oauth}

Ein Programm/API (Consumer) stellt über das OAuth-Protokoll einem
Endbenutzer(User) Zugriff (Autorisierung) auf seine
Daten/Funktionalitäten zur Verfügung. Dieser Zugriff wird von einem
anderen Programm (Service) gemanagt. Das Konzept ist nicht generell neu.
OAuth ist ähnlich zu Google AuthSub, aol OpenAuth, Yahoo BBAuth,
Upcoming api, Flickr api, Amazon Web Services api. OAuth studierte die
existierenden Protokolle und standardisierte und kombinierte die
existierenden industriellen Protokolle. Der wichtigste Unterschied zu
den existierenden Protokollen ist, dass OAuth sowohl offen ist als es
auch geschafft hat, genügend Einsatzgebiete zu finden, um als Standard
zu gelten. Jeden Tag entstehen neue Webseiten, welche neue
Funktionalitäten und Services offerieren und dabei Funktionalitäten von
anderen Webseiten brauchen. OAuth stellt dem Programmierer einerseits
eine standardisierte Implementierung zur Verfügung. Anderseites erhält
der Endbenutzer dank dieses Protokolls die Möglichkeit, Teile seiner
Funktionalität oder Daten bei einem anderen Anbieter zur Verfügung zu
stellen. Bei Facebook OAuth kann der Endbenutzer zum Beispiel seine
Posts zur Verfügung stellen, nicht aber seine Freunde bekannt geben.

Dank der weiten Verbreitung gibt es nun in allen bekannten
Programmiersprachen eine Implementierung, sowohl von Client wie auch vom
Server. Weitere Infos dazu unter oauth.net\href{http://oauth.net/2/}{1}

Die grössten \protect\hyperlink{oauth-1}{OAuth}-Provider wie Google,
Facebook und Twitter erziehlen eine weite Verbreitung weltweit:

\begin{figure}[htbp]
\centering
\includegraphics{images/excel-statistik/socialmedia-aktivenutzer.jpg}
\caption[Aktive Nutzer Weltweit ]{Aktive Nutzer Weltweit
\footnotemark{}}
\end{figure}
\footnotetext{Die Statistik wurde basierend auf den Daten von
  SocialMedia-Institute \autocite{smi}erstellt. Facebook- und
  Twitter-Daten sind am 5. November 2015 und die Google-Daten im 2014
  erhoben worden.}

Ganze \emph{78\%} \autocite{goldbachsocial} der Schweizer Bevölkerung
nutzten SocialMedia und besitzen dadurch einen OAuth-Account:

\begin{figure}[htbp]
\centering
\includegraphics{images/excel-statistik/socialmedia-schweiz.jpg}
\caption[Anzahl Schweizer Nutzer ]{Anzahl Schweizer Nutzer
\footnotemark{}}
\end{figure}
\footnotetext{Die Statistik wurde basierend auf den Daten von Goldbach
  Interactive \autocite{goldbachsocial} generiert. Die Daten sind im
  März 2015 erhoben.}

\newpage

\subsubsection{Vorteile}\label{vorteile}

Mindestes 78\% der Schweizer Bevölkerung besitzt bereits einen OAuth
Account. Das Protokoll ist ein etablierter Standard.

\subsubsection{Nachteile}\label{nachteile}

Mehrfachregistrierungen sind möglich. Jenach OAuth-Provider werden
verschiedene Daten zur Verfügung gestellt. Pro OAuth Provider kann man
sich registrieren. Ein Abgleich der verschiedenen OAuth Provider wird
vom \protect\hyperlink{oauth-1}{OAuth}-Protokoll nicht zur Verfügung
gestellt. Ein Teil der Bevölkerung müsste sich vor Nutzung noch
registrieren. Die Implementierung ist trotz vielen Libraries nicht ohne
tiefere Programmierkenntnisse möglich.

\newpage

\subsection{playbuzz.com}\label{playbuzz.com}

Youtube von Google ist im Jahr 2015 mit Abstand die meist verbreiteste
Videopublishing-Plattform\footnote{\autocite{statista}}. Medienhäuser
nutzen Youtube, um ihren Artikel einfach mit einem Video zu ergänzen.
Neben der einfachen Integration profitieren die Medienhäuser von der
zusätzlichen Verbreitung über youtube.com und der einfachen viralen
Verbreitungsmöglichkeiten von youtube. PlayBuzz möchte das Youtube für
Votings, Quiz und ähnlicher Embeded Content werden. Neben MTV, Focus,
Time oder Bild verwendet seit Herbst 2015 auch ein grosses Medienhaus
der Schweiz die Plattform. Tamedia erfasst neuerdings immer wieder auf
20minuten Votings und Umfragen mit PlayBuzz.

2012 wurde Playbuzz von Shaul Olmert (Sohn des Premie Minster von Israel
Ehuad Olmert) und Tom Pachys ins Leben gerufen. Der offizielle Launch
war im Dezember 2013. Im Juni 2014 wurde Playbuzz bereits das 1. Mal
unter den Top 10 Facebook Shared Publishers aufgelistet. Im Juni 2014
konnte Playbuzz bereits 70 Millionen unique views aufweisen. Im
September 2014 kamen sieben von den zehn Top Shares auf Facebook laut
forbes.com von Playbuzz. Playbuzz setzt nach eigenen Angaben auf Content
wie Votes und Quizes, welche gerne viral geteilt werden, und ermöglicht
Endnutzer und Redaketeueren einfache Verwendung. \footnote{\autocite{t3nplaybuzz}}
\footnote{\autocite{playbuzz}}

\subsubsection{Vorteile}\label{vorteile-1}

Playbuzz ist kostenlos und lässt sich einfach integrieren. Durch
Verwendung von Playbuzz kann die Verbreitung des eigenen Inhalts
gesteigert werden. Die Verwaltungsoberfläche und die Reports sind
übersichtlich und einfach zu bedienen.

\subsubsection{Nachteile}\label{nachteile-1}

Der Verweis auf Playbuzz ist immer klar ersichtlich. Auch beim Posten
auf den SocialMedia-Kanälen ist die Herkunft von Playbuzz
offensichtlich. Die Möglichkeiten in Funktionalität und Design haben
hingegen Grenzen. Individuelle Erweiterungen sind nicht einfach möglich.
Bestehende Interaktivitäten oder Interaktivitäten, welche nicht von
PlayBuzz erstellt werden, können nicht verwendet werden.
Mehrfachteilnahmen waren möglich.

\newpage

\subsection{WebSMS.com
Zwei-Faktor-Authentifizierung}\label{websms.com-zwei-faktor-authentifizierung}

WebSMS.com bittet eine Zwei-Faktor-Authentifizierung über SMS an. Der
User gibt seine Mobilnummer in der Webmaske der Schnittstelle ein und
erhält einen Code, welcher der User danach in der Webschnittstelle
eingibt. Dadurch kann sichergestellt werden, dass der User zur
eingegebenen Mobilenummer passt. Der Service kostet monatlich 20 CHF und
weitere 0.08 CHF pro SMS. \footnote{Die Kosten sind am 28. Dezember 2015
  unter folgendem Link abgerufen worden:
  https://websms.ch/preise\#at-preisuebersicht}

Die Stärke und Sicherheit dieses Services ist direkt mit dem Umgang von
Mobilenummern/SIM-Karten und dessen Authentifizierung verbunden.

Seit 1. Juli 2004 müssen auch bei Prepaid-Karten in der Schweiz
Personalien hinterlegt werden.\footnote{Meldung des UVEKS über
  Gesetzesänderung: \autocite{uvek}} Dadurch ist eine eindeutige
Authentifizierung über Mobilennummern gewährleistet. Die
Mobilefunkanbieter schränken die Anzahl SIM-Karten auf maximal fünf pro
Person ein. Dieses Maximum konnte aber auf den Webseiten der Anbieter
nicht direkt gefunden werden. Daher galt es den Wert zu untersuchen und
mögliche Abweichungen ausfindig zu machen.

\subsubsection{Swisscom}\label{swisscom}

Die Swisscom hat kein öffentlich zugängigliches Dokument, welches die
maximale Anzahl SIM-Karten pro Person beschreibt. Mündlich durch das
Verkaufspersonal des Swisscom-Shops Zürich HB im Dezember 2015 und im
Chatprotokoll \footnote{Chat-Protokoll Swisscom 12.Februar 2016
  http://bit.ly/swisscom-chat} wurde der Wert bestätigt. Es wurde darauf
hingewiesen, dass kein Dokument mit dieser Zahl vorhanden ist.

\paragraph{Selbstversuch}\label{selbstversuch}

Es wurde versucht, bei zwei unabhängigen Handyanbietern mehr als fünf
Swisscom-Perpaid-Abos abzuschliessen. Dabei wurde von Thomas Bachmann
über vier Wochen verteilt bei dem Anbieter Interdiscount im Manor
Winterthur bei verschiedenem Kaufspersonal ein Prepaidhandy eingekauft.
Beim Einkauf des sechsten Handys wurde der Verkauf von der Kasse
abgelehnt. Die Fehlermeldung der Kasse beinhaltete den Hinweis, dass die
Nummer nicht erneut auf den Kunden registriert werden könne, da er schon
fünf SIM Karten bei der Swisscom besitze. Christian Bachmann kaufte über
zwei Wochen verteilt bei dem Anbieter Migros Electronics in der Migros
Limmat, Interdiscount im Manor Winterthur, Interdiscount im Zürich HB
bei verschiedenem Kaufspersonal ein Swisscom Prepaidhandy. Beim Einkauf
des sechsten Handys wurde der Verkauf von der Kasse abgelehnt. Die
Nummer liess sich nicht erneut auf den Kunden registrieren, da er schon
fünf SIM Karten bei der Swisscom besessem hat .

\subsubsection{Sunrise}\label{sunrise}

Die Sunrise hat nach Rücksprache ein PDF mit Ihren Bestell- und
Lieferbedingunge zugesendet.\footnote{Kopie Bestell- und
  Lieferbedingungen http://bit.ly/sunrise-bedienungen} Die maximale
Anzahl SIM-Karten ist in diesen Bestell- und Lieferbedingungen
festgelegt. Auch die Sunrise hat das Maximum auf fünf pro Person
festgelegt.

\subsubsection{SALT}\label{salt}

Bei der Firma SALT konnte mir ebenfalls kein Dokument mit der Kennzahl
gegeben werden. SALT vergibt gemäss ihrer schriftlichen Auskunft
\footnote{E-Mail von Salt 13.Februar 2016 http://bit.ly/salt-email} pro
Person maximum drei SIM Karten.

\subsubsection{Vorteile}\label{vorteile-2}

Die mehrfache Registrierung ist auf maximal fünf beschränkt. Durch die
Kosten für eine SIM-Karte/Mobilenummer wird der Wert zusätzlich
gemindert. Bei Missbrauch kann der User eindeutig identifiziert werden.
Eine Automatisierung ist nahezu unmöglich.

\subsubsection{Nachteile}\label{nachteile-2}

Der Versand von SMS verursacht Kosten. Die Implementation bedarf hohes
technisches Know-How.

\hypertarget{suisseid}{\subsection{SuisseID}\label{suisseid}}

Die SuisseID schafft die rechtlichen und technischen Voraussetzungen für
den elektronischen Geschäftsverkehr. Als digitaler Identitätsausweis im
Internet bietet sie ihren Anwenderinnen und Anwendern eine sichere
Authentifikation zu Web-Applikationen, eindeutige Identifikation für
Internet-Dienste und digitales, rechtsgültiges Signieren von Dokumenten.
Der Erwerb einer solchen SuisseID kostet den Endkunden eine
beträchtliche Summe Geld. Der Anbieter der Authentifizierung erwarten
keine grossen Kosten. Dadurch ist eine kleine Verbreitung für privaten
Nutzen offensichtlich. Entwickler von Integrationen erhalten eine ganzes
SDK und Kontaktmöglichkeiten.

\subsubsection{Vorteile}\label{vorteile-3}

Hohe Sicherheit durch sichere und eindeutige Authentifikation ist
gewährleistet. Rechtliche Vorraussetzungen sind gegeben. Enwickler von
Integrationen werden unterstützt.

\subsubsection{Nachteile}\label{nachteile-3}

Kleine Verbreitung und hohe Kosten für den Enduser sind die Nachteile
von SuisseID.

\section{Fazit}\label{fazit}

Auf dem Markt sind verschiedene Anbieter, welche Interaktivitäten
schützen können oder gar ganze Packages anbieten. Ein Service, welcher
es erlaubt individuell konfigurierbare Sicherheitstufen festzulegen und
diese in eine bestehende Interaktivität einzubauen wurde nicht gefunden.
Einige Anbieter könnten als einzelne Sicherheitsstufe in der Umsetzung
berücksichtigt werden. \footnote{Stand 4. Januar 2016}

\section{Authetentifizierungskomponenten}\label{authetentifizierungskomponenten}

Die Authentifizierung kann mit verschiedenen Komponenten durchgeführt
werden. Folgend gilt es die Komponenten zu erklären.

\subsection{Cookie}\label{cookie}

Ein Cookie ist ein kurzes Text-Snippet, welches beim Besuch einer
Webseite an den Browser gesendet wird. Dabei kann das Cookie
serverseitig vom Webserver an den Browser gesendet werden oder in einem
Skript wie Javascript erstellt werden. Der Browser sendet das Cookie bei
jeder Aufforderung wieder der Webseite zu. Der Erfinder der
Cookie-Technologie ist Vita Lou Montulli, der 1994 nach seinem
Studienabbruch bei Netscape einstieg und zudem den Navigator
mitentwickelte. Der Betreiber der Interaktivität speichert also im
Cookie die Teilnahme. Beim erneuten Aufruf erhält er das Cookie und
weiss so, dass der Teilnehmer schon einmal teilgenommen hat oder nicht.
Das Absichern von Interaktivitäten durch Cookies ist weit verbreitet.
Durch die browserseitige/clientseitige Speicherung kann diese
Speicherung auch clientseitig manipuliert werden. \footnote{\autocite{cookie-centra}}\footnote{\autocite{google-cookies}}

\subsubsection{Automatisierungsmöglichkeit und
Mehrfachteilnahme}\label{automatisierungsmuxf6glichkeit-und-mehrfachteilnahme}

Die Automatisierung ist ohne IT-Knowhow möglich. Es stehen einige
Browser Plugins zur Verfügung, welche es ermöglichen, sein Surfverhalten
über einfache Record-Funktionen aufzunehmen und danach Cookies zu
löschen. So kann mehrfach an einer Interkativität wie Umfragen
teilgenommen werden.

\subsubsection{Kosten}\label{kosten}

Cookies verursachen keine direkten Kosten.

\subsection{Flash-Cookies}\label{flash-cookies}

Ein Flash-Cookie ist, wie es der Name bereits vermuten lässt, ein
Cookie, das an den Adobe-Flash Player gebunden ist. Da der Flash-Player
im Betriebsystem installiert wird, funktionieren die Flash-Cookies
browserunabhängig. Die Bedienungen dieser Flash-Cookies werden von Adobe
festgelegt und der Browser kann nicht direkt in das Handling eingreifen.
Auch hier wird die Speicherung clientseitige durchgeführt und kann diese
Speicherung auch clientseitig manipuliert werden. Seit Steven Jobs mit
Apple keinen Support für die mobilen Geräte in Aussicht stellte und auf
die Probleme und Risiken hinwies, verliert die Plattform gestärkt durch
immer wieder auftretende Sicherheitsprobleme an Usern. So haben am 1.
Januar noch knapp 10\%{[}\^{}flashussage{]} aller Webseitenbesucher den
Flash-Player.

{[}\^{}flashussage{]}\autocite{flashussage}

\subsection{Mehrfachteilnahme}\label{mehrfachteilnahme}

Flash-Cookies können je nach Betriebsystem mit verschiedenem Aufwand
gelöscht werden und dadurch ist eine Mehrfach-Teilnahme möglich.

\subsubsection{Automatisierungsmöglichkeit}\label{automatisierungsmuxf6glichkeit}

Die automatisierte Teilnahme und Löschung ist im Gegensatz zu
klassischen Cookies aufwendiger, aber durchaus machbar.

\subsubsection{Kosten}\label{kosten-1}

Cookies verursachen keine direkten Kosten.

\subsection{IP-Adresse}\label{ip-adresse}

Bei der Nutzung einer Interkativität wird die IP-Adresse des Teilnehmers
gespeichert. So kann bei erneutem Teilnehmen die Teilnahme verweigert
werden. Das Internetprotokoll kurz IP sieht für jedes Gerät, welches an
einem IP-Netzwerk angeschlossen ist, eine eindeutige Adresse vor.
Deshalb auch der naheliegende Name IP-Adresse. Generell wird im Internet
über den ``IP Version 4 Standart'' kommmuniziert. Damit lassen sich aber
nur 4,22 Miliarden eindeutige Adressen im World Wide Web vergeben.
Deshalb mussten einige Methoden entwickelt werden um vorerst das Problem
umgehen zu können. Unter anderem identifiziert sich ein Router wie ein
Rechner und nutzt intern andere IP-Adressen. Gegen aussen haben also
alle Nutzer des Netzwerks die selbe IP-Adresse. Dadurch entsteht die
Problematik an dieser Methode, dass in einem Grossraumbüro mit einem
Internetanschluss auch nur eine Person an einem Wettbewerb teilnehmen
kann.\footnote{\autocite{pclexikon-ip}{]}}

\subsubsection{Mehrfachteilnahme}\label{mehrfachteilnahme-1}

Es gibt verschiedene Möglichkeiten, die IP-Adresse zu wechseln. Eine
einfache Möglichkeit ist durch Verwenden von Proxy-Servern eine andere
IP-Adresse zu benutzen. Die Mehrfachteilnahme ist also einfach möglich.

\subsubsection{Automatisierungsmöglichkeit}\label{automatisierungsmuxf6glichkeit-1}

Das automatisierte Wechseln eines Proxys ist etwas aufwendiger und
braucht technisches Know-How aber durchaus möglich.

\subsection{Kosten}\label{kosten-2}

Das authentifzieren via IP-Adresse verursacht keine direkten Kosten.

\subsection{Captcha}\label{captcha}

Captcha ist ein Test, mit dem festgestellt werden kann, ob sich ein
Mensch oder ein Computer eines Programms bedient \footnote{\autocite{duden}}.

Im Jahr 2000 wurde das Captcha an der Carnegie Mellon University
erfunden. Captcha steht für \textbf{C}ompletely \textbf{A}utomated
\textbf{P}ublic \textbf{T}uring test to tell \textbf{C}omputers and
\textbf{H}umans \textbf{A}part. Luis von Ahn, Professor der
Entwickler-Gruppe, erklärte die Dringlichkeit von Captcha damals so:
``Anybody can write a program to sign up for millions of accounts, and
the idea was to prevent that''. **** \footnote{\autocite{captcha}}

\subsubsection{Captcha Zahlen}\label{captcha-zahlen}

In 2014 wurden 200 Million Captchas pro Tag eingegeben. Dabei braucht
ein User durchschnittlich 10 Sekunden das entspricht 500'000
Stunden.\footnote{Die statistischen Daten wurden von Google 2014 in
  ihrem Blog publiziert \autocite{googlecaptcha}}

\begin{figure}[htbp]
\centering
\includegraphics{images/captcha.png}
\caption{Beispiele von Captchas \emph{Quelle:drupal.org}}
\end{figure}

\subsection{Zwei-Faktor-Authentifizierung}\label{zwei-faktor-authentifizierung}

Die Zwei-Faktor-Authentifizierung wird häufig 2FA genannt. Der User wird
mittels zweier unabhängiger Faktoren identifiziert. Der Begriff
``Faktor'' umschreibt dabei eine Komponente oder
Authentifizierungsmethode. {[}\^{}cnet-2fa{]}

Die Zwei-Faktor-Authentifizierung ist in der Schweiz durch das E-Banking
bekannt geworden. Der User gibt als erstes Faktor Username/Vertragnummer
und Passwort ein. In einem zweiten Schritt gibt er vom System
gewünschten Code aus der Codekarte oder des elektrischen Rechners als
zweiten Faktor ein. Im Alltag bei einem Einkauf im Detailhandel
authentifiziert sich der EC-Kartenchip als erster Faktor. Als zweiter
Faktor hat sich der Kunde ein Passwort auswendig gemerkt, welches er
eingibt.

. Die folgenden Authenfizierungen basieren auf den Prinzip der
Zwei-Faktor-Authentifizierung.

\subsection{E-Mail-Bestätigungscode}\label{e-mail-bestuxe4tigungscode}

Im Registrationsprozess ist das Erhalten eines E-Mails mit
Bestätigungscode quasi zum Standart geworden. Durch diese Methodik kann
man garantieren, dass die angegebene E-Mailadresse auch tatsächlich
existiert und der User darauf Zugriff hat. Der User soll also auch bei
der Authentifizierungsschnittstelle seine E-Mailadresse eintragen und
erhält dann umgehend den Bestätigungslink an diese zugesandt.

\subsubsection{Automatisierungsmöglichkeit}\label{automatisierungsmuxf6glichkeit-2}

Das automatische Auslesen von E-Mails ist möglich. Jedoch ist der
Aufwand dafür sehr hoch.

\subsubsection{Mehrfachteilnahme}\label{mehrfachteilnahme-2}

Ein User kann verschiedene E-Mail Adressen besitzen. Das Erstellen von
neuen E-Mail Adressen ist mit Aufwand verbunden, aber einfach möglich.

Anbieter wie 10-Minutes Mail \footnote{10-Minute Mail
  \autocite{10minutemail}} stellen auf Knopfdruck für einige Minuten
eine temporäre E-Mail Adresse zur Verfügung. Dadurch können schnell
einige E-Mailadressen erstellt werden. Diese Domains müssen über eine
aufwendige Blacklist gefiltert werden oder durch ein zeitversetztes
Bestätigungsmail ausgehebelt werden.

\subsubsection{Kosten}\label{kosten-3}

Das Versenden von E-Mails über einen SMTP-Server ist generell kostenlos.
Bei hohem Gebrauch dieser Komponente lohnt es sich, die E-Mails über
eine professionelle Infrastruktur für Massenversendungen zu versenden
und zu analysieren. Beispiele dafür sind Mailchimp \footnote{www.mailchimp.com}
oder Sendgrid \footnote{sendgrid.com}

\subsection{SMS- Bestätigungscode}\label{sms--bestuxe4tigungscode}

Das Konzept des in einem vorherigen Kapitel erwähnten Anbieters WebSMS
soll von der Authentifizierungsschnittstelle ebenfalls implementiert
werden. Der User gibt im ersten Schritt seine Mobilenummer ein. Er
erhält dann einen Code per SMS zugesandt. Im zweiten Schritt gibt der
User den erhaltenen Mobilecode im Webform ein und bestätigt so, dass ihm
die Mobilenummer gehört. Zum Versenden der SMS ist ein SMS-Gateway
nötig.

\subsubsection{Automatisierungsmöglichkeit}\label{automatisierungsmuxf6glichkeit-3}

Die Automatisierung kann als nicht möglich eingestuft werden.

\subsubsection{Mehrfachteilnahme}\label{mehrfachteilnahme-3}

Die mehrfache Teilnahme wurde bereits im Kapitel zum Anbieter WebSMS
eingehenden behandelt. Daraus resultierte, dass in der Schweiz maximal
fünf Mobilenummern pro Anbieter und Person bezogen werden können.

\subsubsection{Kosten}\label{kosten-4}

Je nach SMS-Gateway, Mobileanbieter des Empfängers und
Verwendungsintensität belaufen sich der Versand eines SMS zwischen 0.04
CHF und 0.15 CHF. \footnote{Die Preise wurden am 1. März 2016 auf
  aspsms.ch/instruction/prices.asp, tropo.com/pricing und
  twilio.com/sms/pricing abgefragt}

\subsection{Telefonanruf mit
Bestätigungscode}\label{telefonanruf-mit-bestuxe4tigungscode}

Nacheingabe der Telefonnummer oder Mobilenummer erhält der User einen
digitalen Anruf. Die Computerstimme liest dem User einen Code vor,
welcher er danach in der Webpage einggibt.

\subsubsection{Automatisierungsmöglichkeit}\label{automatisierungsmuxf6glichkeit-4}

Die Automatisierung kann als nicht möglich eingestuft werden.

\subsubsection{Mehrfach Teilnahme}\label{mehrfach-teilnahme}

Die Teilnahmeanzahl ist von den vorhandenen Telefonanschlüssen abhängig
und daher nur eingeschrenkt möglich.

\subsubsection{Kosten}\label{kosten-5}

Die Kosten berechnen sich bei den analysierten Anbietern basierend auf
einer geringen Monatspauschale zwischen CHF 1.00 und CHF 2.00 und Kosten
pro Minute je nach Telefonanbietern des Empfängers und Voicegateway
zwischen CHF 0.10 und CHF 0.65.\footnote{Die Preise wurden am 1. März
  2016 auf nexmo.com/products/voice/, tropo.com/pricing und
  twilio.com/voice/pricing abgefragt}.

\subsection{Postversand}\label{postversand}

Wie kann sichergestellt werden, dass eine Person auch tatsächlich am
angegebenen Ort wohnt? Im Telefonbuch digital oder analog waren früher
fast alle Personen erfasst. Immer weniger Personen haben heute einen
Fixanschluss und einige lassen Ihre Nummern nicht mehr eintragen. Nur
vereinzelte Personen tragen ihre mobile Telefonnumer und Adresse im
Telefonbuch ein. Google steht vor dem gleichen Problem mit ihrem Produkt
Google Maps. In Google Maps sollen schnell neue Firmendaten,
Veranstaltungslocations oder andere Adresseinträge erfasst werden
können. Doch sollen Betrüger oder Spassvögel daran gehindert werden,
Falscheinträge zu machen. Daher versendet Google zur Verifikation
einfach einen Code per Brief bzw.Postkarte an die Adresse.\footnote{\autocite{googlebusiness}}
Das simple Konzept kann auch für den Authentifizierungsschnittstelle
umsetzt werden um die Adresse eindeutig zu verifizieren. Einen Haken hat
dieses Konzept jedoch. Jemand muss den Brief ausdrucken, in ein Couvert
legen, frankieren und per Post versenden. Dieser ``Jemand'' kann als
Service z.b. beim schweizer Startup pingen.com eingekauft werden.

\subsubsection{Automatisierungsmöglichkeit}\label{automatisierungsmuxf6glichkeit-5}

Die Automatisierung kann als nicht möglich eingestuft werden.

\subsubsection{Mehrfachteilnahme}\label{mehrfachteilnahme-4}

Die Teilnahmeanzahl ist von den vorhandenen Adressanschriften abhängig
und daher ist eine Mehrfachteilnahme nur eingschrenkt möglich.

\subsubsection{Kosten}\label{kosten-6}

Die Kosten berechnen sich für den Versand in der Schweiz bei dem
analysierten Anbietern je nach Druck und Versandart des Empfängers
zwischen CHF 1.20 und CHF 1.65.\footnote{Die Preise wurdem am 10. März
  2016 auf pingen.com abgefragt}

\hypertarget{oauth-1}{\subsection{OAuth}\label{oauth-1}}

Die Zwei-Faktor-Authentifizierung OAuth wurde im Kapitel
\protect\hyperlink{oauth-provider}{OAuth-Provider} ausführlich
erläutert.

\subsubsection{Automatisierungsmöglichkeit}\label{automatisierungsmuxf6glichkeit-6}

Eine OAuth-Registrierung kann als nicht automatisierbar eingestuft
werden. Automatisierbares Anmelden und Verwenden von verschiedenen
Accounts ist durchaus möglich. Plattformen wie kingfluencers.ch zeigen
Möglichkeiten auf, wie automatisiert auf SocialMedia Platfformen von
Dritten zugegriffen werden kann und Tätigkeiten ausgeführt werden
können.

\subsubsection{Mehrfachteilnahme}\label{mehrfachteilnahme-5}

Eine Mehrfachregistration ist möglich.

\subsubsection{Kosten}\label{kosten-7}

OAuth bewirkt keine direkten Kosten.

\subsection{SuisseID Integration}\label{suisseid-integration}

SuisseID wurde bereits im Kapitel \protect\hyperlink{suisseid}{SuisseID}
erläutert.

\subsection{Automatisierungsmöglichkeit}\label{automatisierungsmuxf6glichkeit-7}

Eine Automatisierung ist nahezu unmöglich.

\subsubsection{Mehrfachteilnahme}\label{mehrfachteilnahme-6}

Eine Mehrfachteilnahme ist nahezu unmöglich.

\subsubsection{Kosten}\label{kosten-8}

Für den Betreiber fallen geringe Kosten an. Der Enduser zahlt aber einen
bemerkenswerten Preis.

\subsection{Browser Fingerprints}\label{browser-fingerprints}

Der Fingerabdruck ist aus der Kriminaltechnik nicht mehr wegzudenken.
Bereits vor 2000 Jahren haben Chinesen ihre Schuldscheine mit
Fingerabdrücken unterzeichnet. Es sollten über 19 Jahrhunderte gehen bis
der Fingerabdruck auch in der Kriminaltechnik eingesetzt wurde. Seit
über 100 Jahren, genauer seit 1913, ist der Fingerabdruck auch im Dienst
der Schweizer Eidgenossenschaft. Im Gegensatz zur DNA unterscheidet sich
der Fingerabdruck bei Zwillingen klar, auch wenn ähnliche Merkmale
erkennbar sind. Bereits nach nur vier Monaten Schwangerschaft sind die
Muster der Papillarleisten beim Embryo festgelegt. Der einzigartige
Fingerabdruck des Menschen ist somit fertiggestellt. Dieses Muster
ändert sich bis zur Auflösung des Körpers nach dem Tod nicht mehr.
\footnote{\autocite{derfingerabdruck}}

\begin{figure}[htbp]
\centering
\includegraphics{images/fingerabdruck.jpg}
\caption{Fingerabdruck Mit Kohlepulver werden Fingerabdrücke sichtbar
gemacht und auf Klebefolie gesichert \emph{Quelle:phi-hannover.de}}
\end{figure}

Der Fingerabdruck eignet sich zur Authentifizierung einer Person durch
folgende Merkmale: - Der Fingerabdruck ist eindeutig. - Der
Fingerabdruck ist über den Tod hinaus beständig. - Der Fingerabdruck ist
von aussen einfach ``abrufbar''. Er ist von blossem Augenn sichtbar und
wir hinterlassen das Muster der Papillarleisten auf Gegenständen wie
Glas.

\subsubsection{Fingerabdruck des
Browsers}\label{fingerabdruck-des-browsers}

Im Gegensatz zum Datenschutz wäre es aus Sicht der eindeutigen
Identifikation wünschenswert, wenn digitale Personen oder deren Geräte
auch einen Fingerabdruck von sich geben würde, der sowohl eindeutig,
beständig und abrufbar ist. Immer wieder versuchten unter dem Thema
``Browser Fingerprint'' Personen ein Verfahren zu entwickeln, das genau
dies ermöglicht. Microsoft führte laut eigenen Angaben \footnote{\autocite{xpactivation}}
mit Windows XP Produktaktivierung ein Verfahren ein, das aus
Prozesser-Typ, Grafikkarteninformationen und Festplatte einen
Fingerabdruck des Geräts erstellt. So konnte bei einer zweiten
Aktivierung mit dem selben Registrationsschlüssel Massnahmen getroffen
werden.

Auch der Browser übermittelt an den Server verschiedene Informationen:

{[}\^{}cnet-2fa{]} {[}\^{}cnet-2fa{]}: \autocite{cnet-2fa}

\subsubsection{Cookies}\label{cookies}

\chapter{Anforderungen}\label{anforderungen}

Dieses Kapitel beschreibt das Durchführen einer Anforderungsanalyse.
Anhand der Anforderungsanalyse sollen die Anforderungen für die zu
entwickelnden Softwares ermittelt werden. Die Anforderungen bilden die
Basis für die Architektur, das Softwaredesign, die Implementation und
die Testfälle. Ihnen ist dementsprechend ein sehr grosser Stellenwert
zuzuschreiben.

\section{Akteure}\label{akteure}

\textbf{Programmierer} Der Programmierer ist der Entwickler der
Webseite. Er möchte sein programmiertes oder sein verwendetes
Social-Media-Modul mit dem Authentifizierungsschnittstellen-Service
schützen.

\textbf{User} Der User ist der Endkunde. Er nimmt am Social-Media-Modul
teil und authentifiziert sich über den
Authentifizierungsschnittstellen-Service.

\newpage

\section{Use-Cases}\label{use-cases}

Im Nachfolgenden werden alle UseCases aufgelistet die im Rahmen dieser
Thesis gefunden wurden.

\subsection{Use-Cases Diagramm}\label{use-cases-diagramm}

Das Use-Case Diagramm illustriert die nachfolgenden Use-Cases. Dadurch
kann rasch ein Überblick über die zu entwickelnde Lösung geschaffen
werden.

\begin{figure}[htbp]
\centering
\includegraphics{images/use-case-diagram.png}
\caption{Use-Case Diagram}
\end{figure}

\newpage

\subsection{Use-Cases Beschreibung}\label{use-cases-beschreibung}

Die im Diagramm dargestellten Use-Cases werden nun noch beschrieben. Die
Use-Cases wurden numerisch nach Themenbereichen gruppiert.

\subsubsection{UC-11 Registration für den
Konfigurator}\label{uc-11-registration-fuxfcr-den-konfigurator}

\begin{longtable}[c]{@{}ll@{}}
\toprule
\begin{minipage}[b]{0.34\columnwidth}\raggedright\strut
\textbf{UseCase}
\strut\end{minipage} &
\begin{minipage}[b]{0.60\columnwidth}\raggedright\strut
\strut\end{minipage}\tabularnewline
\midrule
\endhead
\begin{minipage}[t]{0.34\columnwidth}\raggedright\strut
\textbf{Ziel}
\strut\end{minipage} &
\begin{minipage}[t]{0.60\columnwidth}\raggedright\strut
Ein Programmierer ist beim Authentifizierungsschnittstellen-Service
registriert.
\strut\end{minipage}\tabularnewline
\begin{minipage}[t]{0.34\columnwidth}\raggedright\strut
\textbf{Beschreibung}
\strut\end{minipage} &
\begin{minipage}[t]{0.60\columnwidth}\raggedright\strut
Ein Programmierer muss sich am Authentifizierungsschnittstellen-Service
registrieren können.
\strut\end{minipage}\tabularnewline
\begin{minipage}[t]{0.34\columnwidth}\raggedright\strut
\textbf{Akteure}
\strut\end{minipage} &
\begin{minipage}[t]{0.60\columnwidth}\raggedright\strut
Programmierer
\strut\end{minipage}\tabularnewline
\begin{minipage}[t]{0.34\columnwidth}\raggedright\strut
\textbf{Vorbedingung}
\strut\end{minipage} &
\begin{minipage}[t]{0.60\columnwidth}\raggedright\strut
Keine
\strut\end{minipage}\tabularnewline
\begin{minipage}[t]{0.34\columnwidth}\raggedright\strut
\textbf{Ergebnis}
\strut\end{minipage} &
\begin{minipage}[t]{0.60\columnwidth}\raggedright\strut
Registrierter Programmierer
\strut\end{minipage}\tabularnewline
\begin{minipage}[t]{0.34\columnwidth}\raggedright\strut
\textbf{Hauptszenario}
\strut\end{minipage} &
\begin{minipage}[t]{0.60\columnwidth}\raggedright\strut
Der Programmierer füllt ein Registrationsformular aus und bestätigt
seine E-Mail Adresse.
\strut\end{minipage}\tabularnewline
\begin{minipage}[t]{0.34\columnwidth}\raggedright\strut
\textbf{Alternativszenario}
\strut\end{minipage} &
\begin{minipage}[t]{0.60\columnwidth}\raggedright\strut
-
\strut\end{minipage}\tabularnewline
\bottomrule
\end{longtable}

\subsubsection{UC-12 Login Konfigurator}\label{uc-12-login-konfigurator}

\begin{longtable}[c]{@{}ll@{}}
\toprule
\begin{minipage}[b]{0.34\columnwidth}\raggedright\strut
\textbf{UseCase}
\strut\end{minipage} &
\begin{minipage}[b]{0.60\columnwidth}\raggedright\strut
\strut\end{minipage}\tabularnewline
\midrule
\endhead
\begin{minipage}[t]{0.34\columnwidth}\raggedright\strut
\textbf{Ziel}
\strut\end{minipage} &
\begin{minipage}[t]{0.60\columnwidth}\raggedright\strut
Ein Programmierer kann sich beim
Authentifizierungsschnittstellen-Service registrieren.
\strut\end{minipage}\tabularnewline
\begin{minipage}[t]{0.34\columnwidth}\raggedright\strut
\textbf{Beschreibung}
\strut\end{minipage} &
\begin{minipage}[t]{0.60\columnwidth}\raggedright\strut
Ein Programmierer muss sich am Authentifizierungsschnittstellen-Service
authentifizieren können.
\strut\end{minipage}\tabularnewline
\begin{minipage}[t]{0.34\columnwidth}\raggedright\strut
\textbf{Akteure}
\strut\end{minipage} &
\begin{minipage}[t]{0.60\columnwidth}\raggedright\strut
Programmierer
\strut\end{minipage}\tabularnewline
\begin{minipage}[t]{0.34\columnwidth}\raggedright\strut
\textbf{Vorbedingung}
\strut\end{minipage} &
\begin{minipage}[t]{0.60\columnwidth}\raggedright\strut
Der Programmierer ist registriert.
\strut\end{minipage}\tabularnewline
\begin{minipage}[t]{0.34\columnwidth}\raggedright\strut
\textbf{Ergebnis}
\strut\end{minipage} &
\begin{minipage}[t]{0.60\columnwidth}\raggedright\strut
Authentifizierter und eingeloggter Programmierer
\strut\end{minipage}\tabularnewline
\begin{minipage}[t]{0.34\columnwidth}\raggedright\strut
\textbf{Hauptszenario}
\strut\end{minipage} &
\begin{minipage}[t]{0.60\columnwidth}\raggedright\strut
Der Programmierer loggt sich mit E-Mail und Passwort am
Authentifizierungsschnittstellen-Service ein.
\strut\end{minipage}\tabularnewline
\begin{minipage}[t]{0.34\columnwidth}\raggedright\strut
\textbf{Alternativszenario}
\strut\end{minipage} &
\begin{minipage}[t]{0.60\columnwidth}\raggedright\strut
Der Programmierer sendet sich das verpasste Passwort per E-Mail zu. Er
erstellt über den im erhaltenden E-Mail enthaltenen Link ein neues
Passwort und loggt sich mit E-mail und dem neuen Passwort am
Authentifizierungsschnittstellen-Service ein.
\strut\end{minipage}\tabularnewline
\bottomrule
\end{longtable}

\subsubsection{UC-21 Konfigurieren eines
Authentifizierungsvorgang}\label{uc-21-konfigurieren-eines-authentifizierungsvorgang}

\begin{longtable}[c]{@{}ll@{}}
\toprule
\begin{minipage}[b]{0.34\columnwidth}\raggedright\strut
\textbf{UseCase}
\strut\end{minipage} &
\begin{minipage}[b]{0.60\columnwidth}\raggedright\strut
\strut\end{minipage}\tabularnewline
\midrule
\endhead
\begin{minipage}[t]{0.34\columnwidth}\raggedright\strut
\textbf{Ziel}
\strut\end{minipage} &
\begin{minipage}[t]{0.60\columnwidth}\raggedright\strut
Es ist ein neuer Authentifizierungsvorgang für ein neues Social
Media-Modul konfiguriert.
\strut\end{minipage}\tabularnewline
\begin{minipage}[t]{0.34\columnwidth}\raggedright\strut
\textbf{Beschreibung}
\strut\end{minipage} &
\begin{minipage}[t]{0.60\columnwidth}\raggedright\strut
Der Programmierer kann ein neuer Authentifizierungsvorgang eröffnen.
\strut\end{minipage}\tabularnewline
\begin{minipage}[t]{0.34\columnwidth}\raggedright\strut
\textbf{Akteure}
\strut\end{minipage} &
\begin{minipage}[t]{0.60\columnwidth}\raggedright\strut
Programmierer
\strut\end{minipage}\tabularnewline
\begin{minipage}[t]{0.34\columnwidth}\raggedright\strut
\textbf{Vorbedingung}
\strut\end{minipage} &
\begin{minipage}[t]{0.60\columnwidth}\raggedright\strut
Der Programmierer hat sich am System angemeldet.
\strut\end{minipage}\tabularnewline
\begin{minipage}[t]{0.34\columnwidth}\raggedright\strut
\textbf{Ergebnis}
\strut\end{minipage} &
\begin{minipage}[t]{0.60\columnwidth}\raggedright\strut
Neuer Authentifizierungsvorgang
\strut\end{minipage}\tabularnewline
\begin{minipage}[t]{0.34\columnwidth}\raggedright\strut
\textbf{Hauptszenario}
\strut\end{minipage} &
\begin{minipage}[t]{0.60\columnwidth}\raggedright\strut
Der Programmierer eröffnet einen neuen Authentifizierungsvorgang. Er
benennt ihn sinnig. Die zu verwende(n) Authentifizierungskomponenten
werden ausgewählt. Bei der Konfiguration unterstützen die Resultate der
Studie den Programmierer für die optimale Konfiguration. Am Ende der
Konfiguration werden die Akzeptanzkriteren für eine erfolgreiche
Authentifizierung festgelegt.
\strut\end{minipage}\tabularnewline
\begin{minipage}[t]{0.34\columnwidth}\raggedright\strut
\textbf{Alternativszenario}
\strut\end{minipage} &
\begin{minipage}[t]{0.60\columnwidth}\raggedright\strut
Ein bestehender Authentifizierungsvorgang wird dupliziert.
\strut\end{minipage}\tabularnewline
\bottomrule
\end{longtable}

\subsubsection{UC-25 Authentifizierung in vorhandenes System
einbinden}\label{uc-25-authentifizierung-in-vorhandenes-system-einbinden}

\begin{longtable}[c]{@{}ll@{}}
\toprule
\begin{minipage}[b]{0.34\columnwidth}\raggedright\strut
\textbf{UseCase}
\strut\end{minipage} &
\begin{minipage}[b]{0.60\columnwidth}\raggedright\strut
\strut\end{minipage}\tabularnewline
\midrule
\endhead
\begin{minipage}[t]{0.34\columnwidth}\raggedright\strut
\textbf{Ziel}
\strut\end{minipage} &
\begin{minipage}[t]{0.60\columnwidth}\raggedright\strut
Die Authentifizierungsschnittstelle kann in ein (bestehendes) System
eingebunden werden.
\strut\end{minipage}\tabularnewline
\begin{minipage}[t]{0.34\columnwidth}\raggedright\strut
\textbf{Beschreibung}
\strut\end{minipage} &
\begin{minipage}[t]{0.60\columnwidth}\raggedright\strut
Der Programmierer kann die Authentifizierungsschnittstelle in seinem
System integrieren.
\strut\end{minipage}\tabularnewline
\begin{minipage}[t]{0.34\columnwidth}\raggedright\strut
\textbf{Akteure}
\strut\end{minipage} &
\begin{minipage}[t]{0.60\columnwidth}\raggedright\strut
Programmierer
\strut\end{minipage}\tabularnewline
\begin{minipage}[t]{0.34\columnwidth}\raggedright\strut
\textbf{Vorbedingung}
\strut\end{minipage} &
\begin{minipage}[t]{0.60\columnwidth}\raggedright\strut
Der Programmierer hat sich am System angemeldet. Der Programmierer hat
einen neuen Authentifizierungsvorgang konfiguriert.
\strut\end{minipage}\tabularnewline
\begin{minipage}[t]{0.34\columnwidth}\raggedright\strut
\textbf{Ergebnis}
\strut\end{minipage} &
\begin{minipage}[t]{0.60\columnwidth}\raggedright\strut
Der Programmierer hat eine Möglichkeit, die
Authentifizierungsschnittstelle mit seinem konfigurierten
Authentifizierungsvorgang in seiner Software einzubinden.
\strut\end{minipage}\tabularnewline
\begin{minipage}[t]{0.34\columnwidth}\raggedright\strut
\textbf{Hauptszenario}
\strut\end{minipage} &
\begin{minipage}[t]{0.60\columnwidth}\raggedright\strut
Der Programmierer öffnet die Einbindeseite. Es werden ihm alle Schritte
zur erfolgreichen Einbindung aufgelistet. Der Code liegt
individualisiert vor. Der Programmierer kopiert den Code in sein
Programm.
\strut\end{minipage}\tabularnewline
\begin{minipage}[t]{0.34\columnwidth}\raggedright\strut
\textbf{Alternativszenario}
\strut\end{minipage} &
\begin{minipage}[t]{0.60\columnwidth}\raggedright\strut
-
\strut\end{minipage}\tabularnewline
\bottomrule
\end{longtable}

\subsubsection{UC-31 User
Authentifizieren}\label{uc-31-user-authentifizieren}

\begin{longtable}[c]{@{}ll@{}}
\toprule
\begin{minipage}[b]{0.34\columnwidth}\raggedright\strut
\textbf{UseCase}
\strut\end{minipage} &
\begin{minipage}[b]{0.60\columnwidth}\raggedright\strut
\strut\end{minipage}\tabularnewline
\midrule
\endhead
\begin{minipage}[t]{0.34\columnwidth}\raggedright\strut
\textbf{Ziel}
\strut\end{minipage} &
\begin{minipage}[t]{0.60\columnwidth}\raggedright\strut
Der User ist authtentifiziert oder der User abgelehnt.
\strut\end{minipage}\tabularnewline
\begin{minipage}[t]{0.34\columnwidth}\raggedright\strut
\textbf{Beschreibung}
\strut\end{minipage} &
\begin{minipage}[t]{0.60\columnwidth}\raggedright\strut
Der User probiert sich über den Authentifizierungsschnittstellen-Service
zu authentifizieren um an einem Social-Media Modul teilzunehmen.
\strut\end{minipage}\tabularnewline
\begin{minipage}[t]{0.34\columnwidth}\raggedright\strut
\textbf{Akteure}
\strut\end{minipage} &
\begin{minipage}[t]{0.60\columnwidth}\raggedright\strut
User
\strut\end{minipage}\tabularnewline
\begin{minipage}[t]{0.34\columnwidth}\raggedright\strut
\textbf{Vorbedingung}
\strut\end{minipage} &
\begin{minipage}[t]{0.60\columnwidth}\raggedright\strut
Der Programmierer hat den Authentifizierungsvorgang konfiguriert und in
seinem System eingebunden.
\strut\end{minipage}\tabularnewline
\begin{minipage}[t]{0.34\columnwidth}\raggedright\strut
\textbf{Ergebnis}
\strut\end{minipage} &
\begin{minipage}[t]{0.60\columnwidth}\raggedright\strut
Der Authentifizierungsschnittstellen-Service authentifiziert den User
oder lehnt ihn ab.
\strut\end{minipage}\tabularnewline
\begin{minipage}[t]{0.34\columnwidth}\raggedright\strut
\textbf{Hauptszenario}
\strut\end{minipage} &
\begin{minipage}[t]{0.60\columnwidth}\raggedright\strut
Der User wird vom Social-Media-Modul an den
Authentifizierungsschnittstellen-Service weitergeleitet. Der User
authentifiziert sich. Der User kann die Eingabe des Social-Media Moduls
erfolgreich abschliessen.
\strut\end{minipage}\tabularnewline
\begin{minipage}[t]{0.34\columnwidth}\raggedright\strut
\textbf{Alternativszenario}
\strut\end{minipage} &
\begin{minipage}[t]{0.60\columnwidth}\raggedright\strut
Der User wird vom Social Media-Modul an den
Authentifizierungsschnittstellen-Service weitergeleitet. Der User wird
vom System abgelehnt. Der User kann die Eingabe des Social-Media-Modul
nicht erfolgreich abschliessen.
\strut\end{minipage}\tabularnewline
\bottomrule
\end{longtable}

\subsubsection{UC-41 Report eines
Authentifizierungsvorgangs}\label{uc-41-report-eines-authentifizierungsvorgangs}

\begin{longtable}[c]{@{}ll@{}}
\toprule
\begin{minipage}[b]{0.34\columnwidth}\raggedright\strut
\textbf{UseCase}
\strut\end{minipage} &
\begin{minipage}[b]{0.60\columnwidth}\raggedright\strut
\strut\end{minipage}\tabularnewline
\midrule
\endhead
\begin{minipage}[t]{0.34\columnwidth}\raggedright\strut
\textbf{Ziel}
\strut\end{minipage} &
\begin{minipage}[t]{0.60\columnwidth}\raggedright\strut
Die Verwendung des Authentifizierungsvorgangs ist übersichtlich
dargestellt.
\strut\end{minipage}\tabularnewline
\begin{minipage}[t]{0.34\columnwidth}\raggedright\strut
\textbf{Beschreibung}
\strut\end{minipage} &
\begin{minipage}[t]{0.60\columnwidth}\raggedright\strut
Um den Verwendung des Authentifizierungsvorgangs auszuwerten, soll ein
Report erstellt werden.
\strut\end{minipage}\tabularnewline
\begin{minipage}[t]{0.34\columnwidth}\raggedright\strut
\textbf{Akteure}
\strut\end{minipage} &
\begin{minipage}[t]{0.60\columnwidth}\raggedright\strut
Programmierer
\strut\end{minipage}\tabularnewline
\begin{minipage}[t]{0.34\columnwidth}\raggedright\strut
\textbf{Vorbedingung}
\strut\end{minipage} &
\begin{minipage}[t]{0.60\columnwidth}\raggedright\strut
Der Programmierer hat sich am System angemeldet. Der Programmierer hat
einen neuen Authentifizierungsvorgang konfiguriert. (Der
Authentifizerungsvorgang ist eingebunden und verwendet worden).
\strut\end{minipage}\tabularnewline
\begin{minipage}[t]{0.34\columnwidth}\raggedright\strut
\textbf{Ergebnis}
\strut\end{minipage} &
\begin{minipage}[t]{0.60\columnwidth}\raggedright\strut
Report eines Authentifizierungsvorgangs
\strut\end{minipage}\tabularnewline
\begin{minipage}[t]{0.34\columnwidth}\raggedright\strut
\textbf{Hauptszenario}
\strut\end{minipage} &
\begin{minipage}[t]{0.60\columnwidth}\raggedright\strut
Nach Beenden eines Quizes, Votings, Wettbewerbs logt sich der
Programmierer im System ein und generiert einen automatisierten Report,
um die Verwendung des Authentifizierungsvorgangs auszuwerten.
\strut\end{minipage}\tabularnewline
\begin{minipage}[t]{0.34\columnwidth}\raggedright\strut
\textbf{Alternativszenario}
\strut\end{minipage} &
\begin{minipage}[t]{0.60\columnwidth}\raggedright\strut
Um den Zwischenstand eines Quizes, Votings, Wettbewerbs auszuwerten logt
sich der Programmierer im System ein und generiert einen automatisierten
Report, um die Verwendung des Authentifizierungsvorgangs auszuwerten.
\strut\end{minipage}\tabularnewline
\bottomrule
\end{longtable}

\subsubsection{UC-51 Wartbarkeit des
Authentifizierungsservices}\label{uc-51-wartbarkeit-des-authentifizierungsservices}

\begin{longtable}[c]{@{}ll@{}}
\toprule
\begin{minipage}[b]{0.34\columnwidth}\raggedright\strut
\textbf{UseCase}
\strut\end{minipage} &
\begin{minipage}[b]{0.60\columnwidth}\raggedright\strut
\strut\end{minipage}\tabularnewline
\midrule
\endhead
\begin{minipage}[t]{0.34\columnwidth}\raggedright\strut
\textbf{Ziel}
\strut\end{minipage} &
\begin{minipage}[t]{0.60\columnwidth}\raggedright\strut
Der Authentifizierungsschnittstellen-Service soll mit geringem Aufwand
angepasst werden können.
\strut\end{minipage}\tabularnewline
\begin{minipage}[t]{0.34\columnwidth}\raggedright\strut
\textbf{Beschreibung}
\strut\end{minipage} &
\begin{minipage}[t]{0.60\columnwidth}\raggedright\strut
\strut\end{minipage}\tabularnewline
\begin{minipage}[t]{0.34\columnwidth}\raggedright\strut
\textbf{Akteure}
\strut\end{minipage} &
\begin{minipage}[t]{0.60\columnwidth}\raggedright\strut
Entwicklungsteam-Mitglied
\strut\end{minipage}\tabularnewline
\begin{minipage}[t]{0.34\columnwidth}\raggedright\strut
\textbf{Vorbedingung}
\strut\end{minipage} &
\begin{minipage}[t]{0.60\columnwidth}\raggedright\strut
Das Entwicklungsteam-Mitglied hat Zugriff auf das
Entwicklungs-Repository, Testsystem und Livesystem
\strut\end{minipage}\tabularnewline
\begin{minipage}[t]{0.34\columnwidth}\raggedright\strut
\textbf{Ergebnis}
\strut\end{minipage} &
\begin{minipage}[t]{0.60\columnwidth}\raggedright\strut
Die Anpassung ist integriert.
\strut\end{minipage}\tabularnewline
\begin{minipage}[t]{0.34\columnwidth}\raggedright\strut
\textbf{Hauptszenario}
\strut\end{minipage} &
\begin{minipage}[t]{0.60\columnwidth}\raggedright\strut
Dank eingehaltenen Coderichtlinien ist es einfach möglich, die Anpassung
einzupflegen.
\strut\end{minipage}\tabularnewline
\begin{minipage}[t]{0.34\columnwidth}\raggedright\strut
\textbf{Alternativszenario}
\strut\end{minipage} &
\begin{minipage}[t]{0.60\columnwidth}\raggedright\strut
-
\strut\end{minipage}\tabularnewline
\bottomrule
\end{longtable}

\newpage

\section{Anforderungen}\label{anforderungen-1}

Die Anforderungen sollen basierend auf der Satzschablone erstellt
werden. Ziel ist es, sprachliche Missverständnisse dadurch zu vermeiden.
Die Schablone fördert eine syntaktische Eindeutigkeit der Anforderungen
und einen optimalen Zeit- und Kostenrahmen für die Verfassung.

\subsection{Aufbau}\label{aufbau}

Die folgenden Abbildungen zeigen den Aufbau der Satzschablonen. Es wird
zwischen der grundlegenden Version ohne zeitlichen oder
bedienungsorientierten Aspekt und der Schablone mit diesen Eigenschaften
unterschieden.

\begin{figure}[htbp]
\centering
\includegraphics{images/basis-schablone.jpg}
\caption[Basis Schablone \emph{Quelle Rupp}]{Basis Schablone
\emph{Quelle Rupp}\footnotemark{}}
\end{figure}
\footnotetext{Rupp Bilder sind aus dem Buch Basiswissen Requirements
  Engineering \autocite{rupp}}

\begin{figure}[htbp]
\centering
\includegraphics{images/erweiterte-schablone.jpg}
\caption[Erweiterte Schablone \emph{Quelle Rupp}]{Erweiterte Schablone
\emph{Quelle Rupp}\footnotemark{}}
\end{figure}
\footnotetext{Rupp Bilder sind aus dem Buch Basiswissen Requirements
  Engineering \autocite{rupp}}

\newpage

\section{Funktionale Anforderungen}\label{funktionale-anforderungen}

Die funktionalen Anforderungen legen die Funktionen des
Authentifizierungsschnittstellen-Service fest. Die Wünsche des
Arbeitgebers aus sind als Anforderungen umformuliert. Die funktionalen
Anforderungen dienen als Grundlage für die Testfälle. Die Testfälle
wiederum, bringen den Beweis dar, dass alle gewünschten Funktionen
implementiert wurden.

Funktionale Anforderungen werden als FREQ-\emph{Identifikation}
bezeichnet.

\subsection{FREQ-111 Programmierer
Registration}\label{freq-111-programmierer-registration}

\begin{longtable}[c]{@{}ll@{}}
\toprule
\begin{minipage}[t]{0.20\columnwidth}\raggedright\strut
\textbf{UC-Referenz}
\strut\end{minipage} &
\begin{minipage}[t]{0.74\columnwidth}\raggedright\strut
UC-11
\strut\end{minipage}\tabularnewline
\begin{minipage}[t]{0.20\columnwidth}\raggedright\strut
\textbf{Beschreibung}
\strut\end{minipage} &
\begin{minipage}[t]{0.74\columnwidth}\raggedright\strut
Ein Programmierer kann sich beim
Authentifizierungsschnittstellen-Service registrieren.
\strut\end{minipage}\tabularnewline
\begin{minipage}[t]{0.20\columnwidth}\raggedright\strut
\textbf{Techn. Risiko}
\strut\end{minipage} &
\begin{minipage}[t]{0.74\columnwidth}\raggedright\strut
Niedrig
\strut\end{minipage}\tabularnewline
\begin{minipage}[t]{0.20\columnwidth}\raggedright\strut
\textbf{Business Value}
\strut\end{minipage} &
\begin{minipage}[t]{0.74\columnwidth}\raggedright\strut
Hoch
\strut\end{minipage}\tabularnewline
\bottomrule
\end{longtable}

\subsection{FREQ-112 Programmierer
Login}\label{freq-112-programmierer-login}

\begin{longtable}[c]{@{}ll@{}}
\toprule
\begin{minipage}[t]{0.20\columnwidth}\raggedright\strut
\textbf{UC-Referenz}
\strut\end{minipage} &
\begin{minipage}[t]{0.74\columnwidth}\raggedright\strut
UC-12
\strut\end{minipage}\tabularnewline
\begin{minipage}[t]{0.20\columnwidth}\raggedright\strut
\textbf{Beschreibung}
\strut\end{minipage} &
\begin{minipage}[t]{0.74\columnwidth}\raggedright\strut
Ein Programmierer muss sich beim
Authentifizierungsschnittstellen-Service mittels E-Mail und Passwort
anmelden.
\strut\end{minipage}\tabularnewline
\begin{minipage}[t]{0.20\columnwidth}\raggedright\strut
\textbf{Techn. Risiko}
\strut\end{minipage} &
\begin{minipage}[t]{0.74\columnwidth}\raggedright\strut
Niedrig
\strut\end{minipage}\tabularnewline
\begin{minipage}[t]{0.20\columnwidth}\raggedright\strut
\textbf{Business Value}
\strut\end{minipage} &
\begin{minipage}[t]{0.74\columnwidth}\raggedright\strut
Hoch
\strut\end{minipage}\tabularnewline
\bottomrule
\end{longtable}

\subsection{FREQ-113 Programmierer Passwort
vergessen}\label{freq-113-programmierer-passwort-vergessen}

\begin{longtable}[c]{@{}ll@{}}
\toprule
\begin{minipage}[t]{0.20\columnwidth}\raggedright\strut
\textbf{UC-Referenz}
\strut\end{minipage} &
\begin{minipage}[t]{0.74\columnwidth}\raggedright\strut
UC-11, UC-12
\strut\end{minipage}\tabularnewline
\begin{minipage}[t]{0.20\columnwidth}\raggedright\strut
\textbf{Beschreibung}
\strut\end{minipage} &
\begin{minipage}[t]{0.74\columnwidth}\raggedright\strut
Ein Programmierer kann ein Passwort per E-Mail anfordern.
\strut\end{minipage}\tabularnewline
\begin{minipage}[t]{0.20\columnwidth}\raggedright\strut
\textbf{Techn. Risiko}
\strut\end{minipage} &
\begin{minipage}[t]{0.74\columnwidth}\raggedright\strut
Niedrig
\strut\end{minipage}\tabularnewline
\begin{minipage}[t]{0.20\columnwidth}\raggedright\strut
\textbf{Business Value}
\strut\end{minipage} &
\begin{minipage}[t]{0.74\columnwidth}\raggedright\strut
Hoch
\strut\end{minipage}\tabularnewline
\bottomrule
\end{longtable}

\subsection{FREQ-114 Programmierer Passwort
ändern}\label{freq-114-programmierer-passwort-uxe4ndern}

\begin{longtable}[c]{@{}ll@{}}
\toprule
\begin{minipage}[t]{0.20\columnwidth}\raggedright\strut
\textbf{UC-Referenz}
\strut\end{minipage} &
\begin{minipage}[t]{0.74\columnwidth}\raggedright\strut
UC-11, UC-12
\strut\end{minipage}\tabularnewline
\begin{minipage}[t]{0.20\columnwidth}\raggedright\strut
\textbf{Beschreibung}
\strut\end{minipage} &
\begin{minipage}[t]{0.74\columnwidth}\raggedright\strut
Ein Programmierer kann sein Passwort ändern. Dafür muss der
Programmierer das alte und neue Passwort angeben.
\strut\end{minipage}\tabularnewline
\begin{minipage}[t]{0.20\columnwidth}\raggedright\strut
\textbf{Techn. Risiko}
\strut\end{minipage} &
\begin{minipage}[t]{0.74\columnwidth}\raggedright\strut
Niedrig
\strut\end{minipage}\tabularnewline
\begin{minipage}[t]{0.20\columnwidth}\raggedright\strut
\textbf{Business Value}
\strut\end{minipage} &
\begin{minipage}[t]{0.74\columnwidth}\raggedright\strut
Hoch
\strut\end{minipage}\tabularnewline
\bottomrule
\end{longtable}

\subsection{FREQ-211 Konfigurieren eines neuen Social-Media-Modul
Authentifizierungsvorgangs}\label{freq-211-konfigurieren-eines-neuen-social-media-modul-authentifizierungsvorgangs}

\begin{longtable}[c]{@{}ll@{}}
\toprule
\begin{minipage}[t]{0.20\columnwidth}\raggedright\strut
\textbf{UC-Referenz}
\strut\end{minipage} &
\begin{minipage}[t]{0.74\columnwidth}\raggedright\strut
UC-21
\strut\end{minipage}\tabularnewline
\begin{minipage}[t]{0.20\columnwidth}\raggedright\strut
\textbf{Beschreibung}
\strut\end{minipage} &
\begin{minipage}[t]{0.74\columnwidth}\raggedright\strut
Ein Programmierer kann einen neuen Authentifizierungsvorgang für sein
neues Social-Media Modul erfassen.
\strut\end{minipage}\tabularnewline
\begin{minipage}[t]{0.20\columnwidth}\raggedright\strut
\textbf{Techn. Risiko}
\strut\end{minipage} &
\begin{minipage}[t]{0.74\columnwidth}\raggedright\strut
Niedrig
\strut\end{minipage}\tabularnewline
\begin{minipage}[t]{0.20\columnwidth}\raggedright\strut
\textbf{Business Value}
\strut\end{minipage} &
\begin{minipage}[t]{0.74\columnwidth}\raggedright\strut
Sehr Hoch
\strut\end{minipage}\tabularnewline
\bottomrule
\end{longtable}

\subsection{FREQ-212 Antworten der Umfrage in Authentifizerungsservice
importieren}\label{freq-212-antworten-der-umfrage-in-authentifizerungsservice-importieren}

\begin{longtable}[c]{@{}ll@{}}
\toprule
\begin{minipage}[t]{0.20\columnwidth}\raggedright\strut
\textbf{UC-Referenz}
\strut\end{minipage} &
\begin{minipage}[t]{0.74\columnwidth}\raggedright\strut
UC-21
\strut\end{minipage}\tabularnewline
\begin{minipage}[t]{0.20\columnwidth}\raggedright\strut
\textbf{Beschreibung}
\strut\end{minipage} &
\begin{minipage}[t]{0.74\columnwidth}\raggedright\strut
Die Umfrageantworten müssen in den Authentifizerungsservice
abgespeichert werden können. Der Import ist über direkt über die
Datenbank realisierbar.
\strut\end{minipage}\tabularnewline
\begin{minipage}[t]{0.20\columnwidth}\raggedright\strut
\textbf{Techn. Risiko}
\strut\end{minipage} &
\begin{minipage}[t]{0.74\columnwidth}\raggedright\strut
Niedrig
\strut\end{minipage}\tabularnewline
\begin{minipage}[t]{0.20\columnwidth}\raggedright\strut
\textbf{Business Value}
\strut\end{minipage} &
\begin{minipage}[t]{0.74\columnwidth}\raggedright\strut
Mittel
\strut\end{minipage}\tabularnewline
\bottomrule
\end{longtable}

\subsection{FREQ-213 Umfrageergebnisse zur Konfiguration
nutzen}\label{freq-213-umfrageergebnisse-zur-konfiguration-nutzen}

\begin{longtable}[c]{@{}ll@{}}
\toprule
\begin{minipage}[t]{0.20\columnwidth}\raggedright\strut
\textbf{UC-Referenz}
\strut\end{minipage} &
\begin{minipage}[t]{0.74\columnwidth}\raggedright\strut
UC-21
\strut\end{minipage}\tabularnewline
\begin{minipage}[t]{0.20\columnwidth}\raggedright\strut
\textbf{Beschreibung}
\strut\end{minipage} &
\begin{minipage}[t]{0.74\columnwidth}\raggedright\strut
Ein Programmierer kann zur Konfiguration des Authentifizierungsvorangs
die Umfrageergebnisse visualisiert nutzen. Dabei sollen verschiedene
Auswertungsmöglichkeiten zur Anstrengung und Akzeptanz der
Sicherheitsstufe möglich sein.
\strut\end{minipage}\tabularnewline
\begin{minipage}[t]{0.20\columnwidth}\raggedright\strut
\textbf{Techn. Risiko}
\strut\end{minipage} &
\begin{minipage}[t]{0.74\columnwidth}\raggedright\strut
Niedrig
\strut\end{minipage}\tabularnewline
\begin{minipage}[t]{0.20\columnwidth}\raggedright\strut
\textbf{Business Value}
\strut\end{minipage} &
\begin{minipage}[t]{0.74\columnwidth}\raggedright\strut
Mittel
\strut\end{minipage}\tabularnewline
\bottomrule
\end{longtable}

\subsection{FREQ-214 Anpassen eines
Authentifizierungsvorgangs}\label{freq-214-anpassen-eines-authentifizierungsvorgangs}

\begin{longtable}[c]{@{}ll@{}}
\toprule
\begin{minipage}[t]{0.20\columnwidth}\raggedright\strut
\textbf{UC-Referenz}
\strut\end{minipage} &
\begin{minipage}[t]{0.74\columnwidth}\raggedright\strut
UC-21
\strut\end{minipage}\tabularnewline
\begin{minipage}[t]{0.20\columnwidth}\raggedright\strut
\textbf{Beschreibung}
\strut\end{minipage} &
\begin{minipage}[t]{0.74\columnwidth}\raggedright\strut
Ein Programmierer kann ein neues Social-Media-Modul erfassen
\strut\end{minipage}\tabularnewline
\begin{minipage}[t]{0.20\columnwidth}\raggedright\strut
\textbf{Techn. Risiko}
\strut\end{minipage} &
\begin{minipage}[t]{0.74\columnwidth}\raggedright\strut
Hoch
\strut\end{minipage}\tabularnewline
\begin{minipage}[t]{0.20\columnwidth}\raggedright\strut
\textbf{Business Value}
\strut\end{minipage} &
\begin{minipage}[t]{0.74\columnwidth}\raggedright\strut
Mittel
\strut\end{minipage}\tabularnewline
\bottomrule
\end{longtable}

\subsection{FREQ-215 Authentifizerungs-Stufe
auswählen}\label{freq-215-authentifizerungs-stufe-auswuxe4hlen}

\begin{longtable}[c]{@{}ll@{}}
\toprule
\begin{minipage}[t]{0.20\columnwidth}\raggedright\strut
\textbf{UC-Referenz}
\strut\end{minipage} &
\begin{minipage}[t]{0.74\columnwidth}\raggedright\strut
UC-21
\strut\end{minipage}\tabularnewline
\begin{minipage}[t]{0.20\columnwidth}\raggedright\strut
\textbf{Beschreibung}
\strut\end{minipage} &
\begin{minipage}[t]{0.74\columnwidth}\raggedright\strut
Ein Programmierer muss eine Authentifizerungsstufe für den
Authentifizierungsvorgangs auswählen.
\strut\end{minipage}\tabularnewline
\begin{minipage}[t]{0.20\columnwidth}\raggedright\strut
\textbf{Techn. Risiko}
\strut\end{minipage} &
\begin{minipage}[t]{0.74\columnwidth}\raggedright\strut
Niederig
\strut\end{minipage}\tabularnewline
\begin{minipage}[t]{0.20\columnwidth}\raggedright\strut
\textbf{Business Value}
\strut\end{minipage} &
\begin{minipage}[t]{0.74\columnwidth}\raggedright\strut
Hoch
\strut\end{minipage}\tabularnewline
\bottomrule
\end{longtable}

\subsection{FREQ-251 Generierung von Code für Einbinden in ein
vorhandenes
System}\label{freq-251-generierung-von-code-fuxfcr-einbinden-in-ein-vorhandenes-system}

\begin{longtable}[c]{@{}ll@{}}
\toprule
\begin{minipage}[t]{0.20\columnwidth}\raggedright\strut
\textbf{UC-Referenz}
\strut\end{minipage} &
\begin{minipage}[t]{0.74\columnwidth}\raggedright\strut
UC-25
\strut\end{minipage}\tabularnewline
\begin{minipage}[t]{0.20\columnwidth}\raggedright\strut
\textbf{Beschreibung}
\strut\end{minipage} &
\begin{minipage}[t]{0.74\columnwidth}\raggedright\strut
Ein Programmierer kann einen Code generieren lassen. Dieser Code soll
ihm die Integration in sein System vereinfachen.
\strut\end{minipage}\tabularnewline
\begin{minipage}[t]{0.20\columnwidth}\raggedright\strut
\textbf{Techn. Risiko}
\strut\end{minipage} &
\begin{minipage}[t]{0.74\columnwidth}\raggedright\strut
Sehr Hoch
\strut\end{minipage}\tabularnewline
\begin{minipage}[t]{0.20\columnwidth}\raggedright\strut
\textbf{Business Value}
\strut\end{minipage} &
\begin{minipage}[t]{0.74\columnwidth}\raggedright\strut
Hoch
\strut\end{minipage}\tabularnewline
\bottomrule
\end{longtable}

\subsection{FREQ-311 Authentifizieren}\label{freq-311-authentifizieren}

\begin{longtable}[c]{@{}ll@{}}
\toprule
\begin{minipage}[t]{0.20\columnwidth}\raggedright\strut
\textbf{UC-Referenz}
\strut\end{minipage} &
\begin{minipage}[t]{0.74\columnwidth}\raggedright\strut
UC-31
\strut\end{minipage}\tabularnewline
\begin{minipage}[t]{0.20\columnwidth}\raggedright\strut
\textbf{Beschreibung}
\strut\end{minipage} &
\begin{minipage}[t]{0.74\columnwidth}\raggedright\strut
Ein User kann sich über den Authentifizierungsschnittstellen-Service
authentifizieren um am Social-Media-Modul teilzunehmen. Der
Authentifizierungsschnittstellen-Service authentifiziert oder lehnt den
User ab.
\strut\end{minipage}\tabularnewline
\begin{minipage}[t]{0.20\columnwidth}\raggedright\strut
\textbf{Techn. Risiko}
\strut\end{minipage} &
\begin{minipage}[t]{0.74\columnwidth}\raggedright\strut
Mittel
\strut\end{minipage}\tabularnewline
\begin{minipage}[t]{0.20\columnwidth}\raggedright\strut
\textbf{Business Value}
\strut\end{minipage} &
\begin{minipage}[t]{0.74\columnwidth}\raggedright\strut
Sehr Hoch
\strut\end{minipage}\tabularnewline
\bottomrule
\end{longtable}

\subsection{FREQ-411 Report der Authentifizierungen
generieren}\label{freq-411-report-der-authentifizierungen-generieren}

\begin{longtable}[c]{@{}ll@{}}
\toprule
\begin{minipage}[t]{0.20\columnwidth}\raggedright\strut
\textbf{UC-Referenz}
\strut\end{minipage} &
\begin{minipage}[t]{0.74\columnwidth}\raggedright\strut
UC-41
\strut\end{minipage}\tabularnewline
\begin{minipage}[t]{0.20\columnwidth}\raggedright\strut
\textbf{Beschreibung}
\strut\end{minipage} &
\begin{minipage}[t]{0.74\columnwidth}\raggedright\strut
Der Programmierer kann einen Report generieren. Der Report soll die
Verwendung übersichtlich darstellen.
\strut\end{minipage}\tabularnewline
\begin{minipage}[t]{0.20\columnwidth}\raggedright\strut
\textbf{Techn. Risiko}
\strut\end{minipage} &
\begin{minipage}[t]{0.74\columnwidth}\raggedright\strut
Mittel
\strut\end{minipage}\tabularnewline
\begin{minipage}[t]{0.20\columnwidth}\raggedright\strut
\textbf{Business Value}
\strut\end{minipage} &
\begin{minipage}[t]{0.74\columnwidth}\raggedright\strut
Sehr Hoch
\strut\end{minipage}\tabularnewline
\bottomrule
\end{longtable}

\newpage

\section{Nicht Funktionale
Anforderungen}\label{nicht-funktionale-anforderungen}

Nicht Funktionale Anforderungen werden als FREQ-\emph{Identifikation}
bezeichnet.

\subsection{NFREQ-110
Betriebsystemunabhängigkeit}\label{nfreq-110-betriebsystemunabhuxe4ngigkeit}

\begin{longtable}[c]{@{}ll@{}}
\toprule
\begin{minipage}[t]{0.20\columnwidth}\raggedright\strut
\textbf{UC-Referenz}
\strut\end{minipage} &
\begin{minipage}[t]{0.74\columnwidth}\raggedright\strut
Alle
\strut\end{minipage}\tabularnewline
\begin{minipage}[t]{0.20\columnwidth}\raggedright\strut
\textbf{Beschreibung}
\strut\end{minipage} &
\begin{minipage}[t]{0.74\columnwidth}\raggedright\strut
Der Authentifizierungsschnittstellen-Service muss auf allen bekannten
Betriebsystemen mit HTML5 und javascriptfähigen Browser verwendet werden
können.
\strut\end{minipage}\tabularnewline
\begin{minipage}[t]{0.20\columnwidth}\raggedright\strut
\textbf{Techn. Risiko}
\strut\end{minipage} &
\begin{minipage}[t]{0.74\columnwidth}\raggedright\strut
Mittel
\strut\end{minipage}\tabularnewline
\begin{minipage}[t]{0.20\columnwidth}\raggedright\strut
\textbf{Business Value}
\strut\end{minipage} &
\begin{minipage}[t]{0.74\columnwidth}\raggedright\strut
Sehr Hoch
\strut\end{minipage}\tabularnewline
\bottomrule
\end{longtable}

\subsection{NFREQ-115 Wartbarkeit}\label{nfreq-115-wartbarkeit}

\begin{longtable}[c]{@{}ll@{}}
\toprule
\begin{minipage}[t]{0.20\columnwidth}\raggedright\strut
\textbf{UC-Referenz}
\strut\end{minipage} &
\begin{minipage}[t]{0.74\columnwidth}\raggedright\strut
UC-51
\strut\end{minipage}\tabularnewline
\begin{minipage}[t]{0.20\columnwidth}\raggedright\strut
\textbf{Beschreibung}
\strut\end{minipage} &
\begin{minipage}[t]{0.74\columnwidth}\raggedright\strut
Die Wartbarkeit des Systems soll sichergestellt werden.
\strut\end{minipage}\tabularnewline
\begin{minipage}[t]{0.20\columnwidth}\raggedright\strut
\textbf{Techn. Risiko}
\strut\end{minipage} &
\begin{minipage}[t]{0.74\columnwidth}\raggedright\strut
Mittel
\strut\end{minipage}\tabularnewline
\begin{minipage}[t]{0.20\columnwidth}\raggedright\strut
\textbf{Business Value}
\strut\end{minipage} &
\begin{minipage}[t]{0.74\columnwidth}\raggedright\strut
Mittel
\strut\end{minipage}\tabularnewline
\bottomrule
\end{longtable}

\subsection{NFREQ-120 Einfache
Integration}\label{nfreq-120-einfache-integration}

\begin{longtable}[c]{@{}ll@{}}
\toprule
\begin{minipage}[t]{0.20\columnwidth}\raggedright\strut
\textbf{UC-Referenz}
\strut\end{minipage} &
\begin{minipage}[t]{0.74\columnwidth}\raggedright\strut
UC-25, UC-21, UC22
\strut\end{minipage}\tabularnewline
\begin{minipage}[t]{0.20\columnwidth}\raggedright\strut
\textbf{Beschreibung}
\strut\end{minipage} &
\begin{minipage}[t]{0.74\columnwidth}\raggedright\strut
Der Authentifizierungsschnittstellen-Service soll einfach im vorhandenen
System eingebunden werden können.
\strut\end{minipage}\tabularnewline
\begin{minipage}[t]{0.20\columnwidth}\raggedright\strut
\textbf{Techn. Risiko}
\strut\end{minipage} &
\begin{minipage}[t]{0.74\columnwidth}\raggedright\strut
Mittel
\strut\end{minipage}\tabularnewline
\begin{minipage}[t]{0.20\columnwidth}\raggedright\strut
\textbf{Business Value}
\strut\end{minipage} &
\begin{minipage}[t]{0.74\columnwidth}\raggedright\strut
hoch
\strut\end{minipage}\tabularnewline
\bottomrule
\end{longtable}

\subsection{NFREQ-122 Einfache und verständliche visuelle
Konfiguration}\label{nfreq-122-einfache-und-verstuxe4ndliche-visuelle-konfiguration}

\begin{longtable}[c]{@{}ll@{}}
\toprule
\begin{minipage}[t]{0.20\columnwidth}\raggedright\strut
\textbf{UC-Referenz}
\strut\end{minipage} &
\begin{minipage}[t]{0.74\columnwidth}\raggedright\strut
UC-25, UC-21, UC22
\strut\end{minipage}\tabularnewline
\begin{minipage}[t]{0.20\columnwidth}\raggedright\strut
\textbf{Beschreibung}
\strut\end{minipage} &
\begin{minipage}[t]{0.74\columnwidth}\raggedright\strut
Der Authentifizierungsschnittstellen-Service soll einfach und
verständlich optisch konfiguriert werden können.
\strut\end{minipage}\tabularnewline
\begin{minipage}[t]{0.20\columnwidth}\raggedright\strut
\textbf{Techn. Risiko}
\strut\end{minipage} &
\begin{minipage}[t]{0.74\columnwidth}\raggedright\strut
Sehr hoch
\strut\end{minipage}\tabularnewline
\begin{minipage}[t]{0.20\columnwidth}\raggedright\strut
\textbf{Business Value}
\strut\end{minipage} &
\begin{minipage}[t]{0.74\columnwidth}\raggedright\strut
Mittel
\strut\end{minipage}\tabularnewline
\bottomrule
\end{longtable}

\subsection{NFREQ-126 Einfache und verständliche
Authentifizierung}\label{nfreq-126-einfache-und-verstuxe4ndliche-authentifizierung}

\begin{longtable}[c]{@{}ll@{}}
\toprule
\begin{minipage}[t]{0.20\columnwidth}\raggedright\strut
\textbf{UC-Referenz}
\strut\end{minipage} &
\begin{minipage}[t]{0.74\columnwidth}\raggedright\strut
UC-31
\strut\end{minipage}\tabularnewline
\begin{minipage}[t]{0.20\columnwidth}\raggedright\strut
\textbf{Beschreibung}
\strut\end{minipage} &
\begin{minipage}[t]{0.74\columnwidth}\raggedright\strut
Der User soll einfach und verständlich optisch konfiguriert werden
können.
\strut\end{minipage}\tabularnewline
\begin{minipage}[t]{0.20\columnwidth}\raggedright\strut
\textbf{Techn. Risiko}
\strut\end{minipage} &
\begin{minipage}[t]{0.74\columnwidth}\raggedright\strut
Sehr hoch
\strut\end{minipage}\tabularnewline
\begin{minipage}[t]{0.20\columnwidth}\raggedright\strut
\textbf{Business Value}
\strut\end{minipage} &
\begin{minipage}[t]{0.74\columnwidth}\raggedright\strut
Mittel
\strut\end{minipage}\tabularnewline
\bottomrule
\end{longtable}

\subsection{NFREQ-127 Responsives Design für
Authentifizerung}\label{nfreq-127-responsives-design-fuxfcr-authentifizerung}

\begin{longtable}[c]{@{}ll@{}}
\toprule
\begin{minipage}[t]{0.20\columnwidth}\raggedright\strut
\textbf{UC-Referenz}
\strut\end{minipage} &
\begin{minipage}[t]{0.74\columnwidth}\raggedright\strut
UC-25, UC-21, UC22
\strut\end{minipage}\tabularnewline
\begin{minipage}[t]{0.20\columnwidth}\raggedright\strut
\textbf{Beschreibung}
\strut\end{minipage} &
\begin{minipage}[t]{0.74\columnwidth}\raggedright\strut
Der User soll sich mit Desktop, Tablet und Smartphone authentifizieren
können
\strut\end{minipage}\tabularnewline
\begin{minipage}[t]{0.20\columnwidth}\raggedright\strut
\textbf{Techn. Risiko}
\strut\end{minipage} &
\begin{minipage}[t]{0.74\columnwidth}\raggedright\strut
Sehr hoch
\strut\end{minipage}\tabularnewline
\begin{minipage}[t]{0.20\columnwidth}\raggedright\strut
\textbf{Business Value}
\strut\end{minipage} &
\begin{minipage}[t]{0.74\columnwidth}\raggedright\strut
Mittel
\strut\end{minipage}\tabularnewline
\bottomrule
\end{longtable}

\subsection{NFREQ-130 Performance}\label{nfreq-130-performance}

\begin{longtable}[c]{@{}ll@{}}
\toprule
\begin{minipage}[t]{0.20\columnwidth}\raggedright\strut
\textbf{UC-Referenz}
\strut\end{minipage} &
\begin{minipage}[t]{0.74\columnwidth}\raggedright\strut
UC-31
\strut\end{minipage}\tabularnewline
\begin{minipage}[t]{0.20\columnwidth}\raggedright\strut
\textbf{Beschreibung}
\strut\end{minipage} &
\begin{minipage}[t]{0.74\columnwidth}\raggedright\strut
Das System soll insbesondere an der Stelle der Authentifzierung
Performant sein.
\strut\end{minipage}\tabularnewline
\begin{minipage}[t]{0.20\columnwidth}\raggedright\strut
\textbf{Techn. Risiko}
\strut\end{minipage} &
\begin{minipage}[t]{0.74\columnwidth}\raggedright\strut
Sehr hoch
\strut\end{minipage}\tabularnewline
\begin{minipage}[t]{0.20\columnwidth}\raggedright\strut
\textbf{Business Value}
\strut\end{minipage} &
\begin{minipage}[t]{0.74\columnwidth}\raggedright\strut
Mittel
\strut\end{minipage}\tabularnewline
\bottomrule
\end{longtable}

\subsection{NFREQ-132 Skalierbar}\label{nfreq-132-skalierbar}

\begin{longtable}[c]{@{}ll@{}}
\toprule
\begin{minipage}[t]{0.20\columnwidth}\raggedright\strut
\textbf{UC-Referenz}
\strut\end{minipage} &
\begin{minipage}[t]{0.74\columnwidth}\raggedright\strut
UC-31, UC-25, UC-21, UC22
\strut\end{minipage}\tabularnewline
\begin{minipage}[t]{0.20\columnwidth}\raggedright\strut
\textbf{Beschreibung}
\strut\end{minipage} &
\begin{minipage}[t]{0.74\columnwidth}\raggedright\strut
Das System soll eine hohe Skalierbarkeit aufweisen.
\strut\end{minipage}\tabularnewline
\begin{minipage}[t]{0.20\columnwidth}\raggedright\strut
\textbf{Techn. Risiko}
\strut\end{minipage} &
\begin{minipage}[t]{0.74\columnwidth}\raggedright\strut
Sehr hoch
\strut\end{minipage}\tabularnewline
\begin{minipage}[t]{0.20\columnwidth}\raggedright\strut
\textbf{Business Value}
\strut\end{minipage} &
\begin{minipage}[t]{0.74\columnwidth}\raggedright\strut
Mittel
\strut\end{minipage}\tabularnewline
\bottomrule
\end{longtable}

\subsection{NFREQ-135 Hohe
Verfügbarkeit}\label{nfreq-135-hohe-verfuxfcgbarkeit}

\begin{longtable}[c]{@{}ll@{}}
\toprule
\begin{minipage}[t]{0.20\columnwidth}\raggedright\strut
\textbf{UC-Referenz}
\strut\end{minipage} &
\begin{minipage}[t]{0.74\columnwidth}\raggedright\strut
UC-25, UC-21, UC22
\strut\end{minipage}\tabularnewline
\begin{minipage}[t]{0.20\columnwidth}\raggedright\strut
\textbf{Beschreibung}
\strut\end{minipage} &
\begin{minipage}[t]{0.74\columnwidth}\raggedright\strut
Der Authentifizierungsschnittstellen-Service soll eine hohe
Verfügbarkeit von 99.9\% haben.
\strut\end{minipage}\tabularnewline
\begin{minipage}[t]{0.20\columnwidth}\raggedright\strut
\textbf{Techn. Risiko}
\strut\end{minipage} &
\begin{minipage}[t]{0.74\columnwidth}\raggedright\strut
Hoch
\strut\end{minipage}\tabularnewline
\begin{minipage}[t]{0.20\columnwidth}\raggedright\strut
\textbf{Business Value}
\strut\end{minipage} &
\begin{minipage}[t]{0.74\columnwidth}\raggedright\strut
Mittel
\strut\end{minipage}\tabularnewline
\bottomrule
\end{longtable}

\subsection{NFREQ-210 Programmierer kann aus Vielzahl von verschiedenen
Sicherheitsstufen
auswählen}\label{nfreq-210-programmierer-kann-aus-vielzahl-von-verschiedenen-sicherheitsstufen-auswuxe4hlen}

\begin{longtable}[c]{@{}ll@{}}
\toprule
\begin{minipage}[t]{0.20\columnwidth}\raggedright\strut
\textbf{UC-Referenz}
\strut\end{minipage} &
\begin{minipage}[t]{0.74\columnwidth}\raggedright\strut
UC-25, UC-21, UC22
\strut\end{minipage}\tabularnewline
\begin{minipage}[t]{0.20\columnwidth}\raggedright\strut
\textbf{Beschreibung}
\strut\end{minipage} &
\begin{minipage}[t]{0.74\columnwidth}\raggedright\strut
Dem Programmierer stehen verschiedene Sicherheitsstufen zur Verfügung.
Das Wort ``verschieden'' wurde durch folgende Aspekte mit dem
Auftraggeber definiert: Abgeleitet von der Aufgabenstellung sind Aspekte
``Mehrfachteilnahme'' und ``Automatisierung'' definiert worden. Beide
Aspekte können durch eine Sicherheitsstufe mehr oder weniger verhindern
werden. Abhängig von der Art der Interaktivität ist es wirtschaftlich
sinnvoll, dass Kosten entstehen dürfen. Deshalb sind die Höhe der Kosten
ein Aspekt. Ein weiterer Aspekt ist der Aufwand für den Benutzer.
\strut\end{minipage}\tabularnewline
\begin{minipage}[t]{0.20\columnwidth}\raggedright\strut
\textbf{Techn. Risiko}
\strut\end{minipage} &
\begin{minipage}[t]{0.74\columnwidth}\raggedright\strut
Niedrig
\strut\end{minipage}\tabularnewline
\begin{minipage}[t]{0.20\columnwidth}\raggedright\strut
\textbf{Business Value}
\strut\end{minipage} &
\begin{minipage}[t]{0.74\columnwidth}\raggedright\strut
Hoch
\strut\end{minipage}\tabularnewline
\bottomrule
\end{longtable}

\subsection{NFREQ-212 Die verwendeten Sicherheitsstufen sind in der
Schweiz
verbreitet}\label{nfreq-212-die-verwendeten-sicherheitsstufen-sind-in-der-schweiz-verbreitet}

\begin{longtable}[c]{@{}ll@{}}
\toprule
\begin{minipage}[t]{0.20\columnwidth}\raggedright\strut
\textbf{UC-Referenz}
\strut\end{minipage} &
\begin{minipage}[t]{0.74\columnwidth}\raggedright\strut
UC-25, UC-21, UC22
\strut\end{minipage}\tabularnewline
\begin{minipage}[t]{0.20\columnwidth}\raggedright\strut
\textbf{Beschreibung}
\strut\end{minipage} &
\begin{minipage}[t]{0.74\columnwidth}\raggedright\strut
Die eingesetzten Sicherheitsstufen sollten in der Schweiz verbreitet
sein.
\strut\end{minipage}\tabularnewline
\begin{minipage}[t]{0.20\columnwidth}\raggedright\strut
\textbf{Techn. Risiko}
\strut\end{minipage} &
\begin{minipage}[t]{0.74\columnwidth}\raggedright\strut
Niedrig
\strut\end{minipage}\tabularnewline
\begin{minipage}[t]{0.20\columnwidth}\raggedright\strut
\textbf{Business Value}
\strut\end{minipage} &
\begin{minipage}[t]{0.74\columnwidth}\raggedright\strut
Hoch
\strut\end{minipage}\tabularnewline
\bottomrule
\end{longtable}

\newpage

\newpage

\section{Risiken}\label{risiken}

Nachfolgend sind die im Gespräch mit dem Auftraggeber gefundenen Risiken
bezüglich der Bachelorarbeit sowie deren Auswirkungen, aufgeführt.

\subsection{R-01 Akzeptanz}\label{r-01-akzeptanz}

Programmierer und insbesondere auch User, welche den
Authentifizierungsschnittstellen-Service verwenden sollen, sind völlig
unterschiedlich. Deren unterschiedlichen Ansprüche machen es schwierig,
eine Lösung zu entwickeln, welchen den Akteuren gerecht wird.

Da der Auftraggeber sowohl die Zielgruppe Programmierer wie auch User
kennt, kann er hier gezielt Feedback geben.

Die Auswirkung bei Eintritt dieses Risikos ist im Rahmen der
Bachelorarbeit gering, da der Erfolg der Arbeit nicht von der
tatsächlichen Verwendung im produktiven Umfeld abhängt.

\subsection{R-02 Kosten}\label{r-02-kosten}

Da es sich bei diesem Projekt um eine Bachelorarbeit handelt, besteht
kein Personalkostenrisiko. Kostenpflichtige Produkte Dritter werden
nicht verwendet. Einzig der Betrieb/ das Hosting der Bachelorarbeit
verursacht Kosten. Das Kostenrisiko kann dank fixen Leistungsparametern
auf ein Minimum reduziert werden.

\subsection{R-03 Überkomplexität}\label{r-03-uxfcberkomplexituxe4t}

Es besteht die Möglichkeit, dass die Komplexität des zu entwickelnden
Systems viel höher ist, als angenommen. Da die Komplexität nur zu einem
gewissen Grad durch Architekturentscheide beeinflusst werden kann, muss
ein besonderes Augenmerk auf dieses Risiko gelegt werden.

Dieses Risiko wird mit hoher Wahrscheinlichkeit eintreten.

Die Auswirkung bei Eintritt dieses Risikos ist, dass nicht der gesamte
Umfang der Bachelorarbeit erarbeitet werden kann, weil zur Lösung der
Komplexitätsprobleme zusätzliche Zeit benötigt wird.

\subsection{R-04
Systemumfeldänderungen}\label{r-04-systemumfelduxe4nderungen}

Umsysteme könnten während der Projektphase dieser Bachelorarbeit
massgeblich verändert werden.

Dieses Risiko wird mit sehr geringer Wahrscheinlichkeit eintreten.

Die Auswirkung bei Eintritt dieses Risikos kann nicht abgeschätzt
werden. Situativ muss dieses Risiko behandelt werden.

\subsection{R-05 Schlechte/Unzureichende
Framework}\label{r-05-schlechteunzureichende-framework}

Die Bachelorarbeit wird basierend auf verschiedenen Frameworks
umgesetzt. Das Risiko, dass Frameworks nicht wie beschrieben
funktionieren, schlecht dokumtiert oder instabil sind besteht.

Dieses Risiko wird mit mittlerer Wahrscheinlichkeit eintreten Als
Auswirkungen dieses Risikos sind Wechsel des Frameworks oder gar
manuelle Entwicklungen und daraus zusätzlicher, nicht einschätzbarer
Aufwand nötig.

\subsection{R-06 Termineinhaltung}\label{r-06-termineinhaltung}

Den fixe Abgabetermin der Semesterarbeit gilt es einzuhalten. Das
Risiko, dass die Arbeit verspätet abgegeben wird besteht.

Dieses Risiko wird mit geringer Wahrscheinlichkeit eintreten. Die
Auswirkung bei Eintritt dieses Risikos ist das Nichtbestehen der Arbeit.

\subsection{R-07 Auslastung}\label{r-07-auslastung}

Das Projekt wird durch einen Mitarbeiter getragen. Dieser ist sowohl im
Beruf wie auch privat stark ausgelastet. Der hohe schulische Aufwand
kann beeinflusst werden. Mit zusätzlichen Ausfällen durch Krankheit oder
nicht vorhersehbare Vorfällen muss gerechnet werden.

Das Risiko wird mit mittlerer Wahrscheinlichkeit eintreten. Die
Auswirkungen bei Eintritt dieses Risikos werden sich in der Qualität und
Quantität der Arbeit widerspiegeln.

\newpage

\subsection{Risikomatrix}\label{risikomatrix}

\begin{figure}[htbp]
\centering
\includegraphics{images/excel-statistik/risikomatrix.JPG}
\caption[Risikomatrix ]{Risikomatrix \footnotemark{}}
\end{figure}
\footnotetext{Die Risikomatrix wurde basierend auf der Excel-Vorlage der
  Stadtpolizei Zürich Abteilung Informatik entworfen.}

\textbf{Legende}

R1 Akzeptanz

R2 Kosten

R3 Überkomplexität

R4 Systemumfeldänderungen

R5 Schlechte/Unzureichende Frameworks

R6 Termineinhaltung

R7 Auslastung

\subsection{Massnahmen}\label{massnahmen}

Um das Zusammenspiel der verschiedenen Technologien und die daraus
resultierende Komplexität einschätzen zu können, wird vor Projektbeginn
ein Prototyp mittels Durchstich durch alle Technologien erstellt. Danach
kann die Komplexität im Zusammenspiel der Technologie eingeschätzt und
bei Bedarf eine Technologie durch eine andere ersetzt werden. So kann
das Risiko 3 ``Überkomplexität'' und Risiko 5 ``Schlechte/Unzureichende
Frameworks'' minimiert werden.

Das Projekt ist über eine Anzahl von Feiertagen gelegt, welche gebraucht
werden könnten. Zusätzlich wurde vom Studenten eine Arbeitswoche Ferien
genommen, welche im Notfall auch für die Arbeit verwendet werden könnte.
Durch diese Massnahmen sollte das Risiko 6 Termineinhaltung minimal
bleiben. Das Risiko 7 ``Auslastung'' kann nicht direkt vermindert
werden. Die Aktivitäten im Bereich der freiwilligen Arbeit wurde auf ein
Minimum reduziert. Für die restliche freiwilige Arbeit wurde mit
Freunden ein Notfallszenario entwickelt, so kann der Student bei Bedarf
seine freiwillige Arbeit durch andere Personen übernehmen lassen kann.
Der Kontakt mit dem Arbeitgeber wird intensiv gepflegt um bei Bedarf die
Arbeitsbelastung zu vermindern. Die Massnahmen welche für Risiko 6
ergriffen wurden entschärfen auch Risiko 7.

\chapter{Konzept}\label{konzept}

In diesem Kapitel soll ein System für den Authentifizierungsservice
entworfen werden. Das System soll den Anforderungen, welche im
vorherigen Kapitel definiert wurden, entsprechen.

Um die Komponenten unabhängig von einander zu entwickeln, wird bei der
Entwicklung der Architektur des Authtenifizierungsservice darauf geachte
möglichst gerine Kopplung aufzuweisen.

\section{Systemarchitektur}\label{systemarchitektur}

Gemäss den nichtfunktionalen Anforderungen muss die Serversoftware -
unter anderem - folgende Eigenschaften erfüllen:

\begin{itemize}
\tightlist
\item
  Hohe Verfügbarkeit von 99.9\%
\item
  Wartbarkeit
\item
  Performance
\end{itemize}

Die Softwarearchitektur wurde im Hinblick auf diese Anforderungen
erstellt.

\newpage

\section{Architekturübersicht}\label{architekturuxfcbersicht}

Der Authtentifizierungsservice besteht aus drei Hauptkomponenten:
Web-API, Konfigurator und Autorisierung. Die folgende Abbildung zeigt
die Verbindungen der drei Hauptkomponenten im Systemkontext des
Authentifizierungservice auf.

\begin{figure}[htbp]
\centering
\includegraphics{images/draw_io/BA_KomponentenDiagramm.png}
\caption{Übersicht der Hauptkomponenten}
\end{figure}

\newpage

\section{Genereller Ablauf
Authentifizierung}\label{genereller-ablauf-authentifizierung}

Der User nimmt an einer Interaktivität eines Anbieters teil. Dabei füllt
er den Wettbewerb, Umfrage aus oder löst die gegebene Aufgabe und sendet
einmal oder mehrmals ein Feedback an die Anbieter Webseite zurück. Nach
Abschluss der Interaktivität, werden die Datengespeichert und mit der
daraus resultierenden eindeutigen Identität des Feedbacks wird die
Authentifizierung gestartet. Das vom Programmierer definierte
Authentifizierungsverfahren bestehend aus ein oder mehreren
Sicherheitsstufen wird durchgeführt um Identität im gewünschten Masse
sicher zustellen. User und das Anbieter System werden über erfolgreiche
Authentifizierung informiert. Nach Möglichkeit wird auch eine
fehlerhafte Authentifizierung mitgeteilt.

\begin{figure}[htbp]
\centering
\includegraphics{images/draw_io/BA_AutorisationOverview.png}
\caption{Aufbau Inhalt im Card-Design}
\end{figure}

\section{Domänenmodel
Differenziert}\label{domuxe4nenmodel-differenziert}

Ein differenziertes Domänemodel oder auch Domänenmodel Basis Level
genannt, erlaubt eine vereinfachte Kommunikation zwischen
Kunde/Auftraggeber und Entwicklungsteam/Entwicklungsperson. Die
Denkweise im Model erfordert keine Programmierkenntnisse und fördert die
strukturierte Wiedergabe von Datengefässen. Beim Domänenmodel werden die
Begriffe aus der Domäne des Kunden verwendet und fördern so die
Verständlichkeit auf beiden Seiten.

\begin{figure}[htbp]
\centering
\includegraphics{images/domaenenmodell.png}
\caption{Differenziertes Domänemodel des Authentifizierungservice}
\end{figure}

\newpage

\section{Datenbankdesign}\label{datenbankdesign}

In der Systemarchitektur des Authentifizierungservice stehen Objekte nur
während der Ausführungszeit zur Verfügung. Um sie zu persitieren, werden
sie in einer relationalen Datenbank gespeichert. Die Pradigmen der
Objektorientierten Programmiersprache und der relationalen Datenbank
sind grundlegend verschieden. So kapseln Objekte ihren Zustand und ihr
Verhalten hinter einer Schnittstelle und haben eine eindeutige
Identität. Relationale Datenbanken basieren dagegen auf dem
mathematischen Konzept der relationalen Algebra. Dieser konzeptionelle
Widerspruch wurde in den 1990er Jahren als ``object-relational impedance
mismatch'' bekannt.\footnote{\autocite{the-vietnam-of-computer-science}}
Um diesen Wiederspruch zu mindern stellt Microsoft das Entity-Framework
zur Verfügung.

\subsection{Entity-Framework}\label{entity-framework}

Das Entity-Framework hat verschiedene Konzeptionelle Ansätze um
möglichst viele Bedürfnisse an den ORM-Mapper zu erfüllen. Es gilt nun
den richtigen Ansatz für den Authenifizierungsservice zu wählen.

\subsubsection{Database First}\label{database-first}

Beim Database First Ansatz wird zuerst die Datenbank designt. Das
Entity-Framework bildet aus der Datenbank die POCO-Klassen\footnote{Eine
  POCO-Klasse ist ein ganz ``einfaches'' .NET-Objekt. Damit ist es
  geeignet schlank Daten zu transportieren. Weitere Informationen im
  \protect\hyperlink{glossar}{Glossar}} ab. Sollten Anpassungen an den
Entitäten ergeben, werden diese zuerst in der Datenbank implementiert
und daraus werden wiederum neuen POCO-Klassen generiert.

\subsubsection{Code First}\label{code-first}

Beim Code First Ansatz werden zuerst die POCO-Klassen erstellt. Das
Entity-Framework bildet aus den POCO-Klassen die Tabellen in der
Datenbank. Alle Anpassungen werden gleich in den POCO-Klassen umgesetzt
und durch das Entity-Framework in der Datenbank geändert erstellt.

\subsubsection{Entscheidung}\label{entscheidung}

Wenn die POCO-Klassen gleich mehrheitlich für die
Schnittstellendefinition als Parameterdefinition verwendet werden
könnten, fallen Mehraufwendungen für Umwandlungen im Programmcode weg.
Eine Schnittstellendefinition sollte aber nicht willkürlich durch eine
Datenbankänderung beeinflusst werden. Der umgekehrte Fall ist aber
minder wichtig, da die Datenbank nur von der Schnittstelle verwendet
wird. Deshalb wird das Konzept Code First eingesetzt.

\subsection{ERD}\label{erd}

Durch den Codefirst Ansatz werden die Datenbank und alle zugehörigen
Tabellen durch das Entity Framework selbständig generiert

\newpage

\section{Integration der
Schnittstelle}\label{integration-der-schnittstelle}

Wie in der Anforderungsanalyse und Aufgabenstellung geschrieben, soll
die Schnittstelle möglichst einfach in Bestehende Systeme integriert
werden können. Bevor wir untersuchen wie wir die Integration umsetzten
können, bedarf es die wichtigsten bestehenden Systeme zu kennen um evtl
für diese Systeme eine spezifisch einfach Integration zu entwickeln.

\subsection{Bestehende Systeme für Votings, Wettbewerbe und
Quizes}\label{bestehende-systeme-fuxfcr-votings-wettbewerbe-und-quizes}

Das bestehende Social-Media Modul wird als Teil einer Webseite in einer
Webapplikation geführt. Webapplikation, welche Inhalte verwalten, werden
sinngemäss Content Management Systeme genannt. Die Abkürzung CMS hat
sich im IT-Fachjargon etabliert. Statista.com wertetet mehrmals im Jahr
die Verbreitung der verschiedenen CMS aus \footnote{CMS
  Nutzungsstatistik von statista.com \autocite{statisticinfostatista}}.
Folgend ist die Erhebung aus dem November 2015 abgebildet:

\begin{figure}[htbp]
\centering
\includegraphics{images/cms_statistik_november2015.JPG}
\caption{Nutzungsanteil CMS weltweit \emph{Quelle:de.statista.com}}
\end{figure}

Die von statista.com veröffentlichten Zahlen wurden mit Werten von
W3techs.com verglichen\footnote{CMS Nutzungsstatistik von w3techs.com
  \autocite{statisticinfow3techs}}. Die Unterschiede sind für unsere
Verwendung minimal und liegen im 10tels Prozentbereich. Da beide
bekannten Statistik unternehmen auf die selben Werte gekommen sind, kann
von einem hohen Warheitsgrad ausgegangen werden. Beim Betrachten der
Statistik fällt auf das Wordpress mit 25,2 mit Abstand am meisten
genutzt wird. Alle dynamischen Webseiten unter den Top 10 basieren auf
Systemen in PHP\footnote{Die Information wurde von den jeweiligen
  Hersteller- bzw. Communitywebseiten bezogen.}. Adobe Dreamviewer und
FrontPage sind keine Systeme welche auf dem Server betrieben werden. Sie
sind Editoren welche auf dem jeweiligen Computer ausgeführt werden und
danach mehrheitlich HTML, CSS und Javascript Code an den Server
ausliefern. Funktionalitäten werden mit den beiden Editoren manuell
geschrieben.

Basierend auf diesen statistischen Erkenntnissen lohnt es sich die
Wordpress Welt kennen zu lernen und recherchieren wie dort eine
Authentifizierungsschnittstelle eingebunden werden könnte.

\hypertarget{wordpress-plugin-hook}{\subsection{Wordpress PlugIn
Hook}\label{wordpress-plugin-hook}}

Erweiterungen im Wordpress nennen sich Plugins. Die Plugins können
direkt über das CMS-Backend eingespielt werden. Alternativ können Sie
natürlich manuell installiert werden. Zum Beispiel in dem man ein Plugin
selber Programmiert oder beim Hersteller oder über das
Plugin-Verzeichnis von Wordpress{[}\^{}plugin-verzeichnis{]}
downloadedt. Wordpress sammelt zugleich die aktiven Installationen der
PlugIns (sofern man als Entwickler den Informationsaustausch nicht
unterbindet). Die Gesamtzahl wird im CMS-Backend Wordpress und auf Ihrer
Plugin-Verzeichnis Webseite{[}\^{}plugin-verzeichnis{]} veröffentlicht.
Dank dieser Kennzahl kann nun die meist verbreiteten Plugins
herrausgefunden werden.

Wordpress basiert auf einem sogennanten Hook-System. ``Hook'' eins zu
eins übersetzt bedeutet ``Haken'', ``Aufhänger'' oder ``Greifer''. Ein
Hook ist im Wordpress eine definierte Codestelle bei der man seinen
eigenen Code einhaken kann. Der PlugIn Entwickler definiert diese Hooks
um anderen PlugIns oder Funktionalitäten zu erlauben sein PlugIn zu
erweitern. Auch der Core vom Wordpress enthält solche Hooks. Dadurch
soll verhindert werden, dass PlugIn's oder der Core von Wordpress direkt
umgeschrieben werden muss und dann nicht mehr einfach so unabhängig
upgedatet werden kann. Um unsere Schnittstelle einzubinden, könnten wir
evtuell also solche Hooks verwenden. Dieser ``Hook''/Haken hat
lustigerweise auch einen Haken: Der PlugIn-Entwickler kann selbständig
bestimmen ob und wo er solche Hooks einsetzen will und welche
Möglichkeiten dann zur Verfügung stehen. Kommerzielle PlugIn's verfolgen
vielfach den Weg möglichst verschlossen zu agieren um mögliche
Erweiterungen monetär umzusetzen und so eine Abhängigkeit zu erzeugen.
Diese These gilt es nun zu untersuchen. Dafür wurden verschiedene Social
Plugin's ausgewählt. Die Top 1000 installierten Wordpress PlugIns welche
von der Art Social-Media Modul waren, ein paar Stichproben von
kommerziellen Plugins und Stichproben aus in Beiträgen empfohlenen
PlugIns: \footnote{Das Pluginverzeichnis befindet sich unter
  http://de.wordpress.org/plugins}, \footnote{Envato bietet eine
  Plattform für den Verkauf von Wordpress-Plugin's an
  http://market.envato.com}

\begin{longtable}[c]{@{}llll@{}}
\caption{Recherche PlugIn's}\tabularnewline
\toprule
\begin{minipage}[b]{0.24\columnwidth}\raggedright\strut
\textbf{PlugIn}
\strut\end{minipage} &
\begin{minipage}[b]{0.12\columnwidth}\raggedright\strut
\textbf{Kosten}
\strut\end{minipage} &
\begin{minipage}[b]{0.20\columnwidth}\raggedright\strut
\textbf{Installation}
\strut\end{minipage} &
\begin{minipage}[b]{0.33\columnwidth}\raggedright\strut
\textbf{Info zu Hooks}
\strut\end{minipage}\tabularnewline
\midrule
\endfirsthead
\toprule
\begin{minipage}[b]{0.24\columnwidth}\raggedright\strut
\textbf{PlugIn}
\strut\end{minipage} &
\begin{minipage}[b]{0.12\columnwidth}\raggedright\strut
\textbf{Kosten}
\strut\end{minipage} &
\begin{minipage}[b]{0.20\columnwidth}\raggedright\strut
\textbf{Installation}
\strut\end{minipage} &
\begin{minipage}[b]{0.33\columnwidth}\raggedright\strut
\textbf{Info zu Hooks}
\strut\end{minipage}\tabularnewline
\midrule
\endhead
\begin{minipage}[t]{0.24\columnwidth}\raggedright\strut
\textbf{WP-Polls}
\strut\end{minipage} &
\begin{minipage}[t]{0.12\columnwidth}\raggedright\strut
kostenlos
\strut\end{minipage} &
\begin{minipage}[t]{0.20\columnwidth}\raggedright\strut
100000+
\strut\end{minipage} &
\begin{minipage}[t]{0.33\columnwidth}\raggedright\strut
Über ``wp\_polls\_add\_poll'' könnte man den erstellten Poll
authentfizieren und bei fehlerhafter Authentifizierung löschen
\strut\end{minipage}\tabularnewline
\begin{minipage}[t]{0.24\columnwidth}\raggedright\strut
\textbf{Polldaddy Polls \& Ratings}
\strut\end{minipage} &
\begin{minipage}[t]{0.12\columnwidth}\raggedright\strut
Freemium
\strut\end{minipage} &
\begin{minipage}[t]{0.20\columnwidth}\raggedright\strut
20000+
\strut\end{minipage} &
\begin{minipage}[t]{0.33\columnwidth}\raggedright\strut
-
\strut\end{minipage}\tabularnewline
\begin{minipage}[t]{0.24\columnwidth}\raggedright\strut
\textbf{Wp-Pro-Quiz}
\strut\end{minipage} &
\begin{minipage}[t]{0.12\columnwidth}\raggedright\strut
kostenlos
\strut\end{minipage} &
\begin{minipage}[t]{0.20\columnwidth}\raggedright\strut
20000+
\strut\end{minipage} &
\begin{minipage}[t]{0.33\columnwidth}\raggedright\strut
Hooks vorhanden. Nicht für eine Authentifizierungsschnittstelle zu
gebrauchen.
\strut\end{minipage}\tabularnewline
\begin{minipage}[t]{0.24\columnwidth}\raggedright\strut
\textbf{Responsive Poll}
\strut\end{minipage} &
\begin{minipage}[t]{0.12\columnwidth}\raggedright\strut
15\$
\strut\end{minipage} &
\begin{minipage}[t]{0.20\columnwidth}\raggedright\strut
-
\strut\end{minipage} &
\begin{minipage}[t]{0.33\columnwidth}\raggedright\strut
Keine Hooks. Laut Hersteller sind welche geplant (Zeitpunkt ungewiss)
\strut\end{minipage}\tabularnewline
\begin{minipage}[t]{0.24\columnwidth}\raggedright\strut
\textbf{TotalPoll Pro}
\strut\end{minipage} &
\begin{minipage}[t]{0.12\columnwidth}\raggedright\strut
18\$
\strut\end{minipage} &
\begin{minipage}[t]{0.20\columnwidth}\raggedright\strut
-
\strut\end{minipage} &
\begin{minipage}[t]{0.33\columnwidth}\raggedright\strut
Hooks vorhanden. Ähnlich wie bei WP-Polls könnte man evtl. den
erstellten Datensatz löschen. Jedoch ist dies ohne Kauf nicht
ersichtlich.
\strut\end{minipage}\tabularnewline
\begin{minipage}[t]{0.24\columnwidth}\raggedright\strut
\textbf{Easy Polling}
\strut\end{minipage} &
\begin{minipage}[t]{0.12\columnwidth}\raggedright\strut
15\$
\strut\end{minipage} &
\begin{minipage}[t]{0.20\columnwidth}\raggedright\strut
-
\strut\end{minipage} &
\begin{minipage}[t]{0.33\columnwidth}\raggedright\strut
-
\strut\end{minipage}\tabularnewline
\begin{minipage}[t]{0.24\columnwidth}\raggedright\strut
\textbf{Opinion Stage}
\strut\end{minipage} &
\begin{minipage}[t]{0.12\columnwidth}\raggedright\strut
kostenlos
\strut\end{minipage} &
\begin{minipage}[t]{0.20\columnwidth}\raggedright\strut
10000+
\strut\end{minipage} &
\begin{minipage}[t]{0.33\columnwidth}\raggedright\strut
-
\strut\end{minipage}\tabularnewline
\begin{minipage}[t]{0.24\columnwidth}\raggedright\strut
\textbf{Wedgies}
\strut\end{minipage} &
\begin{minipage}[t]{0.12\columnwidth}\raggedright\strut
Freemium
\strut\end{minipage} &
\begin{minipage}[t]{0.20\columnwidth}\raggedright\strut
800+
\strut\end{minipage} &
\begin{minipage}[t]{0.33\columnwidth}\raggedright\strut
-
\strut\end{minipage}\tabularnewline
\bottomrule
\end{longtable}

Wir haben nun verschiedene Wordpress-Plugin's für Umfragen, Wettbewerbe
\& Abstimmungen auf Hooks untersucht. Alle PlugIn's bieten gar keinen
Hook an oder keinen Hook, welcher unseren Anforderungen einer einfachen
Integration genügt. Die aufgelisteten Plugins bilden eine wesentliche
Verbreitung ab. Selbst wenn wieder erwartet alle nicht untersuchten
Plugin's eine perfekte Hookanbindung liefern würden, hätten wir, mit den
nicht getesteten Plugin's eine zu geringe Verbreitung. Der Ansatz die
Integration per Hooks zu machen muss also fallen gelassen werden.

\newpage

\subsection{Parallellen im ähnliches
Anwendungsfeld}\label{parallellen-im-uxe4hnliches-anwendungsfeld}

Der vertieften Research der letzten Kapitel wird verlassen und es wird
probiert einen anderen Herangehensweise zur Findung der Lösung zu
nehmen: Forscher adaptieren immer wieder erfolgreiche Modelle aus
anderen Bereich in ihr Gebiet. Vielfach wird die Natur als erfolgreiches
Vorlagemodell genommen. Ganz soweit wird hier nicht gegangen.
Payment-Gateways wie der Schweizer Anbieter Datatrans müssen
Webshop-Entwicklern auch eine Möglichkeit bieten das Gateway einfach in
Ihren Webshop einbinden zu können. Auch bei Ihnen steht die Sicherheit
auf der obersten Stufe und eine einfache Integration ist für den Erfolg
trotz internationalem Druck von nöten. Dabei fährt Datatrans eine
Zweiwegstrategie. Sie stellen für bekannte Shopsysteme gleich ganze
PlugIns zur Verfügung\footnote{Übersicht der Web-Shop PlugIn's
  \autocite{datatrans-plugin}}. Auf der anderen Seite bieten Sie
ausführliche beschriebene und einfache Schnittstellen an.

\subsubsection{Datatrans Zahlungsablauf}\label{datatrans-zahlungsablauf}

Um die Gateway-Implementation der Datatrans als Ganzes zu verstehen,
führen wir uns der generellen Ablauf eines Payment Gateways eines
Webshopeinkaufs bei Datatrans vor Augen. Der Ablauf:

\begin{figure}[htbp]
\centering
\includegraphics{images/datatrans-autorisierung.JPG}
\caption{Nutzungsanteil Zahlungsablauf Webshop mit Datatrans
\emph{Quelle:datatrans}}
\end{figure}

\begin{enumerate}
\def\labelenumi{\arabic{enumi}.}
\tightlist
\item
  Der Endkunde wählt Produkt aus und schliesst die Bestellung ab
\item
  Der Webshop/Merchant zeigt Zahlungsseite von Datatrans, Karteninhaber
  gibt seine Kartendaten ein. 3.-7.Datatrans autorisiert und verarbeitet
  wennmöglich die Transaktion zum Acquirer.
\item
  Datatrans zeigt den Status dem Kunden an und sendet Status dem
  Merchant zurück.
\item
  Merchant zeigt dem Karteninhaber die Antwortseite (erfolgreich oder
  abgelehnt) \footnote{Für die Bachelorarbeit wurde die V 9.1.13
    verwendet \autocite{datatrans-api}}
\end{enumerate}

\newpage

\subsubsection{Datatrans XML/SOAP API Lightbox
Mode}\label{datatrans-xmlsoap-api-lightbox-mode}

Bei Schritt 2 des Zahlungsablaufs ruft der Webshop das Datatransgateway
auf. Beim ``Lightbox Mode'' wird dabei ein iframe in einem Overlay über
die Webseite gelegt und der Webshop ansich verdunkelt dargestellt.

\begin{figure}[htbp]
\centering
\includegraphics{images/datatrans-lightbox.JPG}
\caption{Datatrans Lightbox Integration \emph{Quelle:datatrans}}
\end{figure}

Das Gateway muss eine minimum an Informationen erhalten, um den
Zahlungsvorgang überhaupt starten zu können. So muss es wissen, wer der
Verchäufer ist. Datatrans regelt dies über eine Merchan-ID. Wie viel
Geld in welcher Währung verkauft werden sollte, muss Datatrans über
amount und currceny mitgeteilt werden. Um dem Shop später mitteilen zu
können, welche Bestellung erfolgreich verarbeitet wurde, braucht es eine
Referenznummer. Die Referenznummer nennt Datatrans singemäss refno. Die
Ganzen Parameter werden optional mit einem sign-Parameter gesichert und
mittels Html-Form dem Javascript übergeben:\footnote{Für die
  Bachelorarbeit wurde die V 9.1.13 verwendet \autocite{datatrans-api}}

\newpage

Implementierungscode der Datatrans:

\begin{verbatim}
<script src="https://code.jquery.com/jquery-1.11.2.min.js"></script>
<script src="https://pilot.datatrans.biz/upp/payment/js/datatrans-1.0.2.js"></script>

    <form id="paymentForm"
         data-merchant-id="1100004624"
         data-amount="1000"
         data-currency="CHF"
         data-refno="123456789"
         data-sign="30916165706580013">
         <button id="paymentButton">Pay</button>
    </form>
    
<script type="text/javascript">
     $("#paymentButton").click(function () {
        Datatrans.startPayment({'form': '#paymentForm'});
     });
</script>
\end{verbatim}

\subsection{Integrationsentscheid}\label{integrationsentscheid}

Die Stragtegie der Paymentintegration von Datatrans soll für den
Authentifizierungservice genutzt werden.

Durch automatisches Öffnen der Lightbox erreicht der Endbenutzer mühelos
den Schritt der Authentifizierung. Die Authentifzierung springt ihm nahe
zu entgegen. Dadurch ist eine Hohe Effiktivität gegeben. Der User bleibt
auf der selben Seite und wird dadurch nicht aus dem Fluss der
Abarbeitung der Interaktivität geworfen. Das Verfahren ist sehr
effizient. Die Javascript und CSS Daten werden beim Laden der
Interaktivität bereits mit geladen. So entsteht eine minimale Wartezeit
beim Einblenden der Lightbox. Dies ist für den User nicht spürbar oder
störend.

Bei der Darstellung der Authentifizierung auf einer einzelnen Seite
müsste das Web-Design des Interaktivitäs-Anbiter adaptiert werden
können. Da die Authentifizierungs-Lightbox auf seiner Seite dargestellt
wird, braucht der Interaktivitäts-Anbieter nicht sein Design mühsam für
eine Authentifizierungsseite zu konfigurieren.

Die Lightbox des Authenifizierungsservice wird mit einer grösseren
Verbreitung einen gewissen Wiedererkennungswert erhalten. So wird die
Lösung als professionelles Produkt wahrgenommen werden. Das Ziel das
Benutzer und Entwickler den Authenifizierungsservice als ein sicheres
und glaubwürdiges Produkt für Interaktivitäten wahrnehmen wird so
versteckt werden.

\newpage

\hypertarget{integrationskonzept}{\subsection{Integrationskonzept}\label{integrationskonzept}}

\subsubsection{Integrationsparameter}\label{integrationsparameter}

\begin{longtable}[c]{@{}lll@{}}
\caption{Parameter Authenifizierungsservice Lightbox}\tabularnewline
\toprule
\begin{minipage}[b]{0.30\columnwidth}\raggedright\strut
\textbf{Feldname}
\strut\end{minipage} &
\begin{minipage}[b]{0.17\columnwidth}\raggedright\strut
\textbf{Wert}
\strut\end{minipage} &
\begin{minipage}[b]{0.43\columnwidth}\raggedright\strut
\textbf{Beschreibung}
\strut\end{minipage}\tabularnewline
\midrule
\endfirsthead
\toprule
\begin{minipage}[b]{0.30\columnwidth}\raggedright\strut
\textbf{Feldname}
\strut\end{minipage} &
\begin{minipage}[b]{0.17\columnwidth}\raggedright\strut
\textbf{Wert}
\strut\end{minipage} &
\begin{minipage}[b]{0.43\columnwidth}\raggedright\strut
\textbf{Beschreibung}
\strut\end{minipage}\tabularnewline
\midrule
\endhead
\begin{minipage}[t]{0.30\columnwidth}\raggedright\strut
\textbf{projectId}
\strut\end{minipage} &
\begin{minipage}[t]{0.17\columnwidth}\raggedright\strut
Integer
\strut\end{minipage} &
\begin{minipage}[t]{0.43\columnwidth}\raggedright\strut
Project ID
\strut\end{minipage}\tabularnewline
\begin{minipage}[t]{0.30\columnwidth}\raggedright\strut
\textbf{providerId}
\strut\end{minipage} &
\begin{minipage}[t]{0.17\columnwidth}\raggedright\strut
String
\strut\end{minipage} &
\begin{minipage}[t]{0.43\columnwidth}\raggedright\strut
Die ID um die Interaktivität seitens Interaktionsanbieter eindeutig zu
erkennen
\strut\end{minipage}\tabularnewline
\begin{minipage}[t]{0.30\columnwidth}\raggedright\strut
\textbf{sign}
\strut\end{minipage} &
\begin{minipage}[t]{0.17\columnwidth}\raggedright\strut
String
\strut\end{minipage} &
\begin{minipage}[t]{0.43\columnwidth}\raggedright\strut
Signatur, welche die Eingaben überprüft.
\strut\end{minipage}\tabularnewline
\bottomrule
\end{longtable}

\subsubsection{Einfache Signature}\label{einfache-signature}

Die Verwendung einer einfachen Signatur beugt Eingabefehler vor. Wenn
auch nur geringfügig, der Aufwand erschwert zusätzlich den Missbrauch.Um
eine korrekte Signatur zu erstellen werden folgende Parameter
konkateniert und mit einem Plus separiert.

\begin{itemize}
\tightlist
\item
  projectId: Parameterfeld
\item
  providerId: Parameterfeld
\item
  validationCode: Beim Anlegen eines Projektes im Konfigurator des
  Authenifizierungsservice soll ein ValidationCode vom
  Authenifizierungsservice generiert werden und dem Programmierer zur
  Verfügung gestellt werden.
\end{itemize}

Beispiel: 30045+12+BUQHFMNZ4P3T8XNVN0LK

Der daraus resultierende String wird mit MD5 verschlüsselt.

Beim Beispiel gäbe es die Signatur b37b3d4cd7cd8cba3f409f07d6f6d9bd

\hypertarget{schlussspeicherung}{\subsection{Schlussspeicherung}\label{schlussspeicherung}}

Nach Abschluss der Authentifizierung erhält der User visualisiert ein
Feedback. Sofern die Authentifizierung erfolgreich war, wird im
Hintergrund die im Konfigurator angegebene Url des Anbieters aufgerufen.
Über die Post-Parameter ProjectId + ProviderID erfährt die
Serverapplikation um welchen Datensatz es sich handelt. Wiederum wird
der sign-Parameter zum Absichern mitgegeben. Die Gefahr besteht
trotzdem, dass diese Redirect-Url auch von einem anderen Programm
aufgerufen werden könnte. Deshalb kann als zweite Absicherung die
Serverapplikation des Anbieters die Validate WebAPI des
Authentifizierungsservice aufrufen und die erhaltenen Daten gegeprüfen.
Bei erfolgreicher Gegenprüfung gilt der Datensatz seitens Anbieter auch
als valide und kann dann persistiert werden.

\newpage

\section{Sicherheitstufen
integrieren}\label{sicherheitstufen-integrieren}

Im Kapitel Recherche wurden einige Sicherheitskomponenten recherchiert
und illustriert. Es gilt nun ein Setting an Komponenten zu finden,
welche dem Developer eine Breite Auswahlmölgichkeit
(\protect\hyperlink{ux5cux23ux5cux23NFREQ-210}{NFREQ-210}) bietet und
eine genügende Verbreitung in der Schweiz hat
(\protect\hyperlink{ux5cux23ux5cux23NFREQ-212}{NFREQ-212}), ihn aber
nicht durch komplexes Auswählen der Sicherheitstufen aufhaltet
(\protect\hyperlink{ux5cux23ux5cux23NFREQ-222}{NFREQ-222}),.

\subsubsection{Cookie}\label{cookie-1}

Durch Speicherung des Cookies soll ein Benutzer der bereits an einer
Interaktivität teilgenommen hat, identifiziert werden. Da die Cookies
clientseitig verwaltet werden, können diese auch vom Anwender
manipuliert werden. Mit Browser Makro Tools wie iMacro kann ganz einfach
ein Cookie gelöscht werden. Dadurch ist sowohl das Verhindern mehrfacher
Teilnahme als auch das verhindern einer automatisierten Teilnahme an
einer Interaktivität ungenügend geschützt. Vorteilhaft für die
Cookiemethode ist, dass der Benutzer keinen Aufwand betreiben muss und
es keine Kosten verursacht.

\subsubsection{IP-Adresse}\label{ip-adresse-1}

Durch Speicherung der IP-Adresse soll ein Benutzer der bereits an einer
Interaktivität teilgenommen hat, identifiziert werden. Eine IP Adressen
vertritt gegen Aussen alle Benutzer mit dem selben
``Internetanschluss''\href{Der\%20Begriff\%20Internetanschluss\%20ist\%20schwamig\%20eingesetzt.}{!internetanschluss}.
Dadurch könnte nur einmal pro Internetanschluss an einer Interaktivität
teilgenommen werden. Dass durch Wechseln des Proxys eine andere
IP-Adresse verwendet werden kann und dies auch ohne IT-Know How durch
Tools möglich ist, lässt sowohl Eindeutigkeit und Verhinderung von
Automatisierung als ungenügend bewerten. Die Methode kann als kostenlos
eingestuft werden und generiert beim Endbenutzer keinen Aufwand

\subsubsection{Browser Fingerprint}\label{browser-fingerprint}

Durch Generierung eine Browser Fingerprints (siehe Recherche)
identifiziert werden. Das Verfahren kann zu 94\% ein User
wiedererkennen. Dass Verwenden mehrerer Browser oder Geräte führt zu
verschiedenen Browser Fingerprints. iPhone taugt nicht für die Methode.
Deshalb muss Eindeutigkeit und Verhinderung von Automatisierung als
ungenügend bewertet werden. Die Methode kann als kostenlos eingestuft
werden und generiert beim Endbenutzer keinen Aufwand.

\subsubsection{SMS Authentifizierung}\label{sms-authentifizierung}

Der Benutzer gibt seine Mobilenummer ein. Durch versenden eines Codes
wird sichergestellt, dass dem Benutzer die Telefonnumer gehört. In der
Schweiz können maximal 5 Mobilenummern bei den Anbietern gekauft
werden.(Siehe Kapitel Recherche) Der Benutzer kann eindeutig anhand der
Mobilenummer erkannt werden. Die möglichen Mobilenummern pro User sind
beschränkt. Eine Automatisierung ist praktisch unmöglich. Die Kosten pro
SMS sind tragbar. Der Benutzer muss bei dieser Methode sein Handy bei
sich tragen und den Code übertragen.

\subsubsection{Telefon
Authentifizierung}\label{telefon-authentifizierung}

Der Benutzer gibt seine Telefonnumer ein. Der Benutzer swird
automatisiert angerufen und die Computerstimme liest ein Code vor,
welcher der Benutzer im Rückbestätigungsformular einträgt. Dadurch wird
sichergestellt, dass die Telefonnummer dem Benutzer gehört.
Mobilenummern sind wie vorhin erwähnt eingeschränkt. Festnetzanschlüsse
unterliegen einer finanziellen Hürde.

\subsubsection{Postversand
Authentifizierung}\label{postversand-authentifizierung}

Der Benutzer gibt seine Adresse ein. Um sicherzustellen, dass die
Adresse dem User gehört wird automatisiert ein Brief an die Adresse
gesendet. Da die Gefahr besteht dass falsch adressierte Briefe den
Empfänger trotzdem erreichen, deshalb wird Unique mit gut und nicht sehr
gut bewertet. Eine Automatisierung ist praktisch unmöglich. Die Kosten
pro Brief sind von allen aufgelisteten Methoden am höchsten. Der
Benutzer muss bei dieser Methode den Brief nach erhalten auf einer
Webseite quittieren

\subsection{Sicherheitstufen bewerten}\label{sicherheitstufen-bewerten}

Die Recherche der verschiedenen Sicherheitsstufen wurden dem
Auftraggeber vorgelegt. Beim Auftraggeber wurden die verschiedenen
Sicherheitsstufen intern besprochen und bewertet. Pro Sicherheitsstufen
wurde den vier definierten Aspekten und dem Musskriterium Verbreitung
eine Schweizer Schulnote vergeben.

\begin{longtable}[c]{@{}lllllc@{}}
\caption{Übersicht der Authentifizierungs Methoden}\tabularnewline
\toprule
\begin{minipage}[b]{0.19\columnwidth}\raggedright\strut
\textbf{Sicherheitsstufen}
\strut\end{minipage} &
\begin{minipage}[b]{0.13\columnwidth}\raggedright\strut
\textbf{Verhinderung Mehrfachteilnahme}
\strut\end{minipage} &
\begin{minipage}[b]{0.13\columnwidth}\raggedright\strut
\textbf{Automat- sierung}
\strut\end{minipage} &
\begin{minipage}[b]{0.11\columnwidth}\raggedright\strut
\textbf{Kosten}
\strut\end{minipage} &
\begin{minipage}[b]{0.13\columnwidth}\raggedright\strut
\textbf{Aufwand Benutzer}
\strut\end{minipage} &
\begin{minipage}[b]{0.13\columnwidth}\centering\strut
\textbf{Verbreitung in der Schweiz}
\strut\end{minipage}\tabularnewline
\midrule
\endfirsthead
\toprule
\begin{minipage}[b]{0.19\columnwidth}\raggedright\strut
\textbf{Sicherheitsstufen}
\strut\end{minipage} &
\begin{minipage}[b]{0.13\columnwidth}\raggedright\strut
\textbf{Verhinderung Mehrfachteilnahme}
\strut\end{minipage} &
\begin{minipage}[b]{0.13\columnwidth}\raggedright\strut
\textbf{Automat- sierung}
\strut\end{minipage} &
\begin{minipage}[b]{0.11\columnwidth}\raggedright\strut
\textbf{Kosten}
\strut\end{minipage} &
\begin{minipage}[b]{0.13\columnwidth}\raggedright\strut
\textbf{Aufwand Benutzer}
\strut\end{minipage} &
\begin{minipage}[b]{0.13\columnwidth}\centering\strut
\textbf{Verbreitung in der Schweiz}
\strut\end{minipage}\tabularnewline
\midrule
\endhead
\begin{minipage}[t]{0.19\columnwidth}\raggedright\strut
\textbf{Cookie}
\strut\end{minipage} &
\begin{minipage}[t]{0.13\columnwidth}\raggedright\strut
2.5
\strut\end{minipage} &
\begin{minipage}[t]{0.13\columnwidth}\raggedright\strut
2.5
\strut\end{minipage} &
\begin{minipage}[t]{0.11\columnwidth}\raggedright\strut
6
\strut\end{minipage} &
\begin{minipage}[t]{0.13\columnwidth}\raggedright\strut
6
\strut\end{minipage} &
\begin{minipage}[t]{0.13\columnwidth}\centering\strut
6
\strut\end{minipage}\tabularnewline
\begin{minipage}[t]{0.19\columnwidth}\raggedright\strut
\textbf{Flash-Cookie}
\strut\end{minipage} &
\begin{minipage}[t]{0.13\columnwidth}\raggedright\strut
2.5
\strut\end{minipage} &
\begin{minipage}[t]{0.13\columnwidth}\raggedright\strut
3
\strut\end{minipage} &
\begin{minipage}[t]{0.11\columnwidth}\raggedright\strut
6
\strut\end{minipage} &
\begin{minipage}[t]{0.13\columnwidth}\raggedright\strut
6
\strut\end{minipage} &
\begin{minipage}[t]{0.13\columnwidth}\centering\strut
6
\strut\end{minipage}\tabularnewline
\begin{minipage}[t]{0.19\columnwidth}\raggedright\strut
\textbf{IP}
\strut\end{minipage} &
\begin{minipage}[t]{0.13\columnwidth}\raggedright\strut
3
\strut\end{minipage} &
\begin{minipage}[t]{0.13\columnwidth}\raggedright\strut
3
\strut\end{minipage} &
\begin{minipage}[t]{0.11\columnwidth}\raggedright\strut
6
\strut\end{minipage} &
\begin{minipage}[t]{0.13\columnwidth}\raggedright\strut
6
\strut\end{minipage} &
\begin{minipage}[t]{0.13\columnwidth}\centering\strut
6
\strut\end{minipage}\tabularnewline
\begin{minipage}[t]{0.19\columnwidth}\raggedright\strut
\textbf{Browser Fingerprint}
\strut\end{minipage} &
\begin{minipage}[t]{0.13\columnwidth}\raggedright\strut
3.5
\strut\end{minipage} &
\begin{minipage}[t]{0.13\columnwidth}\raggedright\strut
3.5
\strut\end{minipage} &
\begin{minipage}[t]{0.11\columnwidth}\raggedright\strut
6
\strut\end{minipage} &
\begin{minipage}[t]{0.13\columnwidth}\raggedright\strut
6
\strut\end{minipage} &
\begin{minipage}[t]{0.13\columnwidth}\centering\strut
6
\strut\end{minipage}\tabularnewline
\begin{minipage}[t]{0.19\columnwidth}\raggedright\strut
\textbf{E-Mail}
\strut\end{minipage} &
\begin{minipage}[t]{0.13\columnwidth}\raggedright\strut
4
\strut\end{minipage} &
\begin{minipage}[t]{0.13\columnwidth}\raggedright\strut
4.5
\strut\end{minipage} &
\begin{minipage}[t]{0.11\columnwidth}\raggedright\strut
6
\strut\end{minipage} &
\begin{minipage}[t]{0.13\columnwidth}\raggedright\strut
4.5
\strut\end{minipage} &
\begin{minipage}[t]{0.13\columnwidth}\centering\strut
6
\strut\end{minipage}\tabularnewline
\begin{minipage}[t]{0.19\columnwidth}\raggedright\strut
\textbf{SMS}
\strut\end{minipage} &
\begin{minipage}[t]{0.13\columnwidth}\raggedright\strut
5.5
\strut\end{minipage} &
\begin{minipage}[t]{0.13\columnwidth}\raggedright\strut
5.75
\strut\end{minipage} &
\begin{minipage}[t]{0.11\columnwidth}\raggedright\strut
5
\strut\end{minipage} &
\begin{minipage}[t]{0.13\columnwidth}\raggedright\strut
4.5
\strut\end{minipage} &
\begin{minipage}[t]{0.13\columnwidth}\centering\strut
5.5
\strut\end{minipage}\tabularnewline
\begin{minipage}[t]{0.19\columnwidth}\raggedright\strut
\textbf{Telefon}
\strut\end{minipage} &
\begin{minipage}[t]{0.13\columnwidth}\raggedright\strut
5.25
\strut\end{minipage} &
\begin{minipage}[t]{0.13\columnwidth}\raggedright\strut
5.75
\strut\end{minipage} &
\begin{minipage}[t]{0.11\columnwidth}\raggedright\strut
5
\strut\end{minipage} &
\begin{minipage}[t]{0.13\columnwidth}\raggedright\strut
4.5
\strut\end{minipage} &
\begin{minipage}[t]{0.13\columnwidth}\centering\strut
5.75
\strut\end{minipage}\tabularnewline
\begin{minipage}[t]{0.19\columnwidth}\raggedright\strut
\textbf{Ausweis- nummer}
\strut\end{minipage} &
\begin{minipage}[t]{0.13\columnwidth}\raggedright\strut
3.5
\strut\end{minipage} &
\begin{minipage}[t]{0.13\columnwidth}\raggedright\strut
3.5
\strut\end{minipage} &
\begin{minipage}[t]{0.11\columnwidth}\raggedright\strut
6
\strut\end{minipage} &
\begin{minipage}[t]{0.13\columnwidth}\raggedright\strut
5
\strut\end{minipage} &
\begin{minipage}[t]{0.13\columnwidth}\centering\strut
6
\strut\end{minipage}\tabularnewline
\begin{minipage}[t]{0.19\columnwidth}\raggedright\strut
\textbf{SuisseID}
\strut\end{minipage} &
\begin{minipage}[t]{0.13\columnwidth}\raggedright\strut
5.5
\strut\end{minipage} &
\begin{minipage}[t]{0.13\columnwidth}\raggedright\strut
5.75
\strut\end{minipage} &
\begin{minipage}[t]{0.11\columnwidth}\raggedright\strut
5
\strut\end{minipage} &
\begin{minipage}[t]{0.13\columnwidth}\raggedright\strut
5
\strut\end{minipage} &
\begin{minipage}[t]{0.13\columnwidth}\centering\strut
3
\strut\end{minipage}\tabularnewline
\bottomrule
\end{longtable}

\ldots{}Tabelle vervollständigen

\subsection{Auswahl der zu integrierenden
Sicherheitsstufen}\label{auswahl-der-zu-integrierenden-sicherheitsstufen}

SuisseID und Flash-Cookies erreichen beim Musskriterium Verbreitung in
der Schweiz (\href{FREQ-212}{NFREQ-212}) keine genügende Note und wird
daher ausgeschlossen. Um die geforderte Breite an Sicherheitsmethoden zu
erlangen wurden die folgenden Methoden mit verschiedenen Stärken in
Aspekten durch den Auftraggeber ausgewählt:

\begin{itemize}
\tightlist
\item
  IP-Adresse
\item
  E-Mail
\item
  SMS
\item
  Telefon
\item
  Ausweisnummer
\end{itemize}

\newpage

\hypertarget{modularituxe4t-und-erweiterbarkeit}{\section{Modularität
und Erweiterbarkeit}\label{modularituxe4t-und-erweiterbarkeit}}

Wie in der Einführung zur {[}Architektur{]} erwähnt, sollte eine
Architektur so konstruiert werden dass Sie möglichst Modular aufgebaut
ist. Auch wenn wir die zu verwendenden Authentifizierungsmethoden im
vorherigen Kapitel definiert haben, werden sich diese in Zukunft ändern.
Anderseits kann sich auch die Authentifizierungsmethoden an sich
komplett verändern. Sehr realistisch ist, dass für einen Browser
Fingerprint neue Berechnungsmethodiken bekannt werden. Der Anbieter der
hinter eine Authentifizierungsmethode steht, kann sich verändern oder
dessen Anbindung anpassen. Kurz gesagt, die Modularität der
Authentifizierungsmethoden muss unbedingt gewährleistet sein. Eine
Implementation der Sicherheitsstufe SMS wie im folgenden einfachen
Beispiel sollte nicht verwendet werden.

\begin{verbatim}
SMSSecurityStep inst = new SMSSecurityStep();
\end{verbatim}

\subsection{Design by Contract}\label{design-by-contract}

Das Design Pattern ``Design by Contract'' soll das Zusammenspiel von
Modulen durch eine Definition/Vertrag regeln. Herr Bertrand Meyer führte
das Pattern bei der Enwicklung der Programmiersprache Eiffel
ein.\footnote{\autocite{design-By-Contract}} Die Verträge enthält
besteht aus

\begin{itemize}
\tightlist
\item
  precondition: ``Die Zusicherung die der Aufrufer einzuhalten hat''
\item
  postcondition: ``Die Zusicherung die der Aufgegrufene einhalten wird''
\item
  Invariants: ``Invariants sorgen dafür,dass bei Eintritts- und
  Austrittspunkten des Server Codes gewisse Conditions erfüllt bzw.
  Zustände gewahrt sind. Invariants sind in gewisser Weise also Pre- und
  Postconditions.''
\end{itemize}

Im Grunde geht es darum den Operator new zu eliminieren.

Der Beispielcode als Design by Contract Pattern:

\begin{verbatim}
ISecurityStep proxy = new SomeFactory.GetSecurityStep(...);
\end{verbatim}

ISecurityStep-Vertrag ist im Beispielcode der Vertrag. Die Instanz proxy
Liefert ein Objekt zurück, welches das nach ISecurityStep-Vertrag
definiert ist. Welches Objekt (Implementierung) sich dahinter verbirgt,
ist uninteressant da diese Komponente gegen eine andere Implementierung
ausgetauscht werden kann. In diesem konkreten Fall, könnten
beispielsweise die Komponenten SMSSecurityStep und CookieSecurityStep
die Schnittstelle ISecurityStep implementieren.

SomeFactory muss für die Umsetzung jedoch implementiert werden. Dafür
gibt es in der .net Welt einiges an Beispiel Code und Frameworks zu
finden. Ein beliebtes Framework ist die Windows Communication
Foundation.\footnote{\autocite{design-By-Contract}}

\subsection{MEF - Managed Extensibility
Framework}\label{mef---managed-extensibility-framework}

MEF das Managed Extensibility Framework ist seit der Version 4.0
Bestandteil des .NET Frameworks. MEF ist eine Bibliothekt und
Implementiert das Problem der Erweiterbarkeit sogar zur Laufzeit. Es
vereinfacht die Implementierung von erweiterbaren Anwendungen und bietet
Ermittlung von Typen, Erzeugung von Instanzen und Composition
Fähigkeiten an.

\begin{figure}[htbp]
\centering
\includegraphics{images/mef_architektur.jpg}
\caption{Vereinfachte Architektur des Managed Extensibility Framework
Quelle: msdn.microsoft.com}
\end{figure}

Die Abbilldung zeigt eine stark vereinfachte Architektur von MEF auf.
Die Hauptmodule vom MEF-Core sind Catalog und CompositionContainer. Der
Catalog kontrolliert und stellt das Laden der Komponenten sicher. Der
CompositionContainer erzeugt aus den Komponenten Instanzen und bindet
diese an die entsprechenden Variablen. Parts sind die Objekte die vom
Type Export oder Import sein können. Die Komponenten die geladen und
instanziert sind nennen sich Exports. Imports sind die Variabeln an den
Exports gebunden werden sollen.

Um das Konzept besser zu verstehen, soll der Beispielcode von Design by
Contract herangezogen werden: In einer MEF Anwendung wäre die Variable
proxy vom Type ISecurityStep und die Instanz dieser Komponente wäre ein
„Import``. Die Objekte der SMSSecurityStep oder CookieSecurityStep wären
in einer MEF Anwendung ein Export.

MEF automatisiert die Instanzierung mit Hilfe von Catalog und Container.

\subsection{Entscheidung}\label{entscheidung-1}

Der Ansatz der Umsetzung des Design by Contract bräuchte eine geeignete
Integration für die Factory um die Modularität für {[}NFREQ-115{]}
sicherzustellen . MEF stellt den vollen Umfang an Funktionalität, zur
Lösung der Problematik, zu Verfügung. MEF bietet des weiteren die
Möglichkeit die DLL's zur Laufzeit auszutauschen und eine automatisierte
Instanzierung. Deshalb sind die Sicherheitsstufen des
Authenifizierungsservice basierend auf MEF zu integrieren.

\newpage

\subsection{Sicherheitsstufen Libaray-Übersicht anhand
MEF}\label{sicherheitsstufen-libaray-uxfcbersicht-anhand-mef}

Basierend auf dem Managed Extensibility Framework wird wir der Aufbau
unstrukturiert. Neu wird nicht alles in einer Library im Webservice
gespeichert sondern mehrere Libarays erstellt. Die Libaray
SecurityStepContracts beinhaltet den Contract/Vertrag der
Sicherheitsstufen ISecurityStep. Es besteht keine Abhängigkeit zwischen
dem Authenifizierungsservice und den Sicherheitsstufen.

\begin{figure}[htbp]
\centering
\includegraphics{images/mef_library_overview.png}
\caption{UML Library Overview}
\end{figure}

\newpage

\hypertarget{mockup}{\section{Mockup}\label{mockup}}

Ein Mockup ist eine grobe Vorlage für die Design-Umsetzung. Es ist eine
ideale Möglichkeit das visuelle Konzept ab zu bilden und mit dem
Auftraggeber vorgängig anzuschauen. Die folgenden Unterkapitel bilden
die Mockups der App ab.

\hypertarget{konfigurator-template}{\subsection{Konfigurator
Template}\label{konfigurator-template}}

Der Konfigurator soll den Programmierer visuell beim Konfigurieren und
Verwalten seiner Authentifizierungssoftware unterstützen. Bei der
Zielgruppe handelt es sich um Programmierer. Es kann deshalb von einem
hohen Know-How ausgegangen werden. Die Oberfläche soll möglichst
effizient gestaltet sein. Die Designelemente sollen deshalb klar und
einheitlich gestaltet werden. Generell ist davon auszugehen, dass der
Programmierer beim Einrichten seines Projektes am Desktop arbeitet. Für
Auswertungen und Präsentationen kann der Programmierer durchaus auch
mobile Endgeräte verwenden. Deshalb soll die Umsetzung responsive
gestaltet werden.

\begin{figure}[htbp]
\centering
\includegraphics{images/mockups/General.png}
\caption{Mockup Konfigurator Template Desktop}
\end{figure}

\newpage

\begin{figure}[htbp]
\centering
\includegraphics{images/mockups/Mobile.png}
\caption{Mockup Konfigurator Template Mobile}
\end{figure}

\subsubsection{Seitenaufbau}\label{seitenaufbau}

Im Header wird der Programmierer anhand des Seitentitels gleich über
seinen aktuellen Standort orientert.

\subsubsection{Navigation}\label{navigation}

Im Designkonzept wurde von einer Klappmenü oder Topnavigation abgesehen.
Die Wichtigkeit durch einen Klick alle Navigationspunkte zuerreichen,
überwiegte den Platzersparnissen in der Breite. Die wenigen
Navigationspunkte erlauben eine flache Navigationsstruktur. Dadurch kann
in der Desktopansicht links immer alle Navigationspunkte angezeigt
werden. Der Programmierer kann rasch auf die gewünsche Seite switchen.
In der Mobileansicht kann durch einen einzigen Klick auf die
``Burger-Navigation'' das gesamte Menü eingefahren werden. Der
Entscheid, für eine statische linke Navigationsstruktur in der
Desktopansicht, wurde ausserdem bekräftigt durch den Wunsch des
Auftraggebers den Konfigurator gestalterisch mit Farb und Bild
aufzuwerten. Dies ist über die linke Spalte einheitlich und einfach
umsetzbar.

\newpage

\subsubsection{Inhaltaufbau}\label{inhaltaufbau}

Trotz unterschiedlichstem Inhalt (Text, Tabellen, Diagramme, Bilder und
Formulare) und Grösse soll eine einheitliche Struktur geschaffen werden.
Die Struktur soll es erlauben einerseits Übersichten wie Dashboards mit
verschiedenen Inhalten auf einer Seite abzubilden. Die selbe Struktur
soll aber auch für Seiten mit nur einem Inhaltselement wie Registration
oder Login-Seite verwendet werden können. Verschiedene Designe lösen
diese Problematik mit einem Karten-Konzept English genannt Card Based
Design. Dabei wird jedes Inhaltselement als ``Card'' dargestellt. Die
``Card'' hat einen klar abgerenzten Darstellungsbereich. Die Card ist in
Header und Content unterteilt. Im Header wird mittels Titel dem Anwender
kommuniziert, was für ein Inhalt im Breich Content der ``Card'' zu
erwarten ist. \footnote{Weitere Informationen und Beispiele auf
  webdesigner.com \autocite{card-based-design}}

\begin{figure}[htbp]
\centering
\includegraphics{images/mockups/card.jpg}
\caption{Aufbau Inhalt im Card-Design}
\end{figure}

\newpage

\hypertarget{authentifizierungs-lightbox-template}{\subsection{Authentifizierungs-Lightbox
Template}\label{authentifizierungs-lightbox-template}}

Die Authentifizierungs-Lightbox wird vom Endbenutzer verwendet. Der
Endbenutzer kann ein geringes technische Know-How aufweisen. Deshalb
muss das Design einen Bereich verfügbar machen, in welchem die zu
tätigenden Schritte erklärt werden können. Die Möglichkeiten und Anzahl
Schritte sollen auf ein Minimum gehalten werden. Im besten Fall kann der
User eine Eingabe machen und dies mit einem Button bestätigen. Damit der
Endbenutzer fokusiert bleibt soll, wie bei einer Lightbox üblich, der
Rest der Seite abgedunkelt werden.

\begin{figure}[htbp]
\centering
\includegraphics{images/mockups/authenticationlightbox.jpg}
\caption{Mockup Konfigurator Template Mobile}
\end{figure}

\subsection{Hinweis zur Zusammenarbeit mit dem
Auftraggeber}\label{hinweis-zur-zusammenarbeit-mit-dem-auftraggeber}

Die hier abgebildeten Mockups und weitere Ansichten sind das Ergebnis
aus den Absprache mit dem Auftraggeber. Sie sind vom Auftraggeber
abgenommen und zur Impelmentation freigegeben

\newpage

\section{\texorpdfstring{``Du'' oder ``Sie'' --
Ansprache}{Du oder Sie -- Ansprache}}\label{du-oder-sie-ansprache}

Die Definition ob der Benutzer im Userinterface mit Du oder Sie
angesprochen werden soll muss laut Jutta Beyer vorgängig klar geregelt
sein.\footnote{\autocite{dusieansprache}} Eine einheitliche
Kommunikation auf der Plattform ist unabdingbar.

\subsection[Bestehende Vorurteile]{\texorpdfstring{Bestehende
Vorurteile\footnote{\autocite{dusieansprache}}}{Bestehende Vorurteile}}\label{bestehende-vorurteiledusieansprache}

\begin{longtable}[c]{@{}lll@{}}
\caption{Auflistung von Vorurteilen}\tabularnewline
\toprule
\begin{minipage}[b]{0.13\columnwidth}\raggedright\strut
\textbf{Art}
\strut\end{minipage} &
\begin{minipage}[b]{0.39\columnwidth}\raggedright\strut
\textbf{Positiv}
\strut\end{minipage} &
\begin{minipage}[b]{0.40\columnwidth}\raggedright\strut
\textbf{Negativ}
\strut\end{minipage}\tabularnewline
\midrule
\endfirsthead
\toprule
\begin{minipage}[b]{0.13\columnwidth}\raggedright\strut
\textbf{Art}
\strut\end{minipage} &
\begin{minipage}[b]{0.39\columnwidth}\raggedright\strut
\textbf{Positiv}
\strut\end{minipage} &
\begin{minipage}[b]{0.40\columnwidth}\raggedright\strut
\textbf{Negativ}
\strut\end{minipage}\tabularnewline
\midrule
\endhead
\begin{minipage}[t]{0.13\columnwidth}\raggedright\strut
\textbf{Du}
\strut\end{minipage} &
\begin{minipage}[t]{0.39\columnwidth}\raggedright\strut
Du steht für\\
- Vertrautheit\\
- Verbundenheit\\
- Modern\\
\strut\end{minipage} &
\begin{minipage}[t]{0.40\columnwidth}\raggedright\strut
Du steht aber auch für\\
- weniger Respekt\\
- weniger Kompetenz\\
- ``Stammtischniveau''\\
\strut\end{minipage}\tabularnewline
\begin{minipage}[t]{0.13\columnwidth}\raggedright\strut
\textbf{Sie}
\strut\end{minipage} &
\begin{minipage}[t]{0.39\columnwidth}\raggedright\strut
Sie steht für\\
- Respekt\\
- Kompetenz\\
- Seriösität\\
\strut\end{minipage} &
\begin{minipage}[t]{0.40\columnwidth}\raggedright\strut
Sie steht aber auch für\\
- Distanz\\
- Altmodische Eisntellungen\\
- Emotionslos\\
\strut\end{minipage}\tabularnewline
\bottomrule
\end{longtable}

Diese Wahrnehmungen sind nicht stichhaltig noch weniger können die
Rückschlüsse stimmen. Dennoch müssen diese Ansichten, zum Teil
entstanden aus Kultur und Tradition, ernstgenommen werden, da sie in
unseren Köpfen tief verankert sind. Kinder werden beigebracht das man
Fremde mit ``Sie'' an spricht. In der Familie, die Geborgenheit und
Vertrautheit ausstrahlt, ist das ``Du'' normal. Gegenüber Lehrern und
anderen Autoritätspersonen sollte das Kind aber ``Sie'' sagen. Also
spricht das Kind auch im fortgeschrittenen Alter Erwachsene, vor denen
es zugleich Respekt haben sollte, mit ``Sie'' an.

Eine generelle Antwort zur Verwendung Du oder Sie auf Webseiten kann
also nicht gemacht werden. Im Jahre 2011 wurden von statista Personen
aus Deutschland gefragt ``Wie möchten Sie in Social Media von
Unternehmen angesprochen werden?''. Dabei möchten 44\% der befragten per
Sie angesprochen. Einem grossen Teil (43\%) ist die Ansprache egal und
13\% würde sich eine Du-Ansprache bevorzugen.\footnote{\autocite{statistadusie}}

\newpage

\subsubsection{Entscheidung}\label{entscheidung-2}

Die Authentifizierung welche vom Endbenutzer durchgeführt wird darf
ruhig sprachlich distanziert und emotionslos wirken. Vielmehr sind
Respekt, Kompetenz und Seriösität wichtige Eckpunkte dieses Produkts.
Deshalb wird in der Authentifizierung der Endbenutzer, falls nötig, mit
Sie angesprochen. Der Konfigurator wird durch Programmierer
administriert. Hier gilt es zu vermitteln, dass der Programmierer sich
angenommen und unterstützt in seinem Problemen / Herausforderungen
fühlt. Das Produkt soll zeitgemäss und trendig sein. Deshalb wird die
Kommunikation über Du geführt. Dies Entscheidung wird durch die Annahme
unterstützt das unter Programmierern auch in der Wirtschaft mehrheitlich
geduzt wird.

\newpage

\section{Wahl des Applikation
Hosters}\label{wahl-des-applikation-hosters}

\subsection{Asp.net Shared Hosting}\label{asp.net-shared-hosting}

Ein Asp.net Shared Hosting ist durchaus für komplexere Webapplikationen
wie der Authentifizierungservice ausgerichtet. Die Kosten sind jährlich
fix und nicht abhängig von der eigentlichen Nutzung. Überschreitet die
Applikation den Speicherbedarf, Zugriffszahlen oder Traffic kann auf ein
grösseres Paket aktualisiert werden. Wechsel zu einem kleineren Paket
ist meist nur jährlich möglich. Die Skalierbarkeit ist stark
eingeschränkt. Die Daten können innerhalb der Schweiz gespeichert
werden. Der zuständige Systemtechniker ist meist direkt oder indirekt
kontaktierbar. Spezielle Konfigurationen am Hosting sind nicht möglich.
Die Datencenter sind meist nicht redundant geführt. Fällt das
Datencenter aus ist, die Applikation nicht verfügbar.

\subsection{Cloud Hosting}\label{cloud-hosting}

Die Serverkosten sind direkt von der eigentlichen Nutzung abhängig. Das
Hosting ist skalierbar und kann sich automatisiert an den aktuellen
Nutzungsbedürfnissen anpassen. Die realen Kosten sind im vornherein
schwerer zu definieren. Die Daten sind in der Cloud redundant geführt.
Fällt ein Datencenter aus kann ein anderes dessen Aufgabe übernehmen.
Ein Anbieter der direkt Asp.net Webservice als Hostingservice anbietet
wurde nicht gefunden.\footnote{Stand 18. Dezember 2015} Indirekt über
z.b. über ein Docker Image könnte auch ein Schweizer Anbieter
berücksichtigt werden. Die genutzten Serverdienste können komplett an
seinen eigenen Bedürfnissen angepasst werden.

\subsection{Entscheidung}\label{entscheidung-3}

Die in {[}NFREQ-132{]} geforderte Skalierbarkeit, nutzungsabhängige
Kosten, Freiheit in der Serverdienst-Konfiguration überwiegen der
einfachen Speicherung der Daten in der Schweiz. Ausserdem wird das
einfache publishen (veröffentlichen) einer Web-Application aus dem
Visual Studio bei allen Cloudanbieter angeboten (bei Shared Hosting sind
es nur vereinzelte Anbieter), was den Development Workflow erheblich
unterstützt. Deshalb ist der Authenifizierungsservice im Cloud Hosting
zu betreiben.

\section{Validierung von
Benutzereingaben}\label{validierung-von-benutzereingaben}

NFREQ-126 und die Sicherheit des Authenifizierungsservice verlangen eine
geeignete Validierung der Benutzereingaben. Um Fehlspeicherungen oder
Fehloperationen vorzubeuegen werden alle Daten Vorgängig validiert. Die
Fehlermeldungen sollen falls möglich klar und spezifisch formuliert
werden. Bei den Daten-Klassen/POCO-Klassen werden die gültigen
Wertebereiche mittels Annotationen festgelegt. Microsoft MVC und
Microsoft Web-API stellen eine ``Modelstate.Valid()'' Methode zur
Verfügung welche das angelieferte Datenobjekt automatisch gegen die
Annotationen prüft. Bei MVC Implementierungen werden mittels Microsoft
jQuery Validate Standart Annotationen bereits in der
Benutzereingabemaske überprüft. So muss der User bei Falscheingabe nicht
zuerst einen manuellen Request auf den Server setzten sondern wird
gleich über die Fehleingabe aufmerksam gemacht. Im Konfigurator, eine
AngularJS-App, ist die benutzerseitige Validierung mittels HTML5 Form
Validation umgesetzt worden.

\section{Testing}\label{testing}

Die gewählte Architektur sowie Dependency Injection vereinfachen das
Testing.

\subsection{Wie kann getestet werden?}\label{wie-kann-getestet-werden}

Der Authenifizierungsservice und die Sicherheitstufen können wie normale
Web-Applikationen in MVC oder Web-API getestet werden. Jedes
Sichrheitstufen-Plugin sollunabhängig gekapselt getestet werden. Um in
Unit-Test keine Datenbank zu nutzen soll das Repository Pattern
eingesetzt werden.

\subsection{Was soll getestet werden?}\label{was-soll-getestet-werden}

Grundsätzlich sollte die Logik, welche im Controller ist, getestet
werden. Wird eine spezielle Logik ausserhalb der Controller verwendet,
so soll auch diese getestet werden.

\subsection{Repository Pattern}\label{repository-pattern}

Wie im Kapitel\protect\hyperlink{testing-1}{Testing} beschrieben, sollte
eine Möglichkeit geschaffen werden Datenbanken losgelöst zu testen.
Dafür wir das Repository-Pattern eingesetzt. Das Repository-Pattern
sieht vor, dass jedes POCO-Objekt genau eine Schnittstelle hat, an denen
es die CRUD-Operationen ausführen kann. Im Prinzip eine Schnittstelle,
die auf alle Anfragen an die Datenbank eine passende Reaktion hat. Diese
Schnittstelle oder der Punkt, an welchem Anliegen bearbeitet werden, ist
das Repository. Für beinahe jedes Objekt, was persistiert wird.

Definition des Repository-Patterns von Edward Hieatt and Rob Mee:
``Vermitteler für den Zugriff auf Domänenobjekte zwischen den Domänen-
und Daten-Mapping-Schichten mit Hilfe einer Collectionartigen
Schnittstelle''

Die Vorteile des Patterns sind zum einen die vereinfachten Unit-Tests.
Man kann jedes Repositoryeinfach testen und so auf seine korrekte
Funktionalität überprüfen. Weiter bieten Repositories eine zentrale
Anlaufstelle für Datenbankoperationen. Eine gemeinsame Schnittstelle
gegenüber den Datenhaltungs-Schichten. Zudem bietet es einen idealen
Punkt, an dem man Mechanismen wie beispielsweise Caching implementieren
kann. \footnote{\autocite{repository}}

\chapter{Proof Of Concept}\label{proof-of-concept}

Das Ziel der Implementation des Prototyps ist es, zu zeigen, dass das
Architekturkonzept auch umsetzbar und sinnvoll ist. Des Weiteren wird
dabei die Entscheidung über die Auswahl der geeigneten Technologie
überprüft. Ausserdem hilft der Prototyp, Probleme im Architekturkonzept
zu erkennen und zu beheben.

\section{Techologien}\label{techologien}

Der Auftraggeber möchte dass die aktuell in seinem Betrieb eingesetzten
Technologien für die Implementation der Arbeit verwendet werden. Die
vorgegebenen Technologien sind im folgenden Kapitel erklärt.

\subsection{C-Sharp}\label{c-sharp}

Im Rahmen der Einführung von .net veröffentlichte Microsoft 2002 die
Programmiersprache C-Sharp oder verkürzt C\#. C\# orientiert sich stark
an Java, C++, Haskell und Delphi. Daher liegt es Nahe das C\# eine
objektorientierte Programmiersprache ist und der Wechsel von den zu
vorgenannten Programmiersprachen auf C\# einfach fällt.

Neben Grundprinzipen der objektorientierten Programmierung resultiert
aus folgende innovativen Sprach-Konstrukte eine vereinfachte
Programmierung:

\begin{itemize}
\tightlist
\item
  Gekapselte Methodensignaturen, Delegaten genannt, die typsichere
  Ereignisbenachrichtigungen ermöglichen
\item
  Eigenschaften, die als Accessoren für private Membervariablen dienen
\item
  Attribute, die zur Laufzeit deklarative Metadaten zu Typen
  bereitstellen
\item
  Inline-XML-Dokumentationskommentare
\item
  Sprachintegrierte Abfrage (Language-Integrated Query, LINQ), die
  integrierte Abfragefunktionen für eine Vielzahl von Datenquellen
  bereitstellt
\end{itemize}

Der C\#-Erstellungsprozess ist im Vergleich zu C und C++ einfach und
flexibler als in Java. Es gibt keine separaten Headerdateien und es ist
nicht erforderlich, Methoden und Typen in einer bestimmten Reihenfolge
zu deklarieren. Eine C\#-Quelldatei kann eine beliebige Anzahl von
Klassen, Strukturen, Schnittstellen und Ereignissen definieren.
\footnote{\autocite{csharpbasic}}

\newpage

\subsection{ASP.net Web API 2 / ASP.net MVC
Framework}\label{asp.net-web-api-2-asp.net-mvc-framework}

Microsoft entwickelte mit dem ASP.net MVC Framework ein schlankes und
einfach zu testendes Präsentationsframework. Wie im Namen enthalten
basiert das Framework auf dem MVC-Pattern. Die klare Trennung von
Eingabelogik, Geschftslogik und Präsentationslogik wird durch die vom
Framework bereitgestellten Komponenten unterstützt. Um
RESTful-Webservices einfach entwickeln zu können stellt Microsoft mit
ASP.net Web API 2 eine einfache zu verwendendes und starkes Software
Paket zur Verfügung. ASP.net Web API 2 basiert auf dem ASP.net MVC
Framework. \footnote{\autocite{csharpbasic}}

\subsection{Entity Framework}\label{entity-framework-1}

Entity Framework (EF) ist eine objektrelationale Zuordnung, die
.NET-Entwicklern über domänenspezifische Objekte die Nutzung
relationaler Daten ermöglicht. Ein Grossteil des Datenzugriffscodes, den
Entwickler normalerweise programmieren, muss folglich nicht geschrieben
werden. \footnote{\autocite{efbasic}}

\subsection{Grunt}\label{grunt}

Grunt.js ist ein sogenannter Taskrunner, d.h. es übernimmt Aufgaben wie
das Kompilieren von CSS, überprüft JavaScript auf Fehler ab und
optimiert alle Assets für das Web. Grunt.js zeichnet isch dadurch aus,
dass, bei richtiger Konfiguration, Grunt.js die Daten selbst überwacht
und bei Änderungen die oben genannten Tasks automatisch ausführt.

\subsection{AngularJS}\label{angularjs}

Mittels AngularJS ist die Client-Browser App entwickelt. AngularJS ist
ein Javascript Framework, welches OpenSource von Google Inc.
veröffentlicht wurde. AngularJS macht einen Grossteil des Codes, den man
normalerweise schreibt, überflüssig. Die Reduktion des Codes begründet
sich durch die Automatisierung von Standardaufgaben. Die manuelle
DOM-Selektion, DOM-Manipulation und Event-Behandlung werden durch
AngularJS überflüssig. Durch Einsatz von Direktiven und Modulen wird die
Wiederverwendbarkeit von Code ermöglicht.

Die normalen Datentypen von JavaScript können verwendet werden. Dadurch
ist es sehr einfach möglich, fremde Bibliotheken einzubinden, ohne eine
weitere Zwischenschicht (Glue Code) zu implementieren. Die Methode, die
AngularJS dazu verwendet nennt sich Dirty-Checking und wird im
Vertiefungskapitel näher erklärt.\footnote{\autocite{angularjsbasic}}

\subsection{jQuery}\label{jquery}

jQuery ist ein die meistverwendete Javascript-Bibliothek. jQuery wird
bei 68\% aller Webseiten\footnote{\autocite{w3techweb2015}} eingesetzt.
jQuery stellt unter anderem Funktion zur einfachen DOM-Manipulation,
Event-Behandlung und Ajax-Komunikation zur Verfügung. Entwicklungen von
grösseren Javascript Projekten ist mit jQuery einfacher als mit blankem
Javascript jedoch zeitintensiver als mit Javascript Frameworks wie
AngularJS. Dafür hat jQuery eine höhrere Browserkompatibilität. Die
Kompatibilität der Authentifizierung des Endbenutzer ist wichtig um eine
grosse Verbreitung zu erreichen. Deshalb wird die Authentifizierung des
Endbenutzer mit jQuery umgesetzt.

\subsection{JSON}\label{json}

Zwischen der AngularJS WebApp und dem Webservice dient JSON(JavaScript
Object Notation) als Datenübertragungsformat. JSON zeichnet sich durch
seine schlanke Notation und der objektnahen Darstellung aus.

\hypertarget{entwicklungswerkezeuge}{\subsection{Entwicklungswerkezeuge}\label{entwicklungswerkezeuge}}

Da die Entwicklungssprache C\# verwendet wird, liegt es nahe das
Entwicklungswerkezeug VisualStudio einzusetzen. Der Student hat während
Studium die JetBrains Entwicklungsplattform PHPStorm kennen gelernt.
Daher wird für die Entwicklung der JavaScript-Webapplikationen PHPStorm
eingesetzt.

\newpage

\section{Umsetzung Sicherheitsstufe}\label{umsetzung-sicherheitsstufe}

\subsection{Plugin Entwicklung}\label{plugin-entwicklung}

Die Entwicklung einer Sicherheitstufe wird wie im Konzept
unter\protect\hyperlink{modularituxe4t-und-erweiterbarkeit}{Modularität
und Erweiterbarkeit} vorgesehen losgelöst und unabhängig entwickelt. Pro
Sicherheitstufe werden 3 VisualStudio Projekte angelegt. Im Hauptprojekt
der Sicherheitstufe wird die klassiche Runtimeumgebung für Webprojekte
mit den benötigten Standartreferenzen und Templates für Microsoft MVC
und Microsoft WebAPI aufgesetzt. Das PlugIn kann in diesem Projekt ohne
Authentifizierungsservice entwickelt und ausgeführt werden. Das
Testprojekt stellt die Lauffähigkeit der im Hauptprojekt entwickelten
Implementationen sicher. Um die Entwicklungen im Hauptprojekt als
DLL-Klassenbibliothek zu generieren die ClassLibary-Projekt. In diesem
werden die entwickelten Klassen aus dem Hauptprojekt verlinkt. Bei
vorhandensein aller nötigen Referenzen und verlinkungen erstellt die
ClassLibary bei einem Build die DLL-Klassenbibliothek unser PlugIn.

\begin{figure}[htbp]
\centering
\includegraphics{images/visualstudio_securitystep.png}
\caption{Screenshot VisualStudio der 3 Projekte der Sicherheitsstufe
E-Mail}
\end{figure}

\subsection{Interface - Vertrag mit den
Sicherheitsstufen}\label{interface---vertrag-mit-den-sicherheitsstufen}

Für den Endbenutzer startet der Authentifizierungsprozess mit öffnnen
der Authentifizierung-Lightbox. Dabei wird die Action ``Validate/Check''
des Authentifizierungsservice aufgerufen. Diese zentrale Funktionalität
überprüft den Status der Verifizierung und ruft die nötigen
Sicherheitstufen auf. Für den Endbenutzer ist der Ablauf der
Authentifizierung pro Sicherheitsstufe sichtbar. Der Ablauf und Inhalt
der Authentifzierung jeder Sicherheitsstufe kann individuell erstellt
werden. Einzig der Startpunkt und Endpunkt wird von
Authentifzierungsservice vorgegeben. So muss die Seite bzw. Action
``Index'' in jeder Sicherheitsstufe für den Start der Authentifizierung
der Sicherheitsstufe vorhanden sein. Am Ende der Authentifizierung soll
es wieder zurück zur Action ``Validate/Check'' des
Authentifzierungsservice gehen. Damit die Action ``Validate/Check''
übperüfen kann, ob die Authentifizierung der Sicherheitsstufe
erfolgreich war oder zum ersten oder wiederholten mal ausgeführt werden
sollte, wir die Methode ``checkIsValidated'' pro Sicherheitsstufe
implementiert. Diese Funktion teil basierend auf den übergebenen
Parameter ProjektID und ProviderID mit ob die Validierung erfolgreich
ist. Das MEF-Contracts Interface aller Sicherheitsstufen enthält
ausserdem zwei Methoden zum Abfrage und Speicherung individuellen
Konfiguration der Sichstufen und die Methode zur Abfrage der
Vergleichsparameter.

\begin{figure}[htbp]
\centering
\includegraphics{images/code/ISecurityStep.png}
\caption{ISecurityStep}
\end{figure}

\subsection{Visualisierung}\label{visualisierung}

\section{Auswahl des
Anzeige-Frameworks}\label{auswahl-des-anzeige-frameworks}

Nach Anforderung {[}NFREQ-127{]} und den Kapitel
\protect\hyperlink{mockup}{Mockup} soll die Authentifizierung-Lightbox
responsive umgesetzt werden. Bootstrap unterstützt den Entwickler bei
der visualisierung von Webapplikationen. AngularJS unterstützt seit
Anfang an Bootstrap. Mit dem PlugIn AngularJS Boostrap UI stehen
erweiterte Bootstrap Funktionalitäten wie Datetime-Picker zur Verfügung.
Der Student hat bereits mehrfach Webseiten und Webapplikationen
basierend auf Bootstrap umgesetzt. Deshalb fällt die Auswahl auf das mit
bekannte Responsive-Framework bootstrap. Neben der Responsiven
Unterstützung mit hilfe des Grid-System stehen dem Entwickler umgesetzte
Vorlagen für die meistgenutzten Webkomponenten zur Verfügung. Diese
können dank zentraler Parametrisierung rasch konfiguriert und
individualisiert werden.

\section{Visualisierung von Daten}\label{visualisierung-von-daten}

Um die Umfrageergebnisse visualisieren zu können wird ein
Charting-Framework eingesetzt. Die drei bekannten Charting-Frameworks
GoogleCharts, ChartJs und D3 wurden verglichen. GoogleCharts und D3
visaulisieren in SVG. ChartJS visualisiert in Canvas. Ein eindeutiger
Vorteil der beiden Konzepte für den Authentifizierungs-Konfigurator ist
nicht zu nennen. Alle drei Charting-Frameworks können mit AngularJS
integriert werden. GoogleCharts und ChartJs bieten fixfertige
Direktiven\footnote{Angular ermöglicht es, benutzerdefinierte
  HTML-Elemente und -Attribute, so genannte Direktiven, zu erstellen
  \newpage} an. Damit ist die Integration in AngularJS der beiden
Frameworks im Gegensatz zu D3 direkt möglich. Alle drei
Charting-Framework bieten die benötigten Diagramme an. ChartJs hat das
kleinste Code-Pakat (5KB) und wirkt in den Code deutlich einfacher und
aufgeräumter. Visuell passt ChartJs mit den leichten Animationen am
besten zum Authentifizierungs-Konfigurator. Zur Visualisierung wird
ChartJS verwendet. Die AngularJS-Direktive, der einfache Code, das
kleine Paket und die visuelle Umsetzung führen zu diesem Entscheid.

\section{Finale Screens}\label{finale-screens}

\subsection{AngularJS-Konfigurator}\label{angularjs-konfigurator}

Dieses Kapitel zeigt die finalen Screens des Konfigurators, welcher mit
AngularJS umgestzt wurde. Diese Screens sind abgeleitet von den
Mockups\footnote{Siehe Kapitel
  \protect\hyperlink{konfigurator-template}{Konfigurator Template}}

\ldots{}Hier würden ein paar Screenshots gezeigt werden

Der Programmierer kann bei Auswahl der Sicherheitsstufe die Bewertungen
vom Auftraggeber inaffect AG und die Umfrageergebnisse einsehen.

\ldots{}Hier würden ein paar Screenshots gezeigt werden

\newpage

\subsection{Authentifizierung-Lightbox mit
Sicherheitsstufen}\label{authentifizierung-lightbox-mit-sicherheitsstufen}

Die Authentifizierung-Lightbox mit Sicherheitsstufen wurde für den
Endbenutzer entworfen. Dieses Kapitel zeigt die finalen Screens welche
von den Mockups\footnote{Siehe Kapitel
  \protect\hyperlink{authentifizierungs-lightbox-template}{Authentifizierungs-Lightbox
  Template}} abgeleitet wurden.

\ldots{}Hier würden ein paar Screenshots gezeigt werden

\newpage

\section{Implementation
Authentifizierung}\label{implementation-authentifizierung}

\subsection{Aufruf der Lightbox}\label{aufruf-der-lightbox}

Die Implementation der Authentifizierung ist wie im Kapitel
\protect\hyperlink{integrationskonzept}{Integrationskonzept} festgelegt,
lean umgesetzt worden. Alle CSS-Befehle können von einer Datei abegrufen
werden. Die Javascript-Entwicklungen sind in einem File öffentlich
verfügbar. Um keine Konflikte mit bereits auf der Webseite
implementierten jQuery Bibliotheken zu erhalten wird diese jQuery nicht
im Authentifizierungsjavascript mitgeliefert.

\begin{figure}[htbp]
\centering
\includegraphics{images/code/implementation_lightbox.png}
\caption{HTML-Beispiel Implementation der Authentifizierung}
\end{figure}

\newpage

\subsection{Gegeprüfung der
Authentifizierung}\label{gegepruxfcfung-der-authentifizierung}

Nach Abschluss der Authentifizierung erhält der User visualisiert ein
Feedback. Wie im Kapitel
\protect\hyperlink{schlussspeicherung}{Schlussspeicherung} im
Architekturkonzept beschrieben, wird im Hintergrund ein Post auf die vom
Programmierer angegebene Url ausgeführt. Als Gegenprüfung steht der
Webservice Validate zur Verfügung. Der Webservice wurde implementiert
und kann unter http://iaauth.azurewebsites.net/api/Validate mit den
Parameter ProjectId und ProviderId konsumiert werden.

\subsection{WordPressPlugIn / Erweiterung
WP-Poll}\label{wordpressplugin-erweiterung-wp-poll}

Die Implementation in einem neuerstellten Testprojekt ist erfolgreich.
Das umgesetzte Implentationskonzept soll nun auch in einer bestehenden
Webapplikation integriert werden. Daher soll das verbreitete Umfrage
Modul WP\_Poll aus dem Kapitel
\protect\hyperlink{wordpress-plugin-hook}{Wordpress PlugIn Hook} eine
Implementation der Authentifizierung-Lightbox erhalten. Dafür wurde eine
neue Wordpress installation mit einem Standartlayout aufgesetzt und das
PlugIn integriert. Statt den Code hardkodiert zu integrieren, wurde ein
eigenes PlugIn entworfen dass nun mit minimaler Konfiguration wieder
verwendet werden kann. Die Integration ist erfolgreich integriert und
auf dem github Account verlinkt.\footnote{https://github.com/coffeefan/bachelorarbeit}

Hier würden ein paar Screenshots gezeigt werden

\newpage

\hypertarget{testing-1}{\section{Testing}\label{testing-1}}

\subsection{Unit-Test Sicherheitsstufe und
Authentifizierungsservice}\label{unit-test-sicherheitsstufe-und-authentifizierungsservice}

Die verschiedenen Sicherheitsstufen könnenen unabhängig geprüft werden.
Jede Sicherheitstufe hat ein eigenes Testprojekt. Die verschiedenen
Testprojekte der Sicherheitsstufen und das Testprojekt des
Authentifizierungsservice basieren auf dem Template des Visual Studio
2015 Unit-Test Frameworks. Die Unit-Tests sind direkt im Visual Studio
eingebetet.

\begin{figure}[htbp]
\centering
\includegraphics{images/screenshot_test.jpg}
\caption{Screenshot Unit-Test E-Mail Sicherheitsstufe}
\end{figure}

\newpage

\chapter{Studie}\label{studie}

\section{Definition der Begriffe aus
Aufgabenstellung}\label{definition-der-begriffe-aus-aufgabenstellung}

Während den Besprechungen zur Definition der Anforderungen wurde der
Begriff ``Geschwindigkeit'' aus der Aufgabenstellung diskutiert. Der
Auftraggeber versteht den Begriff der Geschwindigkeit nicht als objektiv
eindeutigen Parameter Zeit sondern als eine subjektive Wahrnehmung.
Dadurch kann nicht wie angenommen, einfach die Zeit die ein
Umfrageteilnehmer zum Anwenden einer Authentifizieren hat, gemessen
werden, sondern die Wahrnehmung muss auch erfragt werden. Während der
Diskussion wurde der Begriff ``Anstrengung'' verwendet. Deshalb wird
auch die Umfrage auf diesem eindeutigeren Begriff ``Anstrengung''
aufgebaut.

\section{Ziel der Studie}\label{ziel-der-studie}

Das Ziel ist den Programmierer bei der Konfiguration des
Authentifizierungsservice mit visualisierten Kennzahlen zu unterstützen.
Der Programmierer soll die Aktzeptanz der Sicherheitsstufen unter
verschiedenen Bedingungen einsehen können und anderseits soll die
empfundene Angstrengung der Benutzer für das Authentifizieren pro
Sicherheitsstufe visualisiert werden.

\section{Art der Studie}\label{art-der-studie}

Wie die Aufgabenstellung und der Auftraggeber fordert, wird eine Studie
in Form einer Umfrage mit Hilfe eines digitalen Fragebogens
durchgeführt. Bevor die Studie aufgebaut wird gilt es sich Vor- und
Nachteile einer schriftlichen Befragungen bewusst zu machen und
basierend auf diesem Wissen die Studie zu planen.

\newpage

\subsection{Vor - und Nachteile schriftlicher
Fragebogen}\label{vor---und-nachteile-schriftlicher-fragebogen}

Schriftliche Befragungen mit Fragebogen können in verschiedenen
Varianten durchgeführt werden. Deshalb unterscheiden sich zwischen den
Varianten gewisse Vor- und Nachteile zu persönlich-mündlichen oder
telefonischen Studie. Es wird versucht, die Möglichkeiten und Grenzen
mit dem grössten gemeinsamen Nenner aufzuführen. Folgende Punkte ergeben
die wichtigsten Vorteile:

\begin{itemize}
\tightlist
\item
  Die Kosten sind geringer. Hippler \footnote{\autocite{hippler}}
  definiert den Richtwert von einem Viertel der Kosten zu einer
  persönlich-mündlichen oder telfonischen Studie.
\item
  Schriftliche Befragungen mit Fragebogen kann in einem relativ kurzen
  Zeitrum realisiert werden
\item
  Dem zu Befragenden kann eine grössere Anonymität gegeben werden
\item
  Verteilung in verschiedene Regionen einfach und zeitnah möglich.
  Insbesondere bei Online Umfrage.
\item
  Einfluss von aussen gering. Zahlreiche Studien\footnote{Studien und
    Erklärungen zu Fremdbestimmung durch François
    Höpflinger\autocite{umfragemethodik}} belegen, dass Personen welche
  eine Studie im Interview die Beantwortung beeinflussen
\item
  Die Antworten der befragten sind durch die Abwesenheit des
  Interviewers und durch die Anonymität ehrlicher. Dieser Punkt ist
  wissenschaftlich jedoch noch ziemlich umstritten. Schnell bezweifelen
  verschiedene Psychologen und Soziologen diesen Umstand. So auch
  Dr.~Reuband in seinem Paper ``Möglichkeiten und Probleme des Einsatzes
  postalischer Befragungen'' \footnote{\autocite{kzfss01}}
\end{itemize}

Diesen Vorteilen stehen auch gewisse Nachteile gegenüber. Die folgenden
Punkte erläutern die wichtigsten Nachteile die verschiedene Varianten
von Fragebögen gemeinsam haben:

\begin{itemize}
\tightlist
\item
  Wenn eine Studie eine zu grosse Nonresponse-Rate hat, ist eine
  Verallgemeinerung der Resultate unzulässig. Kurz die Bachelorarbeit
  würde mit der Studie das Ziel verfehlen. Bei einer schriftlichen
  Studie kann die Ausfallquote aber nicht im Vornherein eingeschätzt
  werden.
\item
  Die Datenerhebungssituation kann nicht kontrolliert oder bestimmt
  werden. Wo und unter welchen Umständen der Fragebogen beantwortet wird
  kann nicht bestimmt und höchstens erfragt werden.
\item
  Nachfragen basierend auf Antworten können nicht spontan gestellt
  werden, sondern müssen im Vornherein geplant werden.
\item
  Bestimmte Bevölkerungsteile werden durch diese Art der Studie
  ausgeschlossen. Zum Beispiel Analphabeten oder bei Onlineumfragen
  Personen mit zu wenig technischem Know-How oder Hardware.
\end{itemize}

\newpage

\subsection{Fazit}\label{fazit-1}

Es gilt also die Vorteile der schriftlichen Fragebogen bei der
Gestaltung der Studie zu nutzen. Der ausgeschlossene Bevölkerungsteil
verfältscht das Ergebniss nicht, da die Zielgruppe für die Umfrage nur
gerade die Personen sind, welche auch tatsächlich an einer Onlineumfrage
teilnehmen können. Die Nonresponse-Rate ist ein Risiko, dass Rechnung
getragen werden muss um nicht eine ungültige Studie zu erhalten. Damit
die Problematik Nonresponse-Rate gering gehalten wird und eine geeignete
Umgebung für die Datenerhebungssituation vorhanden ist, gilt es sich
weiter den korrekten Aufbau einer Studie zu recherchieren.

\subsection{Webapplikation für
Umfrage}\label{webapplikation-fuxfcr-umfrage}

Basierend auf den Empfehlungen\footnote{Die Universitäten sind
  offizielle Kunden von umfrageonline.ch} der Universität St.~Gallen und
der Universätit Freiburg wurde das Schweizer Unternehmen enuvo GmbH mit
ihrer Webapplikation umfrageonline.ch ausgewählt. Umfrageonline stellt
Studenten den Funktionsumfang für Ihre Studien nach Autorisierung
kostenlos zur Verfügung.

\section{Aufbau Gesamtkonzept}\label{aufbau-gesamtkonzept}

``Ein Fragebogen soll als Gesamtkomzept (Einleitung, Hauptteil, Endteil,
Design, Aufmachung) betrachtet werden, in dem die Reihenfolge und die
Struktur der Frage wichtige Einflussfaktoren zur Erlangung korrekter
Daten sind'' \footnote{Zitat vom Institut für webbasierte Kommunikation
  und E-Learning und Gräf et al. 2001 \autocite{fragebogen}}

In den folgenden Abschnitten wird die Theorie für die Entwicklung dieses
Gesamtkonzept abgebildet.

\subsection{Einleitung}\label{einleitung}

Die Einleitung soll die Befragten motivieren an der Studie teilzunehmen
und allgemeine Hinweise zur Studie geben. Die folgenden Fragen wurden
durch das Institut für webbasierte Kommunikation und E-Learning zusammen
getragen \autocite{fragebogen} und und für die Studie dieser
Bachelorarbeit beantwortet:

\subsubsection{Wer wird befragt?}\label{wer-wird-befragt}

Mit Absprache des Auftraggebers soll die die Zielgruppe sind Schweizer
oder Personen welche in der Schweiz wohnen welche Deutsch sprechen und
zwischen 16 und 65 Jahre alt sind sein. Die Angabe begründete der
Auftraggeber dadurch, dass sich darin die Hauptzielgruppen seiner Kunden
wiederspiegelt. Die Teilnehmer sollen die technische Know-How besitzen
an einer Interaktivität teilzunehmen und den Internetzugriff haben.
Dieses minimale technische Know-How werden sie Beweisen indem sie an der
Umfrage teilnehmen können.

\subsubsection{Was ist der Zweck bzw. das Ziel der
Untersuchung?}\label{was-ist-der-zweck-bzw.-das-ziel-der-untersuchung}

Die Studie dient dem Programmierer zur richtigen Konfiguration der
Authentifizierungsmethode für seinen aktuellen Verwendungszweck.

\subsubsection{Was passiert mit den
Ergebnissen?}\label{was-passiert-mit-den-ergebnissen}

Die Ergebnisse werden Programmierer zum Konfigurieren der
Authentifizierungsmethode zur Verfügung gestellt und in der
Bachelorarbeit veröffentlicht.

\subsubsection{Können die Ergebnisse eingesehen
werden?}\label{kuxf6nnen-die-ergebnisse-eingesehen-werden}

Durch Veröffentlichung der Ergebnisse kann besonders Vertrauen und
Wohlwollen gewonnen werden \footnote{\autocite{fragebogen}}. Deshalb
soll das Kapitel Studie der Bachelorarbeit auf Wunsch den Befragten per
E-Mail zugesendet werden.

\subsubsection{Wer führt die Befragung
durch?}\label{wer-fuxfchrt-die-befragung-durch}

ZHAW Student Christian Bachmann im Auftrag der inaffect AG

\subsubsection{Kontakt für Support und
Fragen}\label{kontakt-fuxfcr-support-und-fragen}

Christian Bachmann, bachmch3@students.zhaw.ch

\subsubsection{Wie viel Zeit muss der Befragte
Investieren?}\label{wie-viel-zeit-muss-der-befragte-investieren}

Eine Einschätzung der durchschnittlich benötigen Zeit und Anzahl der
Fragen sollte zur Beginn der Studie genannt werden. Folgend ist das
Diagramm aus der Studie von Bosnjak und Batini \footnote{\autocite{bosnjak}}
abgebildet. Die Erkenntnis aus der Studie zeigt, dass nicht nur unter
dem Motto je kürzer desto besser gehandelt werden sollte. Die Studie ist
jedoch schon 15 Jahren alt und ist deshalb differenzierter zu sehen. Die
Studie der Bachelorarbeit streben einen Aufwand von 8-12 Minuten an.

\newpage

\section{Hauptteil/Fragen}\label{hauptteilfragen}

Offensichtlich stellt der Hauptteil den Löwenanteil des Aufwands dar.

\subsection{Erste Frage Theorie}\label{erste-frage-theorie}

Die erste Frage ist nach Dillman\footnote{\autocite{dillman}} von
grosser Bedeutung. Mit ihr wird Motivation und Einsatz für den ganzen
Fragebogen gesetzt. Diese Frage soll als Interesse und Neugier der
Befragten bewirken.

Das Institut für webbasierte Kommunikation und E-Learning hat dafür aus
verschiedenen Studien die wichtigsten Kriterien für eine erfolgreiche
erste Frage zusammen getragen\footnote{\autocite{fragebogen}}: -
\textbf{Einfache Formulierung} Der Befragte versteht sofort um was es
geht und glaubt daran dass er die Fragen meistern kann - \textbf{Kurze
Beantwortungszeit, keine offenen Fragen} Ein schnelles überwinden der
ersten ``Hürde'' motiviert den Teilnehmer - \textbf{Angstabbauend}
Ängste wie z.b. die des nicht Beantworten können soll abgebaut werden. -
\textbf{Inhaltlich einführen} Die Frage soll in das Thema einführen und
im Idealfall Interesse und Neugier wecken - \textbf{Keine Fragen zur
Person oder zur Ihrem demographischen Eigenschaften}

Es kann Sinn machen eine ``perfekte'' Einstiegsfrage zu erstellen, die
in der Auswertung der Ergebnisse nicht berücksichtigt wird. Sie dient
lediglich die Anforderungen einzuhalten und en Teilnehmer einen
positives Einstiegserlebnis zu vermitteln.

\section{Erste Frage}\label{erste-frage}

Die 1. Frage der Studie dieser Bachelorarbeit:

\begin{verbatim}
Hatten Sie schon einmal das Gefühl, dass an einem Onlinewettbewerb 
gemogelt werden kann?
0 Ja 0 Nein
\end{verbatim}

\newpage

\section{Theorie Fragen}\label{theorie-fragen}

Fragen sollen eine Funktion übernehmen. Dabei schlägt Kleber\footnote{\autocite{kleber92}}
folgende Klassifizierung vor: - Übergangs- und Vorbereitungsfragen für
Themenwechsel, - Ablenkungs- und Pufferfragen zur Minderung von
Ausstrahlungseffekten, - Filterfragen zum Übergehen von eventuell
irrelevanten Fragen, - Rangier- und Konzentrationsfragen zum Auflockern
langer Darstellungen, - Motivationsfragen zur Stärkung des
Selbstvertrauens und Verminderung von Hemmungen, - Kontrollfragen als
Wahrheitskontrolle der Antworten bzw. Sichtbarmachen von Widersprüchen.

Diese Klassifizierung soll helfen den Fragebogen zu gestalten.

\subsubsection{Frageart}\label{frageart}

Bei der Stellung der Frage sollte festgestellt werden welche Art von
Frage gestellt wird. Da sich dadurch die Antwort markantlich
unterscheidet. Folgende 3 Hauptgruppen gibt es -
\textbf{Einstellungsfragen} Dieser Fragestellung bezieht sich auf
``Wunschbarkeit oder negativen bzw. Beurteilung , den Befragte mit
bestimmten Statements verbinden. - \textbf{Verhaltensfragen} Dabei wird
direkt auf das Verhalten des Befragten bezug genommen. Dabei muss
beachtet werden, dass der Befragte sein Verhalten selbst beschreibt.
Einerseits entspricht die Selbstwahrnehmung der Teilnehmer teilweise
nicht der Realität anderseits kann die Antwort auch dem Wunschdenken des
Befragten zugrunde liegen - \textbf{Eigenschftsfragen} Diese
Fragestellung fragt nach den Eigenschaften von Personen. Vielfach sind
es persönliche und demographische Daten.

\subsubsection{Fragetypen}\label{fragetypen}

Die Fragen können generell in zwei Typen unterteilt werden

\textbf{Offene Frage}\\
Der Aufwand bei der Auswertung ist sehr hoch. Ungeübte Teilnehmer können
unverwertbare Antworten niederschreiben. Antworten sind schwer
vergleichbar Dafür Teilnehmer kann sich so ausdrücken wie er möchte. Er
wird nicht durch vorgegebene Antworten beeinflusst.

\textbf{Geschlossene Frage}\\
Die geschlossene Frage kann leicht ausgewertet werden. Die Gefahr
besteht, dass der Teilnehmer ratet und durch die Antworten beeinflusst
wird. Der Vorbereitungsaufwand für die Frage ist hoch.
Auswahlmöglichkeiten für die Antwort könnten irelevant sein.

\newpage

\section{Fragen über Aktzeptanz}\label{fragen-uxfcber-aktzeptanz}

Es wird die folgende Hypothesen verfolgt: ``Die Akzeptanz von
Sicherheitsstufe ist nicht beständig. Sie ist abhängig von den
Bedienungen der Interkativität: Seriosität des Anbieters, Wichtigkeit
der Umfrage und möglicher Verdienste bei der Teilnahme'' Der Programmier
soll bei der Konfiguration das Umfeld der Interaktivität kategorisieren
können. Die Hauptbereiche sind aus der Aufgabenstellung entnommen. Die
anderen Kategorisierungen ergeben sich aus der Thesis.

\begin{itemize}
\item
  Voting

  \begin{itemize}
  \item
    einfache
  \item
    Casting
  \end{itemize}
\item
  Wettbewerb

  \begin{itemize}
  \item
    Seriöse Firma

    \begin{itemize}
    \item
      Gewinn unter 200 Franken
    \item
      Gewinn über 200 Franken
    \end{itemize}
  \item
    Unbekante Firma

    \begin{itemize}
    \item
      Gewinn unter 200 Franken
    \item
      Gewinn über 200 Franken
    \end{itemize}
  \end{itemize}
\end{itemize}

Aus jeder Kategorie wird in der Studie erfragt welche Sicherheitsstufen
eingesetzt der Umfrageteilnehmer einsetzen würde. Es wird pro Kategorie
eine geschlossene Frage gestellt. Der Fragetyp ist Mehrfachauswahl. Die
Fragen sind von der Klassifizierung Verhaltensfragen. Es wird abgeklärt,
unter welchen Bedienungen sich der User so verhält, dass er die
Sicherheitsstufe akzeptiert. Der User kann pro Kategorie die
Sicherheitsstufen auswählen welche er bereit ist zu verwenden.

\begin{figure}[htbp]
\centering
\includegraphics{images/umfrage_aktzeptanz_beispiel.jpg}
\caption{Screenshot einer Akzeptanzfrage}
\end{figure}

\newpage

\section{Frage Anstrengung}\label{frage-anstrengung}

Die verschiedenen Sicherheitsstufen sollen für den User direkt
vergleichbar beantwortet werden können. Dafür eignet sich eine
Geschlossene Frage vom Type Bewertungsmatrix. Es wurden fünf Abstufungen
zur Einschätzung der Anstrengung definiert. Ausserdem kann der User bei
Unsicherheit keine Antwort geben. Diese Frage ist eine
Einstellungsfragen. Der User gibt bewertet seine Einstellung zu den
Sicherheitsstufe anhand der Anstrengung.

\begin{figure}[htbp]
\centering
\includegraphics{images/umfrage_anstrengung.jpg}
\caption{Screenshot der Umfrage zur Anstrengung}
\end{figure}

\section{\texorpdfstring{Bonus ``Frage'', Umgehung der
Absicherung}{Bonus Frage, Umgehung der Absicherung}}\label{bonus-frage-umgehung-der-absicherung}

Umfrageonline.ch enthält die Sicherheitsstufe Cookie und IP-Adresse.
Wobei die Sicherheitsstufe IP-Adresse standardmässig deaktiviert ist.
Diese beiden Sicherheitstufen, erlauben es wie mehrfach in dieser
Bachelorarbeit dokumentiert, mehrfach an einer Umfrage teilzunehmen. Die
Hypothese wird aufgestellt, dass ein Teilnehmer mit genügend technischem
Know-How insbesondere bei diesem Umfragethema mehrfach teilnehmen wird.

\section{Weitere Fragen}\label{weitere-fragen}

Weiter werden die 3 Eigenschftsfragen gestellt. Dabei soll Alter,
Geschlecht und ob es sich um einen Schweizer oder Bewohner der Schweiz
handelt angegeben werden.

\section{Abschluss}\label{abschluss}

Der Abschluss des Fragebogens kann sehr kurz gehalten werden. Folgende
Elemente sollten enthalten sein:

\subsubsection{Dankensformel}\label{dankensformel}

Eine kurze Dankensformel gehört zum guten Ton und motiviert den
Teilnehmer die Umfrage korrekt abzuschliessen.

\subsubsection{Einladung zur
Kommentierung}\label{einladung-zur-kommentierung}

Durch Kommentare am Schluss können Befragte dem Untersuchter Hinweise
zukommen lassen die für die Auswertung und weitere Untersuchungen
dienlich sind. Dieser Möglichkeit wird nach der Erfahrung von
Reuband\footnote{\autocite{kzfss01}} gewürdigt.

\section{Verständlichkeit}\label{verstuxe4ndlichkeit}

Als der Umfragebogen Personen mit geringem technischem Know-How
vorgelegt wurde, wurde klar das die genannten Authentifizierungsmethoden
nicht bekannt sind. Selbst der Begriff Authentifizieren konnte nicht
erklärt werden. Deshalb wurden die zu analysierenden Methoden erklärt
und illustriert.

\begin{figure}[htbp]
\centering
\includegraphics{images/studien-ilustrationen.jpg}
\caption{Beispiele der Illustrationen für die Umfrage}
\end{figure}

\section{Auswertung}\label{auswertung}

Die Umfragedaten werden in den entworfenen Authentifizierungsservice
eingespielt. Jeder Programmierer kann während dem Konfigurieren seiner
Sicherheitsstufe die gewünschten Diagramme zusammenstellen. Anhand der
visualisierten Daten kann er die Meinung seiner Enduser einschätzen und
optimaler für den Enduser seine Konfigurationen wählen. Damit ist das
Ziel der Studie erreicht und die Möglichkeit der Auswertung erreicht.

Weiter werden noch einige Anmerkungen zu den Fragen erläutert.

\subsection{Repräsentativität}\label{repruxe4sentativituxe4t}

Laut Bundesamt für Statistik ist enthält die definierte Zielgruppe 3.3
Milionen Personen\footnote{\autocite{bfs}}. Dies Zahl beinhaltet die
Deutschweizer welche zwischen 16 und 65 Jahre alt sind. An der Umfrage
habe 176 Personen teilgenommen. Daraus lässt sich ein Konfidenzinterval
von 7,4\% errechnen.

\begin{figure}[htbp]
\centering
\includegraphics{images/repraesentativitaet.png}
\caption{Berechnungsformel für Repräsentativität}
\end{figure}

n= Strichprobengrösse=176 Umfrageteilnehmer

N= Grundgesamtheit = 3.3 Milionen Menschen in der Zielgruppe

p= Stichprobenanteil bei Normalverteilung

q= 1-p (Vereinfachte Darstellung)

t= Normalverteilungsnormierung 1,96 = 95\% Trefferquote

d= Gesuchter Wert, das Konfidenzintervals Fehlertoleranzwert= 7,4\%

\subsection{Gemogelt an Wettbewerben}\label{gemogelt-an-wettbewerben}

Über 65\% der Befragten gehen davon aus, dass sie noch nie an einem
Wettbewerb teilgenommen haben, an welchem gemogelt hätte können. Bei den
40-65 jährigen sind es sogar über 83 Prozent. Die Einstiegsfrage, welche
zur Einführung ins Thema gedacht ist, zeigt überraschend ein hohes
Vertrauen in Onlinewettbewerbe.

\begin{figure}[htbp]
\centering
\includegraphics{images/studie/mogeln.jpg}
\caption{Ergebnise Frage zu mogeln an Onlinewettbewerben}
\end{figure}

\subsection{\texorpdfstring{Bonus ``Frage'', Umgehung der
Absicherung}{Bonus Frage, Umgehung der Absicherung}}\label{bonus-frage-umgehung-der-absicherung-1}

3 Umfrageteilnehmer konnte mit den zur Verfügung stehenden technischen
Mittel als Mehrfachteilnahme registriert werden. Die Bonusfrage wurde
wie angenommen gelöst. Die Thematik der Umfrage bewegt die Teilnehmer
offensichtlich zum Ausprobieren. 1 Teilnehmer brauchte nach Ende seiner
1. Teilnahme genau 17 Sekunden bis er erneut mit der Umfrage starten
konnte.

\section{Anstrengung}\label{anstrengung}

Auf einer 5 stufigen Skala sind die Sicherheitsstufen nach Anstrengung
bewertet: 1 Punkt für sehr anstrengend, 5 Punkte für sehr angenehm.
Anhand dieser Punkte konnte nun ein arithmetisches Mittel errechnet
werden. Das empfinden der Anstrengung ist bei allen Teilnehmer ähnlich
feststellbar und mit einer Standartabweichung von 0.85 bis 1.1 Punkte
festgelegt. Dabei ist feststellbar, dass unser Authentifizierungsservice
Sicherheitsstufen mit angenehmem empfinden bis Sicherheitsstufen mit
sehr anstrengendem empfinden zur Verfügung stellt. Die gewünschte Breite
des Arbeitgebers konnte auch in diesem Aspekt gefunden werden.

\begin{figure}[htbp]
\centering
\includegraphics{images/studie/anstrengung.jpg}
\caption{Überischt der Ergebnisse zur Umfrage der Anstrengung mit
Arithmetischem Mittel und Standardabweichung}
\end{figure}

\section{Aktzeptanz}\label{aktzeptanz}

Die Hypothese, dass die Aktzeptanz zum Einsatz von Sicherheitsstufen mit
Seriösität des Anbieters, Wichtigkeit und möglichen Verdienst verändert
zeigt das Umfrageergebnis in allen Altersstufen. Interessant ist, dass
die Akzeptanz von einem automatischen Telefonanruf geringer ist wie die
Aktzeptanz einer SMS von einer Mobilennumer. Diese Erkenntnis kann auf
alle Fragen angewendet werden, ist also unabhängig von Bedienungen der
Interaktivitäten. Die Angabe seiner eigenen Mobilenummer wird dem
mühsamen abtippen des Anrufes auf ein mögliches Fixnettelefon
vorgezogen. Der Pass oder die ID-Nummer wird nur bei einem
vertrauenswürdigen Unternehmen angegeben. Der zu erhoffende Gewinn hat
keinen bedeutenden Einfluss. Bei unbekannten Unternehmen als Anbieter,
würden die grosse Mehrheit der Teilnehmer ausser Captcha und
E-Mail-Adresse keine Sicherheitsstufe verwenden. Keine andere
Sicherheitsstufe konnte bei diesem Anbieter bei mehr als einem 1/5 der
Teilnehmer Akzeptanz erhalten. Der grosse Unterschied bei den
Ergebnissen macht nicht der mögliche Verdienst oder die Wichtigkeit des
Resultats. Vielmehr ist es der Anbieter und das Vertrauen das der
Endbenutzer diesem geben kann.

\chapter{Fazit}\label{fazit-2}

\newpage

\appendix

\hypertarget{glossar}{\chapter{Glossar}\label{glossar}}

\textbf{2FA} 2FA bedeutet Zwei-Faktor-Authentifizierung. Weitere Infos
im Kapitel {[}AuthentifizierungsKompononten{]}

\textbf{Github} Github ist ein Cloud basierter SourceCode
Verwaltungsdienst für Git. \url{https://github.com}

\textbf{CRUD-Operationen} CRUD steht für Create, Read, Update, Delete.
Diese 4 Operationen sind die Grundlage für alle Interkationen mit der
Datenbank.

\textbf{Non-Response} Nichtbeantwortung einer oder mehrerer einzelner
Fragen. Die Repräsentativität einer Befragung hängt stark ab von der
Rücklaufquote, auch Response-Rate genannt.

\textbf{ORM} ORM steht für object-relational mapping und ist eine
Technik mit der Objekte einer Anwendung in einem relationalen
Datenbanksystem abgelegt werden kann.

\textbf{POCO POCO Klassen} POCO ist die Abkürzung für Plain Old CLR
Object. Eine POCO-Klasse ist ein ganz normales .NET-Objekt, das keine
durch die Infrastruktur bedingte Basisklasse, Annotationen oder eine
Enhancement auf Bytecode-Ebene (MSIL/CIL) erfordert.Damit ist es
geeignet schlank Daten zu transportieren.

\textbf{REST / Restfull} REST steht für Representational State Transfer.
REST ist eine Software Architektur des Webs. System welche die REST
Architektur einhalten nennt man RESTful. REST System kommunizieren
allgemein über das HTTP-Protokoll und nutzen die gleichen HTTP verbs wie
ein Browser der eine Webseite abfragt. Neben GET und POST werden die
weniger bekannten Verben PUT und Delete verwendet. Die URI beschreibt
die zu beziehende oder verändernde Webressource.

\textbf{Sicherheitsstufe} Das Wort Sicherheitstufe ist ein
domänenspezifische Beschreibung eines einzelnen
Authentifizierungsvorgang auch Authentifizierungsart genannt.

\chapter{Verzeichnisse}\label{verzeichnisse}

\section{Abbildungsverzeichnis}\label{abbildungsverzeichnis}

\renewcommand{\listfigurename}{} 

\begingroup\let\clearpage\relax
\listoffigures
\endgroup

\pagebreak

\section{Quellenverzeichnis}\label{quellenverzeichnis}

\renewcommand{\bibname}{}

\begingroup \let\clearpage\relax
\printbibliography
\endgroup

\pagebreak

\section{Tabellenverzeichnis}\label{tabellenverzeichnis}

\renewcommand{\listtablename}{} 

\begingroup \let\clearpage\relax
\listoftables
\endgroup

\newpage

\chapter{Danksagung}\label{danksagung}

\newpage

\chapter{Personalienblatt}\label{personalienblatt}

Name, Vorname Bachmann, Christian Adresse Bahnhofstrasse 2 Wohnort 8355
Aadorf Geboren 5. September 1986 Heimatort Feusisberg

\newpage

\chapter{Bestätigung}\label{bestuxe4tigung}

Hiermit bestätigt der Unterzeichnende, dass die Bachelorthesis mit dem
Thema ``Individuell Konfigurierbarer Authentifizierungsservice für
Votings und Wettbewerb'' gemäss freigegebener Aufgabenstellung ohne jede
fremde Hilfe im Rahmen der gültigen Reglements selbständig ausgeführt
wurde. Alle öffentlichen Quellen sind als solche kenntlich gemacht. Die
vorliegende Arbeit ist in dieser oder anderer Form zuvor nicht zur
Begutachtung vorgelegt worden.

Aadorf den 10.05.2016

Christian Bachmann





\end{document}