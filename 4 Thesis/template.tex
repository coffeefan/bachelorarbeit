%%%%%%%%%%%%%%%%%%%%%%%%%%%%%%%%%%%%%%%%%%%%%%%%%%%%%%%%%%%%%%%%%
%  _____   ____  _____                                          %
% |_   _| /  __||  __ \    Institute of Computitional Physics   %
%   | |  |  /   | |__) |   Zuercher Hochschule Winterthur       %
%   | |  | (    |  ___/    (University of Applied Sciences)     %
%  _| |_ |  \__ | |        8401 Winterthur, Switzerland         %
% |_____| \____||_|                                             %
%%%%%%%%%%%%%%%%%%%%%%%%%%%%%%%%%%%%%%%%%%%%%%%%%%%%%%%%%%%%%%%%%
%
% Project     : LaTeX doc Vorlage für Windows ProTeXt mit TexMakerX
% Title       : 
% File        : header.tex Rev. 00
% Date        : 23.4.12
% Author      : Remo Ritzmann
% Feedback bitte an Email: remo.ritzmann@pfunzle.ch
%
%%%%%%%%%%%%%%%%%%%%%%%%%%%%%%%%%%%%%%%%%%%%%%%%%%%%%%%%%%%%%%%%%

\documentclass[ oneside,openright,titlepage,numbers=noenddot,headinclude,%1headlines,% letterpaper a4paper
                BCOR=5mm,paper=a4,fontsize=11pt,%11pt,a4paper,%
                ngerman,american,%
                ]{scrreprt}

%***********************************************************************
% include some libs
%***********************************************************************
\usepackage[utf8]{inputenc}
\usepackage{listings}
\usepackage{color}
\usepackage{fancyhdr}
\usepackage{rotating}
\usepackage{titlesec}
\usepackage{mathptmx}
% \usepackage{helvet}
\usepackage[scaled]{uarial}
\renewcommand*\familydefault{\sfdefault} %% Only if the base font of the document is to be sans serif
\usepackage[T1]{fontenc}
\usepackage{ngerman}
\usepackage{textgreek}
\usepackage{textcomp}
\usepackage[squaren]{SIunits}
\usepackage{graphicx}
\usepackage{url}
\usepackage{geometry}
\usepackage[absolute]{textpos}
\usepackage{makeidx}
\usepackage{colortbl}
\usepackage{pdflscape}
\usepackage{pdfpages}
\usepackage{tabularx}
\usepackage{lmodern}
\usepackage{longtable}
\usepackage{array}
\usepackage{float}
\usepackage{scrhack}
\usepackage[plainpages=false]{hyperref}
\usepackage{wallpaper} %\ThisTileWallPaper{}
%\usepackage[super,square]{natbib} für BibTeX Literaturverzeichnis
\usepackage{enumitem}
\usepackage{subfig}
\usepackage[export]{adjustbox}
\usepackage{setspace}
\usepackage{amsmath} 
\usepackage{fancyvrb}
\usepackage{graphicx}
\usepackage{booktabs}

% Bibliography
\usepackage[
backend=bibtex,
style=alphabetic-verb,
citestyle=alphabetic-verb
]{biblatex}
\bibliography{./Citer}

\usepackage{calc}



%***********************************************************************
% various styles
%***********************************************************************	

%create index
\makeindex

%define pagestyle
\pagestyle{fancy}

%use sans-serif font 
%\renewcommand{\familydefault}{\sfdefault}

%define page margin
\geometry{a4paper, top=30mm, left=30mm, right=30mm, bottom=30mm,headsep=10mm,footskip=10mm}

%textpos parameter
\setlength{\TPHorizModule}{30mm}
\setlength{\TPVertModule}{\TPHorizModule}
\textblockorigin{10mm}{10mm} % start everything near the top-left corner
\setlength{\parindent}{0pt}

%horizontal lines for titlepage 
\newcommand{\HRule}{\rule{\linewidth}{0.5mm}}

%reference to source items inlc source number
\newcommand{\srcref}[1]{\nameref{src:#1} \cite{#1}}

%header / footer 
\renewcommand{\headrulewidth}{0.3pt}
\renewcommand{\footrulewidth}{0.3pt}

\fancyhead[LO,RE]{} %clear headings for contents 

\fancyhead[RO,LE]{\nouppercase{\rightmark}} %right odd pages and left even pages
\fancyhead[LO,RE]{\MakeUppercase{\leftmark}} %left odd pages and right even pages
%\fancyhead[LO,RE]{\fontsize{9}{9}}
\fancyfoot[LE,RO]{\thepage} %page numbering
\fancyfoot[C]{} %clear centered page numbering 

%define some colors
\definecolor{gray}{rgb}{0.95,0.95,0.95}
\definecolor{darkgray}{rgb}{0.4,0.4,0.4}
%listing colors
\definecolor{lgray}{RGB}{250,250,250}
\definecolor{lgreen}{RGB}{63,127,95}
\definecolor{lred}{RGB}{127,0,85}
\definecolor{lblue}{RGB}{42,0,255}

%***********************************************************************
% listing
%***********************************************************************

\lstset{		
		basicstyle=\small\ttfamily,
		frame=single,
		numbers=left,	
		numberstyle=\tiny,
		%firstnumber=auto,
		numberblanklines=true,
		captionpos=b,
		extendedchars=true,
		float=ht,
		showtabs=false,
		tabsize=2,
		showspaces=false,
		showstringspaces=false,
		breaklines=true,
		%prebreak=\Righttorque,
		backgroundcolor=\color{lgray},
		keywordstyle=\color{lred}\bfseries, 
		commentstyle=\color{lgreen}\ttfamily,
%		morekeywords={printstr, printhexln},
		stringstyle=\color{lblue},
		xleftmargin=0.5cm,
		xrightmargin=0.5cm
}

\lstloadlanguages{R}

%\lstdefinelanguage{xc}{
%     keywords={printstr, printhexln, attributes, class, classend, do, empty, endif, endwhile, fail, function, functionend, if, implements, in, inherit, inout, not, of, operations, out, return, set, then, types, while, use},
%     keywordstyle=\color{lred}\bfseries,
%     ndkeywords={},
%     ndkeywordstyle=\color{yellow}\bfseries,
%     identifierstyle=\color{black},
%     sensitive=false,
%     comment=[l]{//},
%     commentstyle=\color{lgreen}\ttfamily,
%     string=[l]{"},
%     stringstyle=\color{lblue}\ttfamily
%  }




\begin{document}

%%%%%%%%%%%%%%%%%%%%%%%%%%%%%%%%%%%%%%%%%%%%%%%%%%%%%%%%%%%%%%%%%
%  _____   ____  _____                                          %
% |_   _| /  __||  __ \    Institute of Computitional Physics   %
%   | |  |  /   | |__) |   Zuercher Hochschule Winterthur       %
%   | |  | (    |  ___/    (University of Applied Sciences)     %
%  _| |_ |  \__ | |        8401 Winterthur, Switzerland         %
% |_____| \____||_|                                             %
%%%%%%%%%%%%%%%%%%%%%%%%%%%%%%%%%%%%%%%%%%%%%%%%%%%%%%%%%%%%%%%%%
%
% Project     : LaTeX doc Vorlage für Windows ProTeXt mit TexMakerX
% Title       : 
% File        : titlepage.tex Rev. 01
% Date        : 23.4.12
% Author      : Remo Ritzmann
% Feedback bitte an Email: remo.ritzmann@pfunzle.ch
%
%%%%%%%%%%%%%%%%%%%%%%%%%%%%%%%%%%%%%%%%%%%%%%%%%%%%%%%%%%%%%%%%%

\begin{titlepage}

% Logo
\ThisTileWallPaper{\paperwidth}{\paperheight}{images/logos/SoE.pdf} % {images/logos/*.pdf}
% Wählen Sie aus folenden pdf Files: ICP, IDP, IEFE, IMES, IMPE, IMS, INE, InES, InIT, KSR, SoE, ZAMP, ZAV, ZIL, ZPP, ZSN

\begin{minipage}[b]{0.117\textwidth}
\hskip 0.05cm
\end{minipage}
\begin{minipage}[b]{0.91\textwidth}
\begin{tiny}.\end{tiny}\vskip 2.8cm
	{\huge
	
	% Projekt Name
	\textbf{\underline{Bachelorarbeit (Informatikingenieurwesen)}}\\
	%\textbf{\underline{ }}
	
	% Projekt Titel
	Individuell Konfigurierbarer Authentifizierungsservice für Votings und Wettbewerbe
	\vskip 3.5cm}
	
	\begin{minipage}[b]{0.27\textwidth}
	\hrule\vskip 0.5cm
		\textbf{Autor}\\
		\\
	\end{minipage}
	\begin{minipage}[b]{0.03\textwidth}
	\hskip 0.5cm
	\end{minipage}
	\begin{minipage}[b]{0.7\textwidth}
	\hrule\vskip 0.5cm
	    Christian Bachmann\\
	    \\
	\end{minipage}
	
	\begin{minipage}[b]{0.27\textwidth}
	\hrule\vskip 0.5cm
		\textbf{Betreuung}\\
		\\
	\end{minipage}
	\begin{minipage}[b]{0.03\textwidth}
	\hskip 0.5cm
	\end{minipage}
	\begin{minipage}[b]{0.7\textwidth}
	\hrule\vskip 0.5cm
		Jaime Oberle\\
		 \\
	
	\end{minipage}
	
%	\begin{minipage}[b]{0.27\textwidth}
%	\hrule\vskip 0.5cm
%		\textbf{Nebenbetreuung}\\
%		\\
%	\end{minipage}
%	\begin{minipage}[b]{0.03\textwidth}
%	\hskip 0.5cm
%	\end{minipage}
%	\begin{minipage}[b]{0.7\textwidth}
%	\hrule\vskip 0.5cm
%		 keine\\
%		 \\
%	\end{minipage}
	
	\begin{minipage}[b]{0.27\textwidth}
	\hrule\vskip 0.5cm
		\textbf{Auftraggeber}\\
		\\
	\end{minipage}
	\begin{minipage}[b]{0.03\textwidth}
	\hskip 0.5cm
	\end{minipage}
	\begin{minipage}[b]{0.7\textwidth}
	\hrule\vskip 0.5cm
		inaffect AG\\
		\\
	\end{minipage}
	
%	\begin{minipage}[b]{0.27\textwidth}
%	\hrule\vskip 0.5cm
%		\textbf{Externe Betreuung}\\
%		\\
%	\end{minipage}
%	\begin{minipage}[b]{0.03\textwidth}
%	\hskip 0.5cm
%	\end{minipage}
%	\begin{minipage}[b]{0.7\textwidth}
%	\hrule\vskip 0.5cm
%		Elmer Melanie\\
%		Hürlimann Erich\\
%	\end{minipage}
	
	\begin{minipage}[b]{0.27\textwidth}
	\hrule\vskip 0.5cm
		\textbf{Datum}
	\end{minipage}
	\begin{minipage}[b]{0.03\textwidth}
	\hskip 0.5cm
	\end{minipage}
	\begin{minipage}[b]{0.7\textwidth}
	\hrule\vskip 0.5cm
		23.12.2015
	\end{minipage}
\end{minipage}
\vskip 0.5cm


%\textcolor{darkgray}{
%Bitte füllen Sie das Titelblatt aus und berücksichtigen Sie Folgendes:\\
% -> Bitte auf keinen Fall Schriftart und Schriftgrösse ändern. Text soll lediglich überschrieben werden!\\
% -> Bitte pro Tabellenzeile max. 4 Textzeilen!\\
%\\
%•	Vorlage: Haben Sie die richtige Vorlage gewählt? Logo Institut/Zentrum\\
%•	Titel: Fügen Sie Ihren Studiengang direkt nach dem Wort „Bachelorarbeit“ ein (max. 2 Zeilen).\\
%•	Titel der Arbeit: Überschreiben Sie den Lauftext mit dem Titel Ihrer Arbeit (max. 4 Zeilen).\\
%•	Autoren: Tragen Sie Ihre Vor- und Nachnamen ein (alphabetisch nach Name).\\
%•	Betreuer: Tragen Sie Ihren Betreuer / Ihre Betreuer ein (alphabetisch nach Name).\\
%•	Ohne Nebenbetreuung, Industriepartner oder externe Betreuung, ganze Tabellenzeile löschen.\\
%•	Datum: Aktuelles Datum eintragen.\\
%•	Am Schluss löschen Sie den ganzen Beschrieb (grau) und speichern das Dokument als pdf. ab.
%}


\end{titlepage}


$if(graphics)$

% We will generate all images so they have a width \maxwidth. This means
% that they will get their normal width if they fit onto the page, but
% are scaled down if they would overflow the margins.
\makeatletter
\def\maxwidth{\ifdim\Gin@nat@width>\linewidth\linewidth
\else\Gin@nat@width\fi}
\makeatother
\let\Oldincludegraphics\includegraphics
\renewcommand{\includegraphics}[1]{\Oldincludegraphics[width=\maxwidth]{#1}}
$endif$

\VerbatimFootnotes

\setlength{\parindent}{0pt}
\setlength{\parskip}{6pt plus 2pt minus 1pt}
\setlength{\emergencystretch}{3em}  % prevent overfull lines
\providecommand{\tightlist}{%
  \setlength{\itemsep}{0pt}\setlength{\parskip}{0pt}}
\setcounter{secnumdepth}{0}
\VerbatimFootnotes % allows verbatim text in footnotes


\author{Max Muster}
\title{Mein erstes Dokument}
\maketitle
\tableofcontents

\section{Ein Abschnitt}
Dieser Abschnitt hat ein wenig Text, und zwar gerade so viel, dass
es auch zu einem \emph{Zeilenumbruch} kommt. 

Ein zweiter Absatz wird durch mindestens eine Leerzeile vom ersten
getrennt.

\section{Noch ein Abschnitt}
"`Wunderbar"', sagte er dazu, "`das sehe ich zum 1.~Mal!"'

\subsection{Ein Unterabschnitt}
Damit kann man tolle Sachen machen:

$body$
$for(include-after)$
  $include-after$
$endfor$

\end{document}